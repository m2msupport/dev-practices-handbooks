\chapter{Model Deployment and Serving}

\section{Introduction}

Model deployment transforms experimental code into production systems serving millions of predictions daily. A model with 95\% accuracy in development becomes worthless if it cannot handle production load, lacks proper error handling, or experiences downtime during updates. The gap between a trained model and a reliable production service is where most ML projects fail.

\subsection{The Deployment Challenge}

Consider a recommendation system that performs excellently in notebooks but crashes under production load, serves stale predictions after model updates, and requires 30 minutes of downtime for each deployment. These are not edge cases—they are the norm for teams without disciplined deployment practices.

\subsection{Why Deployment Engineering Matters}

Studies show that:
\begin{itemize}
    \item \textbf{87\% of ML models} never make it to production
    \item \textbf{50\% of deployed models} experience service degradation in first month
    \item \textbf{Deployment failures} cost companies \$300K+ in lost revenue and engineering time
    \item \textbf{Manual deployment processes} introduce 10x more errors than automated pipelines
\end{itemize}

\subsection{Chapter Overview}

This chapter provides production-ready deployment frameworks:

\begin{enumerate}
    \item \textbf{Model Serving API}: FastAPI integration with validation and error handling
    \item \textbf{Containerization}: Docker multi-stage builds and resource management
    \item \textbf{Deployment Strategies}: Blue-green, canary, and rolling deployments
    \item \textbf{Auto-scaling}: Load balancing and horizontal pod autoscaling
    \item \textbf{Model Versioning}: Registry integration and rollback procedures
    \item \textbf{Monitoring}: Health checks, readiness probes, and performance metrics
\end{enumerate}

\section{Model Serving API with FastAPI}

Production ML services require robust APIs with request validation, error handling, and comprehensive logging.

\subsection{Model Service Foundation}

\begin{lstlisting}[language=Python, caption={Production Model Service with FastAPI}]
from dataclasses import dataclass
from typing import Dict, List, Optional, Any, Union
from enum import Enum
from pathlib import Path
import logging
from datetime import datetime
import numpy as np
import joblib
import json

from fastapi import FastAPI, HTTPException, Request, status
from fastapi.responses import JSONResponse
from fastapi.middleware.cors import CORSMiddleware
from pydantic import BaseModel, Field, validator
import uvicorn

logger = logging.getLogger(__name__)

class ModelStatus(Enum):
    """Model loading status."""
    UNLOADED = "unloaded"
    LOADING = "loading"
    READY = "ready"
    ERROR = "error"

class PredictionRequest(BaseModel):
    """
    Validated prediction request schema.

    Uses Pydantic for automatic validation and documentation.
    """
    features: Dict[str, Union[float, int, str]] = Field(
        ...,
        description="Feature dictionary with feature names as keys",
        example={"age": 35, "income": 50000, "city": "NYC"}
    )
    model_version: Optional[str] = Field(
        None,
        description="Specific model version to use (defaults to latest)"
    )
    return_probabilities: bool = Field(
        False,
        description="Return class probabilities instead of labels"
    )
    explain: bool = Field(
        False,
        description="Include prediction explanation (SHAP values)"
    )

    @validator('features')
    def validate_features(cls, v):
        """Validate features are not empty."""
        if not v:
            raise ValueError("Features dictionary cannot be empty")
        return v

    class Config:
        """Pydantic configuration."""
        schema_extra = {
            "example": {
                "features": {
                    "age": 35,
                    "income": 50000,
                    "credit_score": 720,
                    "loan_amount": 25000
                },
                "return_probabilities": True,
                "explain": False
            }
        }

class PredictionResponse(BaseModel):
    """Validated prediction response schema."""
    prediction: Union[int, float, str, List[float]]
    model_version: str
    prediction_id: str
    timestamp: datetime
    latency_ms: float
    probabilities: Optional[Dict[str, float]] = None
    explanation: Optional[Dict[str, float]] = None
    metadata: Optional[Dict[str, Any]] = None

class HealthResponse(BaseModel):
    """Health check response."""
    status: str
    model_status: str
    model_version: str
    uptime_seconds: float
    predictions_served: int
    avg_latency_ms: float

@dataclass
class ModelMetrics:
    """Runtime metrics for model service."""
    predictions_served: int = 0
    total_latency_ms: float = 0.0
    errors: int = 0
    start_time: datetime = None

    def __post_init__(self):
        if self.start_time is None:
            self.start_time = datetime.now()

    @property
    def avg_latency_ms(self) -> float:
        """Calculate average prediction latency."""
        if self.predictions_served == 0:
            return 0.0
        return self.total_latency_ms / self.predictions_served

    @property
    def uptime_seconds(self) -> float:
        """Calculate service uptime."""
        return (datetime.now() - self.start_time).total_seconds()

class ModelService:
    """
    Production model serving service.

    Features:
    - Model loading and versioning
    - Request validation
    - Error handling and logging
    - Performance monitoring
    - Health checks
    """

    def __init__(
        self,
        model_path: Path,
        model_name: str = "model",
        preprocessor_path: Optional[Path] = None,
        feature_names: Optional[List[str]] = None
    ):
        """
        Args:
            model_path: Path to serialized model file
            model_name: Name identifier for the model
            preprocessor_path: Optional path to feature preprocessor
            feature_names: Expected feature names for validation
        """
        self.model_path = model_path
        self.model_name = model_name
        self.preprocessor_path = preprocessor_path
        self.feature_names = feature_names or []

        self.model = None
        self.preprocessor = None
        self.model_version = None
        self.status = ModelStatus.UNLOADED
        self.metrics = ModelMetrics()

        # FastAPI app
        self.app = FastAPI(
            title=f"{model_name} Prediction API",
            description="Production ML model serving API",
            version="1.0.0"
        )

        # CORS middleware
        self.app.add_middleware(
            CORSMiddleware,
            allow_origins=["*"],
            allow_credentials=True,
            allow_methods=["*"],
            allow_headers=["*"],
        )

        # Register routes
        self._register_routes()

        # Exception handlers
        self._register_exception_handlers()

    def load_model(self) -> None:
        """Load model and preprocessor from disk."""
        try:
            logger.info(f"Loading model from {self.model_path}")
            self.status = ModelStatus.LOADING

            # Load model
            self.model = joblib.load(self.model_path)

            # Load preprocessor if available
            if self.preprocessor_path and self.preprocessor_path.exists():
                logger.info(f"Loading preprocessor from {self.preprocessor_path}")
                self.preprocessor = joblib.load(self.preprocessor_path)

            # Extract model version from path or metadata
            self.model_version = self._extract_version()

            self.status = ModelStatus.READY
            logger.info(f"Model {self.model_version} loaded successfully")

        except Exception as e:
            self.status = ModelStatus.ERROR
            logger.error(f"Failed to load model: {e}")
            raise

    def _extract_version(self) -> str:
        """Extract model version from path or model metadata."""
        # Try to get version from model metadata
        if hasattr(self.model, 'version'):
            return self.model.version

        # Extract from path (e.g., model_v1.2.3.pkl)
        import re
        version_match = re.search(r'v?(\d+\.\d+\.\d+)', str(self.model_path))
        if version_match:
            return version_match.group(1)

        # Default to timestamp
        return datetime.now().strftime("%Y%m%d_%H%M%S")

    def _validate_features(self, features: Dict[str, Any]) -> None:
        """Validate input features."""
        if self.feature_names:
            missing = set(self.feature_names) - set(features.keys())
            if missing:
                raise ValueError(f"Missing required features: {missing}")

            extra = set(features.keys()) - set(self.feature_names)
            if extra:
                logger.warning(f"Extra features provided (will be ignored): {extra}")

    def _preprocess_features(self, features: Dict[str, Any]) -> np.ndarray:
        """
        Preprocess features for model input.

        Args:
            features: Raw feature dictionary

        Returns:
            Preprocessed feature array
        """
        # Convert to array in correct order
        if self.feature_names:
            feature_array = np.array([
                [features.get(name, 0.0) for name in self.feature_names]
            ])
        else:
            feature_array = np.array([list(features.values())])

        # Apply preprocessor if available
        if self.preprocessor is not None:
            feature_array = self.preprocessor.transform(feature_array)

        return feature_array

    def predict(
        self,
        features: Dict[str, Any],
        return_probabilities: bool = False,
        explain: bool = False
    ) -> Dict[str, Any]:
        """
        Generate prediction.

        Args:
            features: Input features
            return_probabilities: Return class probabilities
            explain: Include SHAP explanation

        Returns:
            Prediction result dictionary
        """
        if self.status != ModelStatus.READY:
            raise RuntimeError(f"Model not ready (status: {self.status.value})")

        start_time = datetime.now()

        try:
            # Validate features
            self._validate_features(features)

            # Preprocess
            X = self._preprocess_features(features)

            # Generate prediction
            if return_probabilities and hasattr(self.model, 'predict_proba'):
                prediction = self.model.predict_proba(X)[0]
                result = {
                    "prediction": prediction.tolist(),
                    "probabilities": dict(zip(
                        self.model.classes_,
                        prediction.tolist()
                    ))
                }
            else:
                prediction = self.model.predict(X)[0]
                result = {"prediction": float(prediction)}

            # Add explanation if requested
            if explain:
                result["explanation"] = self._generate_explanation(X)

            # Record metrics
            latency_ms = (datetime.now() - start_time).total_seconds() * 1000
            self.metrics.predictions_served += 1
            self.metrics.total_latency_ms += latency_ms

            result.update({
                "model_version": self.model_version,
                "latency_ms": latency_ms,
                "timestamp": datetime.now()
            })

            return result

        except Exception as e:
            self.metrics.errors += 1
            logger.error(f"Prediction error: {e}")
            raise

    def _generate_explanation(self, X: np.ndarray) -> Dict[str, float]:
        """
        Generate SHAP explanation for prediction.

        Args:
            X: Preprocessed features

        Returns:
            Feature importance dictionary
        """
        try:
            import shap

            # Create explainer (cache in production)
            explainer = shap.TreeExplainer(self.model)
            shap_values = explainer.shap_values(X)

            # Map to feature names
            if self.feature_names:
                explanation = dict(zip(
                    self.feature_names,
                    shap_values[0].tolist()
                ))
            else:
                explanation = {
                    f"feature_{i}": float(val)
                    for i, val in enumerate(shap_values[0])
                }

            return explanation

        except ImportError:
            logger.warning("SHAP not installed, skipping explanation")
            return {}
        except Exception as e:
            logger.error(f"Explanation generation failed: {e}")
            return {}

    def _register_routes(self) -> None:
        """Register FastAPI routes."""

        @self.app.post("/predict", response_model=PredictionResponse)
        async def predict_endpoint(request: PredictionRequest) -> PredictionResponse:
            """Generate prediction for input features."""
            try:
                result = self.predict(
                    features=request.features,
                    return_probabilities=request.return_probabilities,
                    explain=request.explain
                )

                return PredictionResponse(
                    prediction=result["prediction"],
                    model_version=result["model_version"],
                    prediction_id=f"{self.model_name}_{datetime.now().timestamp()}",
                    timestamp=result["timestamp"],
                    latency_ms=result["latency_ms"],
                    probabilities=result.get("probabilities"),
                    explanation=result.get("explanation")
                )

            except ValueError as e:
                raise HTTPException(
                    status_code=status.HTTP_400_BAD_REQUEST,
                    detail=str(e)
                )
            except Exception as e:
                logger.error(f"Prediction endpoint error: {e}")
                raise HTTPException(
                    status_code=status.HTTP_500_INTERNAL_SERVER_ERROR,
                    detail="Internal server error"
                )

        @self.app.get("/health", response_model=HealthResponse)
        async def health_check() -> HealthResponse:
            """Health check endpoint."""
            return HealthResponse(
                status="healthy" if self.status == ModelStatus.READY else "unhealthy",
                model_status=self.status.value,
                model_version=self.model_version or "unknown",
                uptime_seconds=self.metrics.uptime_seconds,
                predictions_served=self.metrics.predictions_served,
                avg_latency_ms=self.metrics.avg_latency_ms
            )

        @self.app.get("/ready")
        async def readiness_check() -> Dict[str, str]:
            """Kubernetes readiness probe endpoint."""
            if self.status == ModelStatus.READY:
                return {"status": "ready"}
            else:
                raise HTTPException(
                    status_code=status.HTTP_503_SERVICE_UNAVAILABLE,
                    detail=f"Model not ready: {self.status.value}"
                )

        @self.app.get("/metrics")
        async def metrics_endpoint() -> Dict[str, Any]:
            """Prometheus-compatible metrics endpoint."""
            return {
                "predictions_total": self.metrics.predictions_served,
                "errors_total": self.metrics.errors,
                "latency_avg_ms": self.metrics.avg_latency_ms,
                "uptime_seconds": self.metrics.uptime_seconds,
                "model_version": self.model_version
            }

        @self.app.post("/reload")
        async def reload_model() -> Dict[str, str]:
            """Reload model from disk."""
            try:
                self.load_model()
                return {"status": "reloaded", "version": self.model_version}
            except Exception as e:
                raise HTTPException(
                    status_code=status.HTTP_500_INTERNAL_SERVER_ERROR,
                    detail=f"Reload failed: {e}"
                )

    def _register_exception_handlers(self) -> None:
        """Register custom exception handlers."""

        @self.app.exception_handler(ValueError)
        async def value_error_handler(request: Request, exc: ValueError):
            return JSONResponse(
                status_code=status.HTTP_400_BAD_REQUEST,
                content={
                    "error": "Validation error",
                    "detail": str(exc),
                    "timestamp": datetime.now().isoformat()
                }
            )

        @self.app.exception_handler(Exception)
        async def general_exception_handler(request: Request, exc: Exception):
            logger.error(f"Unhandled exception: {exc}", exc_info=True)
            return JSONResponse(
                status_code=status.HTTP_500_INTERNAL_SERVER_ERROR,
                content={
                    "error": "Internal server error",
                    "timestamp": datetime.now().isoformat()
                }
            )

    def run(self, host: str = "0.0.0.0", port: int = 8000) -> None:
        """
        Start the service.

        Args:
            host: Host to bind to
            port: Port to bind to
        """
        # Load model before starting
        self.load_model()

        # Start server
        logger.info(f"Starting {self.model_name} service on {host}:{port}")
        uvicorn.run(self.app, host=host, port=port, log_level="info")
\end{lstlisting}

\section{Containerization with Docker}

Docker containers provide consistent, reproducible deployment environments with proper resource isolation and security.

\subsection{Multi-Stage Docker Build}

\begin{lstlisting}[style=shell, caption={Production Dockerfile with Multi-Stage Build}]
# Stage 1: Build stage with full dependencies
FROM python:3.10-slim as builder

# Install build dependencies
RUN apt-get update && apt-get install -y \
    gcc \
    g++ \
    make \
    libgomp1 \
    && rm -rf /var/lib/apt/lists/*

# Create virtual environment
RUN python -m venv /opt/venv
ENV PATH="/opt/venv/bin:$PATH"

# Copy requirements and install Python dependencies
COPY requirements.txt .
RUN pip install --no-cache-dir -r requirements.txt

# Stage 2: Runtime stage with minimal dependencies
FROM python:3.10-slim

# Create non-root user for security
RUN useradd --create-home --shell /bin/bash mlservice

# Install runtime dependencies only
RUN apt-get update && apt-get install -y \
    libgomp1 \
    && rm -rf /var/lib/apt/lists/*

# Copy virtual environment from builder
COPY --from=builder /opt/venv /opt/venv
ENV PATH="/opt/venv/bin:$PATH"

# Set working directory
WORKDIR /app

# Copy application code
COPY --chown=mlservice:mlservice . /app

# Switch to non-root user
USER mlservice

# Resource limits and configurations
ENV PYTHONUNBUFFERED=1
ENV PYTHONDONTWRITEBYTECODE=1
ENV OMP_NUM_THREADS=4
ENV MKL_NUM_THREADS=4

# Health check
HEALTHCHECK --interval=30s --timeout=10s --start-period=5s --retries=3 \
    CMD curl -f http://localhost:8000/health || exit 1

# Expose port
EXPOSE 8000

# Run service
CMD ["python", "-m", "uvicorn", "main:app", \
     "--host", "0.0.0.0", "--port", "8000", \
     "--workers", "4", "--timeout-keep-alive", "75"]
\end{lstlisting}

\subsection{Docker Compose for Local Testing}

\begin{lstlisting}[style=yaml, caption={Docker Compose Configuration}]
version: '3.8'

services:
  model-service:
    build:
      context: .
      dockerfile: Dockerfile
    ports:
      - "8000:8000"
    environment:
      - MODEL_PATH=/models/model_v1.0.0.pkl
      - LOG_LEVEL=info
      - MAX_WORKERS=4
    volumes:
      - ./models:/models:ro
      - ./logs:/app/logs
    deploy:
      resources:
        limits:
          cpus: '2.0'
          memory: 4G
        reservations:
          cpus: '1.0'
          memory: 2G
    restart: unless-stopped
    healthcheck:
      test: ["CMD", "curl", "-f", "http://localhost:8000/health"]
      interval: 30s
      timeout: 10s
      retries: 3
      start_period: 40s

  prometheus:
    image: prom/prometheus:latest
    ports:
      - "9090:9090"
    volumes:
      - ./prometheus.yml:/etc/prometheus/prometheus.yml
      - prometheus-data:/prometheus
    command:
      - '--config.file=/etc/prometheus/prometheus.yml'
      - '--storage.tsdb.path=/prometheus'
    restart: unless-stopped

  grafana:
    image: grafana/grafana:latest
    ports:
      - "3000:3000"
    environment:
      - GF_SECURITY_ADMIN_PASSWORD=admin
    volumes:
      - grafana-data:/var/lib/grafana
      - ./grafana/dashboards:/etc/grafana/provisioning/dashboards
    depends_on:
      - prometheus
    restart: unless-stopped

volumes:
  prometheus-data:
  grafana-data:
\end{lstlisting}

\section{Deployment Strategies}

Different deployment strategies balance risk, speed, and resource requirements.

\subsection{Blue-Green Deployment}

\begin{lstlisting}[language=Python, caption={Blue-Green Deployment Manager}]
from typing import Literal
import requests
import time

DeploymentColor = Literal["blue", "green"]

@dataclass
class DeploymentEnvironment:
    """Deployment environment configuration."""
    name: str
    color: DeploymentColor
    endpoint: str
    version: str
    is_active: bool
    health_status: str

class BlueGreenDeployment:
    """
    Blue-green deployment strategy.

    Maintains two identical environments (blue and green).
    Traffic routes to one while the other is updated.
    Instant rollback by switching traffic back.
    """

    def __init__(
        self,
        blue_endpoint: str,
        green_endpoint: str,
        router_endpoint: str
    ):
        """
        Args:
            blue_endpoint: Blue environment URL
            green_endpoint: Green environment URL
            router_endpoint: Load balancer/router API endpoint
        """
        self.blue = DeploymentEnvironment(
            name="blue",
            color="blue",
            endpoint=blue_endpoint,
            version="unknown",
            is_active=True,
            health_status="unknown"
        )
        self.green = DeploymentEnvironment(
            name="green",
            color="green",
            endpoint=green_endpoint,
            version="unknown",
            is_active=False,
            health_status="unknown"
        )
        self.router_endpoint = router_endpoint

    def get_active_environment(self) -> DeploymentEnvironment:
        """Get currently active environment."""
        return self.blue if self.blue.is_active else self.green

    def get_inactive_environment(self) -> DeploymentEnvironment:
        """Get currently inactive environment."""
        return self.green if self.blue.is_active else self.blue

    def check_health(self, environment: DeploymentEnvironment) -> bool:
        """
        Check environment health.

        Args:
            environment: Environment to check

        Returns:
            True if healthy, False otherwise
        """
        try:
            response = requests.get(
                f"{environment.endpoint}/health",
                timeout=10
            )

            if response.status_code == 200:
                health_data = response.json()
                environment.health_status = health_data.get("status", "unknown")
                environment.version = health_data.get("model_version", "unknown")
                return environment.health_status == "healthy"
            else:
                environment.health_status = "unhealthy"
                return False

        except Exception as e:
            logger.error(f"Health check failed for {environment.name}: {e}")
            environment.health_status = "error"
            return False

    def deploy_new_version(
        self,
        new_version_path: str,
        validation_requests: Optional[List[Dict]] = None
    ) -> bool:
        """
        Deploy new model version using blue-green strategy.

        Steps:
        1. Deploy to inactive environment
        2. Run health checks
        3. Validate with test requests
        4. Switch traffic
        5. Monitor for issues

        Args:
            new_version_path: Path to new model version
            validation_requests: Test requests for validation

        Returns:
            True if deployment successful
        """
        inactive = self.get_inactive_environment()
        active = self.get_active_environment()

        logger.info(f"Deploying new version to {inactive.name} environment")

        # Step 1: Deploy to inactive environment
        logger.info("Step 1: Deploying to inactive environment")
        if not self._deploy_to_environment(inactive, new_version_path):
            logger.error("Deployment failed")
            return False

        # Step 2: Health check
        logger.info("Step 2: Running health checks")
        time.sleep(5)  # Allow startup time
        if not self.check_health(inactive):
            logger.error(f"Health check failed for {inactive.name}")
            return False

        # Step 3: Validation
        logger.info("Step 3: Validating with test requests")
        if validation_requests:
            if not self._validate_environment(inactive, validation_requests):
                logger.error("Validation failed")
                return False

        # Step 4: Switch traffic
        logger.info("Step 4: Switching traffic to new version")
        if not self._switch_traffic(inactive):
            logger.error("Traffic switch failed")
            return False

        # Update state
        inactive.is_active = True
        active.is_active = False

        # Step 5: Monitor
        logger.info("Step 5: Monitoring new deployment")
        if not self._monitor_deployment(inactive, duration_seconds=300):
            logger.warning("Issues detected, consider rollback")
            return False

        logger.info(f"Deployment successful: {inactive.version} active on {inactive.name}")
        return True

    def _deploy_to_environment(
        self,
        environment: DeploymentEnvironment,
        model_path: str
    ) -> bool:
        """Deploy model to environment."""
        try:
            # In practice, this would trigger CI/CD pipeline
            # or Kubernetes deployment update
            response = requests.post(
                f"{environment.endpoint}/reload",
                json={"model_path": model_path},
                timeout=60
            )
            return response.status_code == 200
        except Exception as e:
            logger.error(f"Deployment to {environment.name} failed: {e}")
            return False

    def _validate_environment(
        self,
        environment: DeploymentEnvironment,
        validation_requests: List[Dict]
    ) -> bool:
        """Validate environment with test requests."""
        logger.info(f"Validating {environment.name} with {len(validation_requests)} requests")

        for i, request_data in enumerate(validation_requests):
            try:
                response = requests.post(
                    f"{environment.endpoint}/predict",
                    json=request_data,
                    timeout=30
                )

                if response.status_code != 200:
                    logger.error(f"Validation request {i} failed: {response.status_code}")
                    return False

                # Optional: Check prediction quality
                result = response.json()
                logger.debug(f"Validation {i}: {result}")

            except Exception as e:
                logger.error(f"Validation request {i} error: {e}")
                return False

        logger.info("All validation requests passed")
        return True

    def _switch_traffic(self, new_active: DeploymentEnvironment) -> bool:
        """Switch traffic to new environment."""
        try:
            # Update load balancer configuration
            response = requests.post(
                f"{self.router_endpoint}/switch",
                json={
                    "target": new_active.name,
                    "endpoint": new_active.endpoint
                },
                timeout=30
            )
            return response.status_code == 200
        except Exception as e:
            logger.error(f"Traffic switch failed: {e}")
            return False

    def _monitor_deployment(
        self,
        environment: DeploymentEnvironment,
        duration_seconds: int = 300
    ) -> bool:
        """
        Monitor deployment for issues.

        Args:
            environment: Environment to monitor
            duration_seconds: How long to monitor

        Returns:
            True if no issues detected
        """
        logger.info(f"Monitoring {environment.name} for {duration_seconds}s")

        start_time = time.time()
        check_interval = 30

        while time.time() - start_time < duration_seconds:
            # Check health
            if not self.check_health(environment):
                logger.error(f"Health check failed during monitoring")
                return False

            # Check metrics (error rate, latency, etc.)
            metrics = self._get_metrics(environment)
            if self._detect_anomalies(metrics):
                logger.warning(f"Anomalies detected in metrics: {metrics}")
                return False

            time.sleep(check_interval)

        logger.info("Monitoring complete: no issues detected")
        return True

    def _get_metrics(self, environment: DeploymentEnvironment) -> Dict[str, float]:
        """Get metrics from environment."""
        try:
            response = requests.get(
                f"{environment.endpoint}/metrics",
                timeout=10
            )
            if response.status_code == 200:
                return response.json()
            return {}
        except Exception as e:
            logger.error(f"Failed to get metrics: {e}")
            return {}

    def _detect_anomalies(self, metrics: Dict[str, float]) -> bool:
        """Detect anomalies in metrics."""
        # Simple threshold-based detection
        if metrics.get("errors_total", 0) > 10:
            return True
        if metrics.get("latency_avg_ms", 0) > 1000:
            return True
        return False

    def rollback(self) -> bool:
        """
        Rollback to previous version.

        Simply switches traffic back to previous environment.
        """
        current_active = self.get_active_environment()
        previous = self.get_inactive_environment()

        logger.info(f"Rolling back from {current_active.name} to {previous.name}")

        # Check previous environment health
        if not self.check_health(previous):
            logger.error(f"Cannot rollback: {previous.name} is unhealthy")
            return False

        # Switch traffic
        if not self._switch_traffic(previous):
            logger.error("Rollback traffic switch failed")
            return False

        # Update state
        current_active.is_active = False
        previous.is_active = True

        logger.info(f"Rollback successful: {previous.version} active on {previous.name}")
        return True
\end{lstlisting}

\subsection{Canary Deployment}

\begin{lstlisting}[language=Python, caption={Canary Deployment Strategy}]
from typing import List
import random

class CanaryDeployment:
    """
    Canary deployment strategy.

    Gradually routes traffic to new version while monitoring metrics.
    Rolls back automatically if issues detected.
    """

    def __init__(
        self,
        stable_endpoint: str,
        canary_endpoint: str,
        router_endpoint: str
    ):
        """
        Args:
            stable_endpoint: Stable version endpoint
            canary_endpoint: Canary version endpoint
            router_endpoint: Load balancer API endpoint
        """
        self.stable_endpoint = stable_endpoint
        self.canary_endpoint = canary_endpoint
        self.router_endpoint = router_endpoint
        self.canary_weight = 0.0

    def deploy_canary(
        self,
        new_version_path: str,
        traffic_stages: List[float] = [0.05, 0.10, 0.25, 0.50, 1.0],
        stage_duration_seconds: int = 300,
        validation_requests: Optional[List[Dict]] = None
    ) -> bool:
        """
        Deploy canary with gradual traffic increase.

        Args:
            new_version_path: Path to new model
            traffic_stages: Traffic percentages for canary (e.g., [5%, 10%, 25%])
            stage_duration_seconds: How long to run each stage
            validation_requests: Test requests for validation

        Returns:
            True if deployment successful
        """
        logger.info(f"Starting canary deployment with stages: {traffic_stages}")

        # Deploy canary version
        logger.info("Deploying canary version")
        if not self._deploy_canary_version(new_version_path):
            logger.error("Canary deployment failed")
            return False

        # Validate canary
        if validation_requests:
            logger.info("Validating canary version")
            if not self._validate_canary(validation_requests):
                logger.error("Canary validation failed")
                self._cleanup_canary()
                return False

        # Gradual traffic increase
        for stage_pct in traffic_stages:
            logger.info(f"Increasing canary traffic to {stage_pct:.0%}")

            # Update traffic split
            if not self._update_traffic_split(stage_pct):
                logger.error("Failed to update traffic split")
                self.rollback_canary()
                return False

            # Monitor stage
            logger.info(f"Monitoring stage for {stage_duration_seconds}s")
            if not self._monitor_canary_stage(stage_duration_seconds):
                logger.error("Issues detected, rolling back")
                self.rollback_canary()
                return False

            logger.info(f"Stage {stage_pct:.0%} successful")

        # Promote canary to stable
        logger.info("Promoting canary to stable")
        self._promote_canary()

        logger.info("Canary deployment successful")
        return True

    def _deploy_canary_version(self, model_path: str) -> bool:
        """Deploy new version to canary environment."""
        try:
            response = requests.post(
                f"{self.canary_endpoint}/reload",
                json={"model_path": model_path},
                timeout=60
            )

            if response.status_code == 200:
                # Wait for startup
                time.sleep(5)

                # Health check
                health_response = requests.get(
                    f"{self.canary_endpoint}/health",
                    timeout=10
                )
                return health_response.status_code == 200

            return False

        except Exception as e:
            logger.error(f"Canary deployment failed: {e}")
            return False

    def _validate_canary(self, validation_requests: List[Dict]) -> bool:
        """Validate canary with test requests."""
        logger.info(f"Validating canary with {len(validation_requests)} requests")

        for i, request_data in enumerate(validation_requests):
            try:
                # Send to canary
                canary_response = requests.post(
                    f"{self.canary_endpoint}/predict",
                    json=request_data,
                    timeout=30
                )

                # Send to stable for comparison
                stable_response = requests.post(
                    f"{self.stable_endpoint}/predict",
                    json=request_data,
                    timeout=30
                )

                if canary_response.status_code != 200:
                    logger.error(f"Canary request {i} failed")
                    return False

                # Optional: Compare predictions
                canary_pred = canary_response.json()
                stable_pred = stable_response.json()

                logger.debug(f"Validation {i}:")
                logger.debug(f"  Stable: {stable_pred['prediction']}")
                logger.debug(f"  Canary: {canary_pred['prediction']}")

            except Exception as e:
                logger.error(f"Validation error: {e}")
                return False

        return True

    def _update_traffic_split(self, canary_weight: float) -> bool:
        """
        Update traffic split between stable and canary.

        Args:
            canary_weight: Fraction of traffic to canary (0.0 to 1.0)
        """
        try:
            response = requests.post(
                f"{self.router_endpoint}/weight",
                json={
                    "stable_weight": 1.0 - canary_weight,
                    "canary_weight": canary_weight,
                    "stable_endpoint": self.stable_endpoint,
                    "canary_endpoint": self.canary_endpoint
                },
                timeout=30
            )

            if response.status_code == 200:
                self.canary_weight = canary_weight
                return True

            return False

        except Exception as e:
            logger.error(f"Failed to update traffic split: {e}")
            return False

    def _monitor_canary_stage(self, duration_seconds: int) -> bool:
        """
        Monitor canary stage for issues.

        Compares canary metrics to stable metrics.
        """
        logger.info(f"Monitoring canary stage for {duration_seconds}s")

        start_time = time.time()
        check_interval = 30

        while time.time() - start_time < duration_seconds:
            # Get metrics from both versions
            stable_metrics = self._get_metrics(self.stable_endpoint)
            canary_metrics = self._get_metrics(self.canary_endpoint)

            # Compare metrics
            if self._detect_canary_issues(stable_metrics, canary_metrics):
                logger.error("Canary issues detected")
                return False

            time.sleep(check_interval)

        logger.info("Stage monitoring complete: no issues")
        return True

    def _get_metrics(self, endpoint: str) -> Dict[str, float]:
        """Get metrics from endpoint."""
        try:
            response = requests.get(f"{endpoint}/metrics", timeout=10)
            if response.status_code == 200:
                return response.json()
            return {}
        except Exception as e:
            logger.error(f"Failed to get metrics from {endpoint}: {e}")
            return {}

    def _detect_canary_issues(
        self,
        stable_metrics: Dict[str, float],
        canary_metrics: Dict[str, float]
    ) -> bool:
        """
        Detect issues by comparing canary to stable metrics.

        Returns True if issues detected.
        """
        # Error rate comparison
        stable_errors = stable_metrics.get("errors_total", 0)
        canary_errors = canary_metrics.get("errors_total", 0)

        stable_predictions = stable_metrics.get("predictions_total", 1)
        canary_predictions = canary_metrics.get("predictions_total", 1)

        stable_error_rate = stable_errors / stable_predictions
        canary_error_rate = canary_errors / canary_predictions

        # Canary error rate significantly higher?
        if canary_error_rate > stable_error_rate * 2 and canary_error_rate > 0.01:
            logger.error(f"Canary error rate too high: "
                        f"{canary_error_rate:.2%} vs {stable_error_rate:.2%}")
            return True

        # Latency comparison
        stable_latency = stable_metrics.get("latency_avg_ms", 0)
        canary_latency = canary_metrics.get("latency_avg_ms", 0)

        # Canary latency significantly higher?
        if canary_latency > stable_latency * 1.5 and canary_latency > 500:
            logger.error(f"Canary latency too high: "
                        f"{canary_latency:.0f}ms vs {stable_latency:.0f}ms")
            return True

        return False

    def rollback_canary(self) -> bool:
        """Rollback canary deployment."""
        logger.info("Rolling back canary deployment")

        # Route all traffic to stable
        if not self._update_traffic_split(0.0):
            logger.error("Failed to rollback traffic")
            return False

        # Cleanup canary
        self._cleanup_canary()

        logger.info("Canary rollback complete")
        return True

    def _promote_canary(self) -> None:
        """Promote canary to stable."""
        # In practice, this would update stable environment
        # to run canary version and cleanup old stable
        logger.info("Promoting canary to stable")

        # Swap endpoints
        self.stable_endpoint, self.canary_endpoint = \
            self.canary_endpoint, self.stable_endpoint

        # Reset traffic split
        self.canary_weight = 0.0

    def _cleanup_canary(self) -> None:
        """Cleanup canary deployment."""
        logger.info("Cleaning up canary deployment")
        # In practice, this would terminate canary pods/containers
\end{lstlisting}

\section{Kubernetes Deployment Configuration}

Kubernetes provides robust orchestration for containerized ML services with auto-scaling, health checks, and rolling updates.

\subsection{Kubernetes Deployment and Service}

\begin{lstlisting}[style=yaml, caption={Kubernetes Deployment Configuration}]
apiVersion: apps/v1
kind: Deployment
metadata:
  name: model-service
  labels:
    app: model-service
    version: v1.0.0
spec:
  replicas: 3
  strategy:
    type: RollingUpdate
    rollingUpdate:
      maxSurge: 1
      maxUnavailable: 0
  selector:
    matchLabels:
      app: model-service
  template:
    metadata:
      labels:
        app: model-service
        version: v1.0.0
    spec:
      containers:
      - name: model-service
        image: your-registry/model-service:v1.0.0
        imagePullPolicy: Always
        ports:
        - containerPort: 8000
          name: http

        # Resource limits and requests
        resources:
          requests:
            cpu: "500m"
            memory: "1Gi"
          limits:
            cpu: "2000m"
            memory: "4Gi"

        # Environment variables
        env:
        - name: MODEL_PATH
          value: "/models/model_v1.0.0.pkl"
        - name: LOG_LEVEL
          value: "info"
        - name: MAX_WORKERS
          value: "4"
        - name: POD_NAME
          valueFrom:
            fieldRef:
              fieldPath: metadata.name
        - name: POD_NAMESPACE
          valueFrom:
            fieldRef:
              fieldPath: metadata.namespace

        # Volume mounts
        volumeMounts:
        - name: model-storage
          mountPath: /models
          readOnly: true
        - name: cache
          mountPath: /tmp

        # Liveness probe - is container alive?
        livenessProbe:
          httpGet:
            path: /health
            port: 8000
          initialDelaySeconds: 30
          periodSeconds: 30
          timeoutSeconds: 10
          failureThreshold: 3

        # Readiness probe - is container ready for traffic?
        readinessProbe:
          httpGet:
            path: /ready
            port: 8000
          initialDelaySeconds: 10
          periodSeconds: 10
          timeoutSeconds: 5
          failureThreshold: 3

        # Startup probe - has container started successfully?
        startupProbe:
          httpGet:
            path: /health
            port: 8000
          initialDelaySeconds: 0
          periodSeconds: 10
          timeoutSeconds: 5
          failureThreshold: 30

        # Security context
        securityContext:
          runAsNonRoot: true
          runAsUser: 1000
          allowPrivilegeEscalation: false
          readOnlyRootFilesystem: true
          capabilities:
            drop:
            - ALL

      # Volumes
      volumes:
      - name: model-storage
        persistentVolumeClaim:
          claimName: model-pvc
      - name: cache
        emptyDir: {}

      # Node affinity and tolerations
      affinity:
        podAntiAffinity:
          preferredDuringSchedulingIgnoredDuringExecution:
          - weight: 100
            podAffinityTerm:
              labelSelector:
                matchExpressions:
                - key: app
                  operator: In
                  values:
                  - model-service
              topologyKey: kubernetes.io/hostname
---
apiVersion: v1
kind: Service
metadata:
  name: model-service
  labels:
    app: model-service
spec:
  type: ClusterIP
  ports:
  - port: 80
    targetPort: 8000
    protocol: TCP
    name: http
  selector:
    app: model-service
---
apiVersion: autoscaling/v2
kind: HorizontalPodAutoscaler
metadata:
  name: model-service-hpa
spec:
  scaleTargetRef:
    apiVersion: apps/v1
    kind: Deployment
    name: model-service
  minReplicas: 3
  maxReplicas: 10
  metrics:
  - type: Resource
    resource:
      name: cpu
      target:
        type: Utilization
        averageUtilization: 70
  - type: Resource
    resource:
      name: memory
      target:
        type: Utilization
        averageUtilization: 80
  behavior:
    scaleDown:
      stabilizationWindowSeconds: 300
      policies:
      - type: Percent
        value: 50
        periodSeconds: 60
    scaleUp:
      stabilizationWindowSeconds: 0
      policies:
      - type: Percent
        value: 100
        periodSeconds: 30
      - type: Pods
        value: 2
        periodSeconds: 30
      selectPolicy: Max
\end{lstlisting}

\subsection{Model Registry Integration}

\begin{lstlisting}[language=Python, caption={Model Registry for Version Management}]
from typing import Optional, List
from pathlib import Path
import hashlib
import shutil
from datetime import datetime

@dataclass
class ModelMetadata:
    """Metadata for registered model."""
    model_id: str
    version: str
    name: str
    framework: str  # 'sklearn', 'pytorch', 'tensorflow', etc.
    metrics: Dict[str, float]
    training_date: datetime
    author: str
    description: str
    tags: List[str]
    file_path: Path
    file_hash: str
    status: str  # 'registered', 'staging', 'production', 'archived'

class ModelRegistry:
    """
    Centralized model registry for version management.

    Features:
    - Version tracking
    - Metadata storage
    - Model promotion (dev -> staging -> production)
    - Rollback capabilities
    - Model comparison
    """

    def __init__(self, registry_path: Path):
        """
        Args:
            registry_path: Base path for model registry storage
        """
        self.registry_path = Path(registry_path)
        self.registry_path.mkdir(parents=True, exist_ok=True)

        self.metadata_file = self.registry_path / "registry.json"
        self.models: Dict[str, ModelMetadata] = {}

        # Load existing registry
        self._load_registry()

    def _load_registry(self) -> None:
        """Load registry from disk."""
        if self.metadata_file.exists():
            with open(self.metadata_file, 'r') as f:
                data = json.load(f)
                self.models = {
                    k: ModelMetadata(**v) for k, v in data.items()
                }
            logger.info(f"Loaded {len(self.models)} models from registry")

    def _save_registry(self) -> None:
        """Save registry to disk."""
        with open(self.metadata_file, 'w') as f:
            data = {
                k: {
                    **v.__dict__,
                    'training_date': v.training_date.isoformat(),
                    'file_path': str(v.file_path)
                }
                for k, v in self.models.items()
            }
            json.dump(data, f, indent=2)

    def _compute_file_hash(self, file_path: Path) -> str:
        """Compute SHA256 hash of model file."""
        sha256 = hashlib.sha256()
        with open(file_path, 'rb') as f:
            for chunk in iter(lambda: f.read(4096), b''):
                sha256.update(chunk)
        return sha256.hexdigest()

    def register_model(
        self,
        model_path: Path,
        name: str,
        version: str,
        framework: str,
        metrics: Dict[str, float],
        author: str,
        description: str = "",
        tags: Optional[List[str]] = None
    ) -> str:
        """
        Register new model version.

        Args:
            model_path: Path to model file
            name: Model name
            version: Version string (e.g., '1.0.0')
            framework: ML framework used
            metrics: Model performance metrics
            author: Model author
            description: Model description
            tags: Optional tags for organization

        Returns:
            Model ID
        """
        # Generate model ID
        model_id = f"{name}_{version}_{datetime.now().strftime('%Y%m%d_%H%M%S')}"

        # Copy model to registry
        registry_model_path = self.registry_path / model_id / "model"
        registry_model_path.parent.mkdir(parents=True, exist_ok=True)
        shutil.copy2(model_path, registry_model_path)

        # Compute hash
        file_hash = self._compute_file_hash(registry_model_path)

        # Create metadata
        metadata = ModelMetadata(
            model_id=model_id,
            version=version,
            name=name,
            framework=framework,
            metrics=metrics,
            training_date=datetime.now(),
            author=author,
            description=description,
            tags=tags or [],
            file_path=registry_model_path,
            file_hash=file_hash,
            status='registered'
        )

        self.models[model_id] = metadata
        self._save_registry()

        logger.info(f"Registered model {model_id}")
        return model_id

    def promote_model(self, model_id: str, target_status: str) -> bool:
        """
        Promote model to new status.

        Args:
            model_id: Model to promote
            target_status: Target status ('staging' or 'production')

        Returns:
            True if successful
        """
        if model_id not in self.models:
            logger.error(f"Model {model_id} not found")
            return False

        model = self.models[model_id]

        # Validation based on target status
        if target_status == 'production':
            # Check if model was in staging
            if model.status != 'staging':
                logger.warning(f"Promoting {model_id} to production "
                             f"without staging (current: {model.status})")

        # Demote previous production models if promoting to production
        if target_status == 'production':
            for mid, m in self.models.items():
                if m.name == model.name and m.status == 'production':
                    m.status = 'archived'
                    logger.info(f"Archived previous production model: {mid}")

        model.status = target_status
        self._save_registry()

        logger.info(f"Promoted {model_id} to {target_status}")
        return True

    def get_model(
        self,
        name: str,
        version: Optional[str] = None,
        status: str = 'production'
    ) -> Optional[ModelMetadata]:
        """
        Get model by name, version, and status.

        Args:
            name: Model name
            version: Specific version (None for latest)
            status: Model status filter

        Returns:
            ModelMetadata or None
        """
        # Filter by name and status
        candidates = [
            m for m in self.models.values()
            if m.name == name and m.status == status
        ]

        if not candidates:
            return None

        # Filter by version if specified
        if version:
            candidates = [m for m in candidates if m.version == version]
            if not candidates:
                return None
            return candidates[0]

        # Return latest (by training date)
        return max(candidates, key=lambda m: m.training_date)

    def rollback_production(self, name: str) -> Optional[str]:
        """
        Rollback to previous production model.

        Args:
            name: Model name

        Returns:
            Rolled back model ID or None
        """
        # Get current production model
        current = self.get_model(name, status='production')
        if not current:
            logger.error(f"No production model found for {name}")
            return None

        # Archive current
        current.status = 'archived'

        # Find previous production (now archived)
        archived = [
            m for m in self.models.values()
            if m.name == name and m.status == 'archived'
            and m.model_id != current.model_id
        ]

        if not archived:
            logger.error(f"No previous version found for rollback")
            return None

        # Get most recent archived
        previous = max(archived, key=lambda m: m.training_date)

        # Promote to production
        previous.status = 'production'
        self._save_registry()

        logger.info(f"Rolled back {name} from {current.version} to {previous.version}")
        return previous.model_id

    def compare_models(
        self,
        model_id_1: str,
        model_id_2: str
    ) -> Dict[str, Any]:
        """
        Compare two models.

        Returns:
            Comparison dictionary with metrics differences
        """
        if model_id_1 not in self.models or model_id_2 not in self.models:
            raise ValueError("One or both models not found")

        model1 = self.models[model_id_1]
        model2 = self.models[model_id_2]

        # Compare metrics
        metric_comparison = {}
        all_metrics = set(model1.metrics.keys()) | set(model2.metrics.keys())

        for metric in all_metrics:
            val1 = model1.metrics.get(metric)
            val2 = model2.metrics.get(metric)

            if val1 is not None and val2 is not None:
                diff = val2 - val1
                pct_change = (diff / val1 * 100) if val1 != 0 else float('inf')
                metric_comparison[metric] = {
                    "model1": val1,
                    "model2": val2,
                    "difference": diff,
                    "percent_change": pct_change
                }

        return {
            "model1": {
                "id": model1.model_id,
                "version": model1.version,
                "status": model1.status
            },
            "model2": {
                "id": model2.model_id,
                "version": model2.version,
                "status": model2.status
            },
            "metrics": metric_comparison
        }

    def list_models(
        self,
        name: Optional[str] = None,
        status: Optional[str] = None
    ) -> List[ModelMetadata]:
        """
        List models with optional filters.

        Args:
            name: Filter by model name
            status: Filter by status

        Returns:
            List of matching models
        """
        models = list(self.models.values())

        if name:
            models = [m for m in models if m.name == name]

        if status:
            models = [m for m in models if m.status == status]

        # Sort by training date (newest first)
        models.sort(key=lambda m: m.training_date, reverse=True)

        return models
\end{lstlisting}

\section{CI/CD Pipeline for Model Deployment}

Automated deployment pipelines ensure consistent, tested deployments with proper validation.

\subsection{GitHub Actions Deployment Pipeline}

\begin{lstlisting}[style=yaml, caption={CI/CD Pipeline with GitHub Actions}]
name: Model Deployment Pipeline

on:
  push:
    branches:
      - main
    paths:
      - 'models/**'
      - 'src/**'
      - 'requirements.txt'
      - 'Dockerfile'

  workflow_dispatch:
    inputs:
      environment:
        description: 'Deployment environment'
        required: true
        type: choice
        options:
          - staging
          - production

env:
  REGISTRY: ghcr.io
  IMAGE_NAME: ${{ github.repository }}/model-service

jobs:
  test:
    name: Run Tests
    runs-on: ubuntu-latest
    steps:
      - name: Checkout code
        uses: actions/checkout@v3

      - name: Set up Python
        uses: actions/setup-python@v4
        with:
          python-version: '3.10'
          cache: 'pip'

      - name: Install dependencies
        run: |
          pip install -r requirements.txt
          pip install pytest pytest-cov

      - name: Run unit tests
        run: |
          pytest tests/unit --cov=src --cov-report=xml

      - name: Run integration tests
        run: |
          pytest tests/integration

      - name: Upload coverage
        uses: codecov/codecov-action@v3
        with:
          files: ./coverage.xml

  validate-model:
    name: Validate Model
    needs: test
    runs-on: ubuntu-latest
    steps:
      - name: Checkout code
        uses: actions/checkout@v3

      - name: Set up Python
        uses: actions/setup-python@v4
        with:
          python-version: '3.10'

      - name: Install dependencies
        run: pip install -r requirements.txt

      - name: Validate model performance
        run: |
          python scripts/validate_model.py \
            --model-path models/model_latest.pkl \
            --test-data data/test.csv \
            --min-accuracy 0.85 \
            --max-latency-ms 500

      - name: Check model size
        run: |
          MODEL_SIZE=$(stat -f%z models/model_latest.pkl)
          if [ $MODEL_SIZE -gt 1073741824 ]; then
            echo "Model size exceeds 1GB limit"
            exit 1
          fi

  build:
    name: Build and Push Docker Image
    needs: [test, validate-model]
    runs-on: ubuntu-latest
    permissions:
      contents: read
      packages: write
    outputs:
      image-tag: ${{ steps.meta.outputs.tags }}
    steps:
      - name: Checkout code
        uses: actions/checkout@v3

      - name: Set up Docker Buildx
        uses: docker/setup-buildx-action@v2

      - name: Log in to Container Registry
        uses: docker/login-action@v2
        with:
          registry: ${{ env.REGISTRY }}
          username: ${{ github.actor }}
          password: ${{ secrets.GITHUB_TOKEN }}

      - name: Extract metadata
        id: meta
        uses: docker/metadata-action@v4
        with:
          images: ${{ env.REGISTRY }}/${{ env.IMAGE_NAME }}
          tags: |
            type=ref,event=branch
            type=sha,prefix={{branch}}-
            type=semver,pattern={{version}}

      - name: Build and push Docker image
        uses: docker/build-push-action@v4
        with:
          context: .
          push: true
          tags: ${{ steps.meta.outputs.tags }}
          labels: ${{ steps.meta.outputs.labels }}
          cache-from: type=gha
          cache-to: type=gha,mode=max

      - name: Run security scan
        uses: aquasecurity/trivy-action@master
        with:
          image-ref: ${{ steps.meta.outputs.tags }}
          format: 'sarif'
          output: 'trivy-results.sarif'

      - name: Upload scan results
        uses: github/codeql-action/upload-sarif@v2
        with:
          sarif_file: 'trivy-results.sarif'

  deploy-staging:
    name: Deploy to Staging
    needs: build
    if: github.ref == 'refs/heads/main'
    runs-on: ubuntu-latest
    environment:
      name: staging
      url: https://staging.api.example.com
    steps:
      - name: Checkout code
        uses: actions/checkout@v3

      - name: Configure kubectl
        uses: azure/k8s-set-context@v3
        with:
          kubeconfig: ${{ secrets.KUBECONFIG_STAGING }}

      - name: Deploy to Kubernetes
        run: |
          kubectl set image deployment/model-service \
            model-service=${{ needs.build.outputs.image-tag }} \
            -n staging

          kubectl rollout status deployment/model-service -n staging

      - name: Run smoke tests
        run: |
          python scripts/smoke_test.py \
            --endpoint https://staging.api.example.com \
            --requests 100

      - name: Notify deployment
        uses: 8398a7/action-slack@v3
        with:
          status: ${{ job.status }}
          text: 'Staging deployment completed'
          webhook_url: ${{ secrets.SLACK_WEBHOOK }}
        if: always()

  deploy-production:
    name: Deploy to Production
    needs: [build, deploy-staging]
    if: github.event.inputs.environment == 'production'
    runs-on: ubuntu-latest
    environment:
      name: production
      url: https://api.example.com
    steps:
      - name: Checkout code
        uses: actions/checkout@v3

      - name: Configure kubectl
        uses: azure/k8s-set-context@v3
        with:
          kubeconfig: ${{ secrets.KUBECONFIG_PRODUCTION }}

      - name: Create backup
        run: |
          kubectl get deployment model-service -n production -o yaml > backup.yaml

      - name: Deploy with canary strategy
        run: |
          python scripts/canary_deploy.py \
            --image ${{ needs.build.outputs.image-tag }} \
            --namespace production \
            --stages 0.05,0.10,0.25,0.50,1.0 \
            --stage-duration 600

      - name: Run production smoke tests
        run: |
          python scripts/smoke_test.py \
            --endpoint https://api.example.com \
            --requests 1000

      - name: Update model registry
        run: |
          python scripts/update_registry.py \
            --model-id ${{ github.sha }} \
            --status production

      - name: Notify deployment
        uses: 8398a7/action-slack@v3
        with:
          status: ${{ job.status }}
          text: 'Production deployment completed'
          webhook_url: ${{ secrets.SLACK_WEBHOOK }}
        if: always()

  rollback:
    name: Rollback Deployment
    if: failure()
    needs: [deploy-production]
    runs-on: ubuntu-latest
    steps:
      - name: Rollback to previous version
        run: |
          kubectl rollout undo deployment/model-service -n production
          kubectl rollout status deployment/model-service -n production

      - name: Notify rollback
        uses: 8398a7/action-slack@v3
        with:
          status: 'failure'
          text: 'Production deployment failed, rollback initiated'
          webhook_url: ${{ secrets.SLACK_WEBHOOK }}
\end{lstlisting}

\section{Real-World Scenario: The Black Friday Deployment Disaster}

\subsection{RetailML's Production Outage}

RetailML, an e-commerce recommendation platform serving 5 million daily predictions, deployed a new recommendation model on Black Friday eve—the highest traffic day of the year. The deployment used a simple rolling update without proper validation or monitoring.

Within 15 minutes of deployment:
\begin{itemize}
    \item \textbf{Recommendation API latency} increased from 50ms to 8 seconds
    \item \textbf{Error rate} spiked to 23\% (from baseline 0.1\%)
    \item \textbf{Customer purchases} dropped 47\% as pages failed to load
    \item \textbf{Revenue loss}: \$1.2M in the first hour
\end{itemize}

\subsection{Root Cause Analysis}

The incident investigation revealed multiple failures:

\textbf{1. Insufficient Resource Allocation}
\begin{itemize}
    \item New model required 3.5GB memory vs. 1GB allocated
    \item Containers OOMKilled and restarted continuously
    \item No resource request updates in deployment configuration
\end{itemize}

\textbf{2. Missing Performance Validation}
\begin{itemize}
    \item Model inference time: 800ms (vs. 30ms for old model)
    \item No load testing performed before production
    \item Staging environment had 10x less traffic than production
\end{itemize}

\textbf{3. Poor Deployment Strategy}
\begin{itemize}
    \item Rolling update deployed to all pods simultaneously
    \item No canary testing with small traffic percentage
    \item No automatic rollback on error rate increase
\end{itemize}

\textbf{4. Inadequate Monitoring}
\begin{itemize}
    \item No alerting on latency degradation
    \item 15 minutes until manual detection
    \item No automated circuit breaker to old version
\end{itemize}

\subsection{The Recovery Process}

\textbf{Immediate Actions (15-30 minutes)}:
\begin{enumerate}
    \item Manual rollback to previous deployment (10 minutes)
    \item Verified old version health checks passing
    \item Confirmed error rate returned to baseline
    \item Revenue recovery began
\end{enumerate}

\textbf{Root Cause Fixes (Week 1)}:
\begin{enumerate}
    \item Updated Kubernetes deployment with 4GB memory limit
    \item Optimized model inference (quantization + ONNX runtime)
    \item Reduced inference time from 800ms to 45ms
    \item Implemented model performance gates in CI/CD
\end{enumerate}

\textbf{Process Improvements (Week 2-4)}:
\begin{enumerate}
    \item Implemented canary deployment strategy
    \item Added automated rollback on SLO violations
    \item Enhanced monitoring with p95/p99 latency alerts
    \item Created production-scale load testing environment
    \item Established deployment windows (never on high-traffic periods)
\end{enumerate}

\subsection{The Corrective Deployment}

After fixes, the team successfully deployed with canary strategy:

\begin{enumerate}
    \item \textbf{5\% canary}: Monitored for 30 minutes, p99 latency 52ms (OK)
    \item \textbf{25\% canary}: Monitored for 1 hour, error rate 0.08\% (OK)
    \item \textbf{50\% canary}: Monitored for 2 hours, all metrics healthy (OK)
    \item \textbf{100\% rollout}: Completed successfully
\end{enumerate}

Results after successful deployment:
\begin{itemize}
    \item Recommendation quality improved 12\% (measured by CTR)
    \item Latency maintained at p99 < 100ms
    \item Zero errors during deployment
    \item \$450K additional weekly revenue from better recommendations
\end{itemize}

\subsection{Lessons Learned}

\begin{enumerate}
    \item \textbf{Never deploy on peak traffic days}: Deployment windows matter
    \item \textbf{Canary deployments are mandatory}: Catch issues with 5\% traffic, not 100\%
    \item \textbf{Performance testing is non-negotiable}: Staging must match production load
    \item \textbf{Resource requirements must be validated}: Memory/CPU limits prevent OOM kills
    \item \textbf{Automated rollback saves millions}: Manual detection is too slow
    \item \textbf{Monitoring must be proactive}: Alert before customers notice
    \item \textbf{Model optimization is deployment engineering}: Fast models prevent incidents
\end{enumerate}

\section{Exercises}

\subsection{Exercise 1: FastAPI Model Service (Easy)}

Implement a complete FastAPI service for a classification model:
\begin{itemize}
    \item Request/response validation with Pydantic
    \item Health and readiness endpoints
    \item Prediction endpoint with error handling
    \item Metrics endpoint for monitoring
    \item CORS middleware configuration
\end{itemize}

Test with curl or Python requests library.

\subsection{Exercise 2: Docker Containerization (Easy)}

Create a production Dockerfile for your model service:
\begin{itemize}
    \item Multi-stage build (builder + runtime)
    \item Non-root user for security
    \item Minimal base image (python:3.10-slim)
    \item Health check configuration
    \item Proper layer caching for dependencies
\end{itemize}

Build and run the container, verify endpoints work.

\subsection{Exercise 3: Kubernetes Deployment (Medium)}

Write Kubernetes manifests for model deployment:
\begin{itemize}
    \item Deployment with 3 replicas
    \item Resource requests and limits
    \item Liveness and readiness probes
    \item Service with ClusterIP
    \item HorizontalPodAutoscaler based on CPU
\end{itemize}

Deploy to local Kubernetes (minikube or kind) and test scaling.

\subsection{Exercise 4: Model Registry (Medium)}

Implement a model registry system:
\begin{itemize}
    \item Register models with metadata (version, metrics, author)
    \item Promote models through stages (dev → staging → production)
    \item Compare model metrics between versions
    \item Rollback to previous production version
    \item List models with filtering
\end{itemize}

Test with multiple model versions and promotions.

\subsection{Exercise 5: Blue-Green Deployment (Medium)}

Implement blue-green deployment:
\begin{itemize}
    \item Maintain two identical environments
    \item Deploy new version to inactive environment
    \item Run validation tests on new version
    \item Switch traffic to new version
    \item Implement instant rollback capability
\end{itemize}

Simulate a deployment and rollback scenario.

\subsection{Exercise 6: Canary Deployment with Monitoring (Advanced)}

Implement canary deployment with automated decision making:
\begin{itemize}
    \item Gradual traffic increase (5\% → 10\% → 25\% → 50\% → 100\%)
    \item Real-time metrics comparison (error rate, latency)
    \item Automated rollback on threshold violations
    \item Stage duration configuration
    \item Comprehensive logging
\end{itemize}

Simulate both successful deployment and automatic rollback.

\subsection{Exercise 7: Complete CI/CD Pipeline (Advanced)}

Build end-to-end deployment pipeline:

\begin{enumerate}
    \item \textbf{Testing Stage}:
    \begin{itemize}
        \item Unit tests with coverage
        \item Integration tests
        \item Model performance validation
    \end{itemize}

    \item \textbf{Build Stage}:
    \begin{itemize}
        \item Docker image build
        \item Security scanning
        \item Image registry push
    \end{itemize}

    \item \textbf{Deploy Stage}:
    \begin{itemize}
        \item Staging deployment with canary
        \item Smoke tests
        \item Production deployment approval
        \item Production canary deployment
    \end{itemize}

    \item \textbf{Monitoring}:
    \begin{itemize}
        \item Automated metrics collection
        \item Alerting on SLO violations
        \item Automatic rollback on failure
    \end{itemize}
\end{enumerate}

Implement with GitHub Actions, GitLab CI, or Jenkins.

\section{Summary}

This chapter provided comprehensive production deployment strategies:

\begin{itemize}
    \item \textbf{Model Serving API}: FastAPI with request validation, error handling, health checks, and metrics endpoints for production-grade ML services

    \item \textbf{Containerization}: Multi-stage Docker builds with security best practices, resource limits, and optimized layer caching for efficient deployments

    \item \textbf{Deployment Strategies}: Blue-green for zero-downtime with instant rollback, canary for gradual rollout with risk mitigation, rolling for resource-efficient updates

    \item \textbf{Kubernetes Orchestration}: Deployment configurations with resource management, HPA for auto-scaling, health probes for reliability

    \item \textbf{Model Versioning}: Centralized registry for version tracking, promotion workflows (dev → staging → production), rollback capabilities

    \item \textbf{CI/CD Automation}: Automated testing, validation, building, and deployment pipelines with security scanning and smoke tests
\end{itemize}

Deployment engineering transforms models from experimental code into reliable production systems. By implementing proper containerization, deployment strategies, monitoring, and automation, teams can deploy models confidently with minimal downtime, rapid rollback capabilities, and continuous validation of production performance.

The key insight: deployment is not a one-time event but a continuous process requiring rigorous testing, gradual rollout, comprehensive monitoring, and instant rollback capabilities. Organizations that master deployment engineering achieve higher model velocity, lower incident rates, and greater business impact from ML investments.
