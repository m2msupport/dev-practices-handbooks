\chapter{Checklists, Templates, and Resources}

\section{Introduction}

This appendix provides production-ready templates, checklists, and automation frameworks for implementing ML engineering best practices. Use these resources to accelerate project setup, ensure quality standards, and maintain operational excellence.

\section{Project Health Assessment Framework}

Automated framework for assessing ML project health across multiple dimensions.

\subsection{HealthCheckFramework: Automated Assessment}

\begin{lstlisting}[language=Python, caption={ML Project Health Assessment}]
from dataclasses import dataclass, field
from typing import Dict, List, Optional
from pathlib import Path
from enum import Enum
import subprocess
import json
import logging

logger = logging.getLogger(__name__)

class HealthCategory(Enum):
    """Project health categories."""
    CODE_QUALITY = "code_quality"
    TESTING = "testing"
    DOCUMENTATION = "documentation"
    VERSIONING = "versioning"
    DEPLOYMENT = "deployment"
    MONITORING = "monitoring"

@dataclass
class HealthCheck:
    """
    Individual health check.

    Attributes:
        name: Check name
        category: Health category
        description: What this checks
        check_function: Function to run check
        weight: Importance weight (0-1)
        required: Whether this is mandatory
    """
    name: str
    category: HealthCategory
    description: str
    check_function: callable
    weight: float = 1.0
    required: bool = False

@dataclass
class HealthScore:
    """
    Health assessment result.

    Attributes:
        category: Category assessed
        score: Score 0-100
        passed_checks: Number of passed checks
        total_checks: Total number of checks
        issues: List of failed checks
        recommendations: Improvement suggestions
    """
    category: HealthCategory
    score: float
    passed_checks: int
    total_checks: int
    issues: List[str] = field(default_factory=list)
    recommendations: List[str] = field(default_factory=list)

class HealthCheckFramework:
    """
    Comprehensive ML project health assessment.

    Evaluates code quality, testing, documentation, deployment
    readiness, and operational maturity.

    Example:
        >>> framework = HealthCheckFramework(project_path=".")
        >>> results = framework.assess_health()
        >>> print(f"Overall Score: {results['overall_score']:.1f}/100")
    """

    def __init__(self, project_path: str = "."):
        """
        Initialize health checker.

        Args:
            project_path: Path to ML project
        """
        self.project_path = Path(project_path)
        self.checks: Dict[HealthCategory, List[HealthCheck]] = {
            category: [] for category in HealthCategory
        }

        # Register default checks
        self._register_default_checks()

        logger.info(f"Initialized health checker for {project_path}")

    def _register_default_checks(self):
        """Register default health checks."""

        # Code Quality Checks
        self.add_check(HealthCheck(
            name="code_formatting",
            category=HealthCategory.CODE_QUALITY,
            description="Code follows Black formatting",
            check_function=self._check_black_formatting,
            weight=0.8,
            required=True
        ))

        self.add_check(HealthCheck(
            name="linting",
            category=HealthCategory.CODE_QUALITY,
            description="Code passes flake8 linting",
            check_function=self._check_flake8,
            weight=1.0,
            required=True
        ))

        self.add_check(HealthCheck(
            name="type_hints",
            category=HealthCategory.CODE_QUALITY,
            description="Type hints coverage",
            check_function=self._check_type_hints,
            weight=0.6
        ))

        # Testing Checks
        self.add_check(HealthCheck(
            name="test_coverage",
            category=HealthCategory.TESTING,
            description="Unit test coverage >= 80%",
            check_function=self._check_test_coverage,
            weight=1.0,
            required=True
        ))

        self.add_check(HealthCheck(
            name="integration_tests",
            category=HealthCategory.TESTING,
            description="Integration tests exist",
            check_function=self._check_integration_tests,
            weight=0.8
        ))

        # Documentation Checks
        self.add_check(HealthCheck(
            name="readme",
            category=HealthCategory.DOCUMENTATION,
            description="README.md exists and comprehensive",
            check_function=self._check_readme,
            weight=1.0,
            required=True
        ))

        self.add_check(HealthCheck(
            name="api_documentation",
            category=HealthCategory.DOCUMENTATION,
            description="API documentation exists",
            check_function=self._check_api_docs,
            weight=0.7
        ))

        # Versioning Checks
        self.add_check(HealthCheck(
            name="git_repo",
            category=HealthCategory.VERSIONING,
            description="Project is a Git repository",
            check_function=self._check_git_repo,
            weight=1.0,
            required=True
        ))

        self.add_check(HealthCheck(
            name="requirements_file",
            category=HealthCategory.VERSIONING,
            description="Dependencies tracked",
            check_function=self._check_requirements,
            weight=1.0,
            required=True
        ))

        # Deployment Checks
        self.add_check(HealthCheck(
            name="dockerfile",
            category=HealthCategory.DEPLOYMENT,
            description="Dockerfile exists",
            check_function=self._check_dockerfile,
            weight=0.8
        ))

        self.add_check(HealthCheck(
            name="ci_cd",
            category=HealthCategory.DEPLOYMENT,
            description="CI/CD pipeline configured",
            check_function=self._check_ci_cd,
            weight=1.0
        ))

        # Monitoring Checks
        self.add_check(HealthCheck(
            name="logging",
            category=HealthCategory.MONITORING,
            description="Structured logging implemented",
            check_function=self._check_logging,
            weight=0.8
        ))

        self.add_check(HealthCheck(
            name="metrics",
            category=HealthCategory.MONITORING,
            description="Metrics instrumentation present",
            check_function=self._check_metrics,
            weight=0.7
        ))

    def add_check(self, check: HealthCheck):
        """Add custom health check."""
        self.checks[check.category].append(check)

    def assess_health(self) -> Dict:
        """
        Run all health checks and generate report.

        Returns:
            Dictionary with assessment results
        """
        logger.info("Running health assessment")

        category_scores = {}
        all_issues = []
        all_recommendations = []

        for category in HealthCategory:
            score = self._assess_category(category)
            category_scores[category.value] = score

            all_issues.extend(score.issues)
            all_recommendations.extend(score.recommendations)

        # Calculate overall score (weighted average)
        category_weights = {
            HealthCategory.CODE_QUALITY: 0.25,
            HealthCategory.TESTING: 0.25,
            HealthCategory.DOCUMENTATION: 0.15,
            HealthCategory.VERSIONING: 0.10,
            HealthCategory.DEPLOYMENT: 0.15,
            HealthCategory.MONITORING: 0.10
        }

        overall_score = sum(
            score.score * category_weights[category]
            for category, score in category_scores.items()
        )

        # Determine health level
        if overall_score >= 90:
            health_level = "Excellent"
        elif overall_score >= 75:
            health_level = "Good"
        elif overall_score >= 60:
            health_level = "Fair"
        else:
            health_level = "Needs Improvement"

        results = {
            'overall_score': overall_score,
            'health_level': health_level,
            'category_scores': {
                cat.value: {
                    'score': score.score,
                    'passed': score.passed_checks,
                    'total': score.total_checks
                }
                for cat, score in category_scores.items()
            },
            'issues': all_issues,
            'recommendations': all_recommendations[:10],  # Top 10
            'passed_required': self._check_required_checks(category_scores)
        }

        self._log_summary(results)

        return results

    def _assess_category(self, category: HealthCategory) -> HealthScore:
        """Assess single health category."""
        checks = self.checks[category]

        if not checks:
            return HealthScore(
                category=category,
                score=100.0,
                passed_checks=0,
                total_checks=0
            )

        passed = 0
        issues = []
        recommendations = []

        for check in checks:
            try:
                result = check.check_function(self.project_path)

                if result:
                    passed += 1
                else:
                    issues.append(f"{check.name}: {check.description}")
                    recommendations.append(
                        f"Fix {check.name} to improve {category.value}"
                    )

            except Exception as e:
                logger.error(f"Check {check.name} failed: {e}")
                issues.append(f"{check.name}: Error running check")

        # Weighted score
        total_weight = sum(c.weight for c in checks)
        passed_weight = sum(
            c.weight for c in checks
            if c.check_function(self.project_path)
        )

        score = (passed_weight / total_weight * 100) if total_weight > 0 else 0

        return HealthScore(
            category=category,
            score=score,
            passed_checks=passed,
            total_checks=len(checks),
            issues=issues,
            recommendations=recommendations
        )

    def _check_required_checks(
        self,
        category_scores: Dict[HealthCategory, HealthScore]
    ) -> bool:
        """Check if all required checks passed."""
        for category in HealthCategory:
            checks = self.checks[category]
            required_checks = [c for c in checks if c.required]

            for check in required_checks:
                if not check.check_function(self.project_path):
                    return False

        return True

    # Individual check implementations

    def _check_black_formatting(self, path: Path) -> bool:
        """Check if code is Black formatted."""
        try:
            result = subprocess.run(
                ["black", "--check", str(path / "src")],
                capture_output=True,
                timeout=30
            )
            return result.returncode == 0
        except Exception:
            return False

    def _check_flake8(self, path: Path) -> bool:
        """Check flake8 linting."""
        try:
            result = subprocess.run(
                ["flake8", str(path / "src")],
                capture_output=True,
                timeout=30
            )
            return result.returncode == 0
        except Exception:
            return False

    def _check_type_hints(self, path: Path) -> bool:
        """Check type hint coverage."""
        try:
            result = subprocess.run(
                ["mypy", str(path / "src"), "--ignore-missing-imports"],
                capture_output=True,
                timeout=30
            )
            # Accept if mypy runs without fatal errors
            return "error" not in result.stdout.decode().lower()
        except Exception:
            return False

    def _check_test_coverage(self, path: Path) -> bool:
        """Check test coverage >= 80%."""
        try:
            result = subprocess.run(
                ["pytest", "--cov=src", "--cov-report=json"],
                cwd=path,
                capture_output=True,
                timeout=60
            )

            # Parse coverage report
            coverage_file = path / "coverage.json"
            if coverage_file.exists():
                with open(coverage_file) as f:
                    coverage = json.load(f)
                    total_coverage = coverage['totals']['percent_covered']
                    return total_coverage >= 80.0

            return False
        except Exception:
            return False

    def _check_integration_tests(self, path: Path) -> bool:
        """Check if integration tests exist."""
        integration_test_paths = [
            path / "tests" / "integration",
            path / "tests" / "test_integration.py"
        ]

        return any(p.exists() for p in integration_test_paths)

    def _check_readme(self, path: Path) -> bool:
        """Check if README exists and has minimum content."""
        readme_path = path / "README.md"

        if not readme_path.exists():
            return False

        content = readme_path.read_text()

        # Check for essential sections
        required_sections = [
            "install", "usage", "contributing"
        ]

        return all(
            section in content.lower()
            for section in required_sections
        )

    def _check_api_docs(self, path: Path) -> bool:
        """Check for API documentation."""
        docs_paths = [
            path / "docs",
            path / "API.md"
        ]

        return any(p.exists() for p in docs_paths)

    def _check_git_repo(self, path: Path) -> bool:
        """Check if project is a Git repo."""
        return (path / ".git").exists()

    def _check_requirements(self, path: Path) -> bool:
        """Check if dependencies are tracked."""
        requirement_files = [
            "requirements.txt",
            "environment.yml",
            "pyproject.toml",
            "Pipfile"
        ]

        return any((path / f).exists() for f in requirement_files)

    def _check_dockerfile(self, path: Path) -> bool:
        """Check if Dockerfile exists."""
        return (path / "Dockerfile").exists()

    def _check_ci_cd(self, path: Path) -> bool:
        """Check for CI/CD configuration."""
        ci_paths = [
            path / ".github" / "workflows",
            path / ".gitlab-ci.yml",
            path / "Jenkinsfile",
            path / ".circleci"
        ]

        return any(p.exists() for p in ci_paths)

    def _check_logging(self, path: Path) -> bool:
        """Check for structured logging."""
        # Search for logging configuration
        src_path = path / "src"

        if not src_path.exists():
            return False

        # Check for logging imports
        for py_file in src_path.rglob("*.py"):
            content = py_file.read_text()
            if "import logging" in content or "from logging" in content:
                return True

        return False

    def _check_metrics(self, path: Path) -> bool:
        """Check for metrics instrumentation."""
        src_path = path / "src"

        if not src_path.exists():
            return False

        # Check for metrics libraries
        metrics_libraries = [
            "prometheus_client",
            "statsd",
            "datadog"
        ]

        for py_file in src_path.rglob("*.py"):
            content = py_file.read_text()
            if any(lib in content for lib in metrics_libraries):
                return True

        return False

    def _log_summary(self, results: Dict):
        """Log assessment summary."""
        logger.info("=" * 70)
        logger.info("ML PROJECT HEALTH ASSESSMENT")
        logger.info("=" * 70)
        logger.info(f"Overall Score: {results['overall_score']:.1f}/100")
        logger.info(f"Health Level: {results['health_level']}")
        logger.info("")
        logger.info("Category Scores:")

        for category, scores in results['category_scores'].items():
            logger.info(
                f"  {category}: {scores['score']:.1f}/100 "
                f"({scores['passed']}/{scores['total']} checks passed)"
            )

        if results['issues']:
            logger.info("")
            logger.info(f"Issues Found: {len(results['issues'])}")
            for issue in results['issues'][:5]:
                logger.info(f"  - {issue}")

        logger.info("=" * 70)

    def generate_report(self, results: Dict, output_path: str = "health_report.md"):
        """
        Generate markdown health report.

        Args:
            results: Assessment results
            output_path: Path for report
        """
        lines = ["# ML Project Health Report\n"]

        # Overall score with emoji
        if results['overall_score'] >= 90:
            emoji = "(GREEN)"
        elif results['overall_score'] >= 75:
            emoji = "(YELLOW)"
        else:
            emoji = "(RED)"

        lines.append(f"{emoji} **Overall Score**: {results['overall_score']:.1f}/100\n")
        lines.append(f"**Health Level**: {results['health_level']}\n")
        lines.append("")

        # Category breakdown
        lines.append("## Category Scores\n")

        for category, scores in results['category_scores'].items():
            status = "(PASS)" if scores['score'] >= 75 else "(WARN)"
            lines.append(
                f"{status} **{category.title()}**: {scores['score']:.1f}/100 "
                f"({scores['passed']}/{scores['total']} checks passed)\n"
            )

        # Issues
        if results['issues']:
            lines.append("\n## Issues\n")
            for issue in results['issues']:
                lines.append(f"- {issue}\n")

        # Recommendations
        if results['recommendations']:
            lines.append("\n## Recommendations\n")
            for i, rec in enumerate(results['recommendations'][:10], 1):
                lines.append(f"{i}. {rec}\n")

        # Save report
        with open(output_path, 'w') as f:
            f.writelines(lines)

        logger.info(f"Health report saved to {output_path}")
\end{lstlisting}

\section{ML Project Templates}

\subsection{ProjectTemplate: Automated Project Setup}

\begin{lstlisting}[language=Python, caption={ML Project Template Generator}]
from pathlib import Path
from typing import Dict, List, Optional
import logging

logger = logging.getLogger(__name__)

class ProjectTemplate:
    """
    Generate standardized ML project structure.

    Creates directories, configuration files, and initial code.

    Example:
        >>> template = ProjectTemplate("my_ml_project")
        >>> template.generate()
    """

    def __init__(
        self,
        project_name: str,
        project_type: str = "ml_service",
        include_docker: bool = True,
        include_ci: bool = True
    ):
        """
        Initialize project template.

        Args:
            project_name: Name of project
            project_type: "ml_service", "research", or "batch"
            include_docker: Include Docker configuration
            include_ci: Include CI/CD configuration
        """
        self.project_name = project_name
        self.project_type = project_type
        self.include_docker = include_docker
        self.include_ci = include_ci

        self.project_path = Path(project_name)

    def generate(self):
        """Generate complete project structure."""
        logger.info(f"Generating project: {self.project_name}")

        # Create directory structure
        self._create_directories()

        # Generate configuration files
        self._create_pyproject_toml()
        self._create_requirements_txt()
        self._create_environment_yml()

        if self.include_docker:
            self._create_dockerfile()
            self._create_docker_compose()

        if self.include_ci:
            self._create_github_actions()

        # Create initial code files
        self._create_src_structure()
        self._create_tests()

        # Create documentation
        self._create_readme()
        self._create_contributing()

        # Create configuration
        self._create_config_files()

        # Create pre-commit hooks
        self._create_precommit_config()

        logger.info(f"Project generated at {self.project_path}")

    def _create_directories(self):
        """Create project directory structure."""
        directories = [
            "",  # Root
            "src",
            "src/models",
            "src/data",
            "src/features",
            "src/utils",
            "tests",
            "tests/unit",
            "tests/integration",
            "notebooks",
            "data/raw",
            "data/processed",
            "data/features",
            "models/trained",
            "models/optimized",
            "config",
            "docs",
            "scripts",
            ".github/workflows" if self.include_ci else None
        ]

        for directory in directories:
            if directory:
                (self.project_path / directory).mkdir(parents=True, exist_ok=True)

    def _create_pyproject_toml(self):
        """Create pyproject.toml."""
        content = f'''[tool.poetry]
name = "{self.project_name}"
version = "0.1.0"
description = "ML project for {self.project_name}"
authors = ["Your Name <your.email@example.com>"]
readme = "README.md"

[tool.poetry.dependencies]
python = "^3.9"
numpy = "^1.24.0"
pandas = "^2.0.0"
scikit-learn = "^1.3.0"
torch = "^2.0.0"
fastapi = "^0.104.0"
pydantic = "^2.0.0"
pyyaml = "^6.0"
python-dotenv = "^1.0.0"

[tool.poetry.group.dev.dependencies]
pytest = "^7.4.0"
pytest-cov = "^4.1.0"
black = "^23.7.0"
flake8 = "^6.0.0"
mypy = "^1.4.0"
pre-commit = "^3.3.0"

[tool.black]
line-length = 100
target-version = ['py39']
include = '\\.pyi?$'

[tool.isort]
profile = "black"
line_length = 100

[tool.mypy]
python_version = "3.9"
warn_return_any = true
warn_unused_configs = true
disallow_untyped_defs = false

[tool.pytest.ini_options]
testpaths = ["tests"]
python_files = "test_*.py"
python_functions = "test_*"
addopts = "--cov=src --cov-report=html --cov-report=term"

[build-system]
requires = ["poetry-core"]
build-backend = "poetry.core.masonry.api"
'''

        (self.project_path / "pyproject.toml").write_text(content)

    def _create_requirements_txt(self):
        """Create requirements.txt."""
        content = '''# Core ML Libraries
numpy==1.24.0
pandas==2.0.0
scikit-learn==1.3.0
torch==2.0.0

# API Framework
fastapi==0.104.0
uvicorn==0.23.0
pydantic==2.0.0

# Utilities
pyyaml==6.0
python-dotenv==1.0.0
requests==2.31.0

# Monitoring
prometheus-client==0.17.0

# Development
pytest==7.4.0
pytest-cov==4.1.0
black==23.7.0
flake8==6.0.0
mypy==1.4.0
'''

        (self.project_path / "requirements.txt").write_text(content)

    def _create_environment_yml(self):
        """Create environment.yml for conda."""
        content = f'''name: {self.project_name}
channels:
  - conda-forge
  - defaults
dependencies:
  - python=3.9
  - pip
  - pip:
    - -r requirements.txt
'''

        (self.project_path / "environment.yml").write_text(content)

    def _create_dockerfile(self):
        """Create Dockerfile."""
        content = '''FROM python:3.9-slim

WORKDIR /app

# Install system dependencies
RUN apt-get update && apt-get install -y \\
    build-essential \\
    && rm -rf /var/lib/apt/lists/*

# Copy requirements
COPY requirements.txt .

# Install Python dependencies
RUN pip install --no-cache-dir -r requirements.txt

# Copy application code
COPY src/ ./src/
COPY config/ ./config/

# Create non-root user
RUN useradd -m -u 1000 ml-user && chown -R ml-user:ml-user /app
USER ml-user

# Expose port
EXPOSE 8000

# Run application
CMD ["uvicorn", "src.api.main:app", "--host", "0.0.0.0", "--port", "8000"]
'''

        (self.project_path / "Dockerfile").write_text(content)

    def _create_docker_compose(self):
        """Create docker-compose.yml."""
        content = f'''version: '3.8'

services:
  {self.project_name}:
    build: .
    ports:
      - "8000:8000"
    environment:
      - ENVIRONMENT=development
    volumes:
      - ./config:/app/config
      - ./models:/app/models
    restart: unless-stopped

  prometheus:
    image: prom/prometheus:latest
    ports:
      - "9090:9090"
    volumes:
      - ./config/prometheus.yml:/etc/prometheus/prometheus.yml
    command:
      - '--config.file=/etc/prometheus/prometheus.yml'

  grafana:
    image: grafana/grafana:latest
    ports:
      - "3000:3000"
    environment:
      - GF_SECURITY_ADMIN_PASSWORD=admin
    volumes:
      - grafana-storage:/var/lib/grafana

volumes:
  grafana-storage:
'''

        (self.project_path / "docker-compose.yml").write_text(content)

    def _create_github_actions(self):
        """Create GitHub Actions CI/CD."""
        content = '''name: CI/CD

on:
  push:
    branches: [ main, develop ]
  pull_request:
    branches: [ main ]

jobs:
  test:
    runs-on: ubuntu-latest

    steps:
    - uses: actions/checkout@v3

    - name: Set up Python
      uses: actions/setup-python@v4
      with:
        python-version: '3.9'

    - name: Install dependencies
      run: |
        python -m pip install --upgrade pip
        pip install -r requirements.txt
        pip install -r requirements-dev.txt

    - name: Lint with flake8
      run: |
        flake8 src/ --count --select=E9,F63,F7,F82 --show-source --statistics
        flake8 src/ --count --max-line-length=100 --statistics

    - name: Check formatting with black
      run: black --check src/

    - name: Type check with mypy
      run: mypy src/ --ignore-missing-imports
      continue-on-error: true

    - name: Run tests
      run: pytest tests/ -v --cov=src --cov-report=xml

    - name: Upload coverage
      uses: codecov/codecov-action@v3
      with:
        file: ./coverage.xml
'''

        ci_path = self.project_path / ".github" / "workflows"
        ci_path.mkdir(parents=True, exist_ok=True)
        (ci_path / "ci.yml").write_text(content)

    def _create_src_structure(self):
        """Create initial source code structure."""
        # Main API file
        api_content = '''"""Main API application."""
from fastapi import FastAPI
from src.models.predictor import ModelPredictor

app = FastAPI(title="ML API")
predictor = ModelPredictor()

@app.get("/health")
def health_check():
    """Health check endpoint."""
    return {"status": "healthy"}

@app.post("/predict")
def predict(features: dict):
    """Make prediction."""
    prediction = predictor.predict(features)
    return {"prediction": prediction}
'''

        api_path = self.project_path / "src" / "api"
        api_path.mkdir(exist_ok=True)
        (api_path / "__init__.py").touch()
        (api_path / "main.py").write_text(api_content)

        # Model predictor
        predictor_content = '''"""Model prediction logic."""
import logging

logger = logging.getLogger(__name__)

class ModelPredictor:
    """Handle model predictions."""

    def __init__(self):
        """Initialize predictor."""
        self.model = None
        self._load_model()

    def _load_model(self):
        """Load trained model."""
        # TODO: Implement model loading
        logger.info("Model loaded")

    def predict(self, features: dict):
        """Make prediction."""
        # TODO: Implement prediction logic
        return 0.5
'''

        (self.project_path / "src" / "models" / "__init__.py").touch()
        (self.project_path / "src" / "models" / "predictor.py").write_text(predictor_content)

    def _create_tests(self):
        """Create initial test files."""
        test_content = '''"""Test model predictor."""
import pytest
from src.models.predictor import ModelPredictor

def test_predictor_initialization():
    """Test predictor initializes correctly."""
    predictor = ModelPredictor()
    assert predictor is not None

def test_prediction():
    """Test prediction returns expected format."""
    predictor = ModelPredictor()
    result = predictor.predict({"feature1": 1.0})
    assert isinstance(result, (int, float))
'''

        (self.project_path / "tests" / "__init__.py").touch()
        (self.project_path / "tests" / "unit" / "__init__.py").touch()
        (self.project_path / "tests" / "unit" / "test_predictor.py").write_text(test_content)

    def _create_readme(self):
        """Create README.md."""
        content = f'''# {self.project_name}

## Overview

ML project for {self.project_name}.

## Installation

```bash
# Using pip
pip install -r requirements.txt

# Using conda
conda env create -f environment.yml
conda activate {self.project_name}

# Using poetry
poetry install
```

## Usage

```python
from src.models.predictor import ModelPredictor

predictor = ModelPredictor()
prediction = predictor.predict({{"feature1": 1.0}})
```

### API

Start the API server:

```bash
uvicorn src.api.main:app --reload
```

### Docker

```bash
docker-compose up
```

## Development

```bash
# Run tests
pytest

# Format code
black src/ tests/

# Lint code
flake8 src/ tests/

# Type check
mypy src/
```

## Project Structure

```
{self.project_name}/
|-- src/              # Source code
|   |-- api/          # API endpoints
|   |-- models/       # Model logic
|   |-- data/         # Data processing
|   +-- features/     # Feature engineering
|-- tests/            # Tests
|-- notebooks/        # Jupyter notebooks
|-- data/             # Data storage
|-- models/           # Trained models
|-- config/           # Configuration
+-- docs/             # Documentation
```

## Contributing

1. Fork the repository
2. Create feature branch (`git checkout -b feature/amazing-feature`)
3. Commit changes (`git commit -m 'Add amazing feature'`)
4. Push to branch (`git push origin feature/amazing-feature`)
5. Open Pull Request

## License

MIT License
'''

        (self.project_path / "README.md").write_text(content)

    def _create_contributing(self):
        """Create CONTRIBUTING.md."""
        content = '''# Contributing Guidelines

## Development Setup

1. Clone repository
2. Install dependencies: `pip install -r requirements-dev.txt`
3. Install pre-commit hooks: `pre-commit install`

## Code Standards

- Follow PEP 8 style guide
- Use Black for formatting (line length 100)
- Use type hints where appropriate
- Write docstrings for all public functions
- Maintain test coverage above 80%

## Testing

- Write unit tests for all new code
- Run tests before committing: `pytest`
- Ensure all tests pass
- Check coverage: `pytest --cov=src`

## Pull Request Process

1. Update README if needed
2. Update tests
3. Ensure CI passes
4. Request review from maintainers
'''

        (self.project_path / "CONTRIBUTING.md").write_text(content)

    def _create_config_files(self):
        """Create configuration files."""
        # Development config
        dev_config = '''# Development Configuration
environment: development

model:
  name: "ml_model"
  version: "v1.0"
  path: "models/trained/model.pkl"

api:
  host: "0.0.0.0"
  port: 8000
  workers: 4

logging:
  level: "DEBUG"
  format: "json"

monitoring:
  enabled: true
  prometheus_port: 9090
'''

        (self.project_path / "config" / "development.yaml").write_text(dev_config)

        # Production config
        prod_config = '''# Production Configuration
environment: production

model:
  name: "ml_model"
  version: "v1.0"
  path: "models/trained/model.pkl"

api:
  host: "0.0.0.0"
  port: 8000
  workers: 8

logging:
  level: "INFO"
  format: "json"

monitoring:
  enabled: true
  prometheus_port: 9090
'''

        (self.project_path / "config" / "production.yaml").write_text(prod_config)

    def _create_precommit_config(self):
        """Create .pre-commit-config.yaml."""
        content = '''repos:
  - repo: https://github.com/pre-commit/pre-commit-hooks
    rev: v4.4.0
    hooks:
      - id: trailing-whitespace
      - id: end-of-file-fixer
      - id: check-yaml
      - id: check-added-large-files
      - id: check-json

  - repo: https://github.com/psf/black
    rev: 23.7.0
    hooks:
      - id: black
        language_version: python3.9

  - repo: https://github.com/PyCQA/flake8
    rev: 6.0.0
    hooks:
      - id: flake8
        args: ['--max-line-length=100']

  - repo: https://github.com/pre-commit/mirrors-mypy
    rev: v1.4.1
    hooks:
      - id: mypy
        additional_dependencies: [types-all]
'''

        (self.project_path / ".pre-commit-config.yaml").write_text(content)

        # .gitignore
        gitignore_content = '''# Python
__pycache__/
*.py[cod]
*$py.class
*.so
.Python
env/
venv/
ENV/

# Testing
.pytest_cache/
.coverage
htmlcov/
*.cover

# IDEs
.vscode/
.idea/
*.swp
*.swo

# Models and Data
models/trained/*.pkl
models/trained/*.h5
data/raw/*
data/processed/*
!data/raw/.gitkeep
!data/processed/.gitkeep

# Logs
*.log
logs/

# OS
.DS_Store
Thumbs.db
'''

        (self.project_path / ".gitignore").write_text(gitignore_content)
\end{lstlisting}

\section{Deployment Checklists}

\subsection{Pre-Deployment Checklist}

\begin{enumerate}
    \item \textbf{Code Quality}
    \begin{itemize}
        \item[ ] All code follows style guide (Black, flake8 passing)
        \item[ ] Type hints added to public functions
        \item[ ] Code review completed and approved
        \item[ ] No commented-out code or TODOs
    \end{itemize}

    \item \textbf{Testing}
    \begin{itemize}
        \item[ ] Unit test coverage $\geq$ 80\%
        \item[ ] Integration tests pass
        \item[ ] Performance tests pass (latency, throughput)
        \item[ ] Load testing completed
    \end{itemize}

    \item \textbf{Model Validation}
    \begin{itemize}
        \item[ ] Model accuracy meets requirements
        \item[ ] Fairness metrics evaluated and passing
        \item[ ] Model card created and reviewed
        \item[ ] A/B test results favorable
    \end{itemize}

    \item \textbf{Documentation}
    \begin{itemize}
        \item[ ] README updated
        \item[ ] API documentation current
        \item[ ] Runbook created
        \item[ ] Architecture diagram updated
    \end{itemize}

    \item \textbf{Infrastructure}
    \begin{itemize}
        \item[ ] Docker image builds successfully
        \item[ ] Kubernetes manifests validated
        \item[ ] Resource limits configured
        \item[ ] Auto-scaling rules defined
    \end{itemize}

    \item \textbf{Security}
    \begin{itemize}
        \item[ ] Dependency vulnerabilities scanned
        \item[ ] Secrets managed securely (not in code)
        \item[ ] TLS/SSL configured
        \item[ ] Access controls reviewed
    \end{itemize}

    \item \textbf{Monitoring}
    \begin{itemize}
        \item[ ] Logging configured and tested
        \item[ ] Metrics instrumentation added
        \item[ ] Alerts configured
        \item[ ] Dashboard created
    \end{itemize}

    \item \textbf{Compliance}
    \begin{itemize}
        \item[ ] GDPR compliance verified
        \item[ ] Data retention policies implemented
        \item[ ] Audit trail configured
        \item[ ] Ethics review completed (if required)
    \end{itemize}

    \item \textbf{Rollback Plan}
    \begin{itemize}
        \item[ ] Previous version identified
        \item[ ] Rollback procedure tested
        \item[ ] Rollback triggers defined
        \item[ ] Communication plan ready
    \end{itemize}

    \item \textbf{Communication}
    \begin{itemize}
        \item[ ] Stakeholders notified of deployment
        \item[ ] Maintenance window scheduled (if needed)
        \item[ ] On-call rotation updated
        \item[ ] Incident response plan ready
    \end{itemize}
\end{enumerate}

\subsection{Post-Deployment Checklist}

\begin{enumerate}
    \item \textbf{Immediate Validation (0-30 minutes)}
    \begin{itemize}
        \item[ ] Health checks passing
        \item[ ] All pods/instances running
        \item[ ] No error spikes in logs
        \item[ ] Latency within SLO (p95, p99)
        \item[ ] Traffic routing correctly
    \end{itemize}

    \item \textbf{Short-term Monitoring (1-24 hours)}
    \begin{itemize}
        \item[ ] Model predictions reasonable
        \item[ ] No unexpected errors
        \item[ ] Resource utilization normal
        \item[ ] User feedback monitored
        \item[ ] Business metrics stable
    \end{itemize}

    \item \textbf{Long-term Validation (1-7 days)}
    \begin{itemize}
        \item[ ] Model performance metrics stable
        \item[ ] No data drift detected
        \item[ ] Cost within budget
        \item[ ] SLOs consistently met
        \item[ ] No critical alerts
    \end{itemize}
\end{enumerate}

\section{Runbook Template}

\subsection{Service Runbook}

\begin{lstlisting}[style=yaml, caption={runbook.yml}]
service_name: ML Prediction Service
version: v2.1
last_updated: 2024-01-15
on_call: ml-team@company.com

# Service Overview
description: >
  Provides ML predictions for fraud detection.
  Handles 10M requests/day with <100ms latency SLO.

dependencies:
  - name: PostgreSQL
    purpose: Feature store
    contact: data-team@company.com
  - name: Redis
    purpose: Caching layer
    contact: infrastructure@company.com

# Operational Procedures

## Service Management

start_service: |
  kubectl apply -f k8s/deployment.yaml
  kubectl rollout status deployment/ml-service

stop_service: |
  kubectl scale deployment/ml-service --replicas=0

restart_service: |
  kubectl rollout restart deployment/ml-service

check_health: |
  curl https://api.company.com/health
  # Expected: {"status": "healthy"}

# Troubleshooting

## High Latency

symptoms:
  - P95 latency > 100ms
  - User complaints about slowness

investigation:
  1. Check current latency:
     kubectl logs deployment/ml-service | grep "latency"

  2. Check resource usage:
     kubectl top pods -l app=ml-service

  3. Check cache hit rate:
     curl https://api.company.com/metrics | grep cache_hit_rate

resolution:
  - If CPU > 80%: Scale up instances
  - If memory > 80%: Increase memory limits
  - If cache hit rate < 50%: Warm cache or increase size

## Prediction Errors

symptoms:
  - Error rate > 1%
  - Alerts firing

investigation:
  1. Check error logs:
     kubectl logs deployment/ml-service --tail=100 | grep ERROR

  2. Check model version:
     curl https://api.company.com/model-info

  3. Check feature availability:
     psql -h feature-store -c "SELECT COUNT(*) FROM features"

resolution:
  - If model loading failed: Rollback to previous version
  - If features unavailable: Check feature pipeline
  - If unknown error: Page on-call engineer

## Rollback Procedure

steps:
  1. Identify last known good version:
     kubectl rollout history deployment/ml-service

  2. Rollback:
     kubectl rollout undo deployment/ml-service

  3. Verify:
     kubectl rollout status deployment/ml-service
     curl https://api.company.com/health

  4. Monitor for 30 minutes:
     - Check error rate
     - Check latency
     - Check prediction quality

# Monitoring and Alerts

## Key Metrics

latency:
  p50: < 50ms
  p95: < 100ms
  p99: < 200ms

throughput: > 100 req/sec

error_rate: < 1%

availability: > 99.9%

## Alert Definitions

high_latency:
  condition: p95_latency > 100ms for 5 minutes
  severity: warning
  action: Check "High Latency" troubleshooting

critical_error_rate:
  condition: error_rate > 5% for 2 minutes
  severity: critical
  action: Immediate rollback

low_availability:
  condition: availability < 99% for 10 minutes
  severity: critical
  action: Check pod health, scale if needed

# Escalation

level_1:
  - Check runbook
  - Check logs and metrics
  - Apply standard fixes

level_2:
  - If not resolved in 15 minutes
  - Page on-call engineer
  - Slack: #ml-incidents

level_3:
  - If not resolved in 30 minutes
  - Page engineering manager
  - Start incident call
  - Update status page

# Maintenance

regular_tasks:
  daily:
    - Check error logs
    - Review dashboards
    - Verify backups

  weekly:
    - Review model performance
    - Check resource usage trends
    - Update dependencies

  monthly:
    - Model retraining
    - Capacity planning
    - Disaster recovery drill

# Useful Commands

kubectl_commands:
  get_pods: kubectl get pods -l app=ml-service
  get_logs: kubectl logs -f deployment/ml-service
  describe: kubectl describe deployment/ml-service
  exec: kubectl exec -it <pod-name> -- /bin/bash

database_commands:
  connect: psql -h feature-store -U ml_user -d features
  check_size: SELECT pg_size_pretty(pg_database_size('features'))
  recent_features: SELECT * FROM features ORDER BY created_at DESC LIMIT 10

monitoring_commands:
  metrics: curl https://api.company.com/metrics
  health: curl https://api.company.com/health
  model_info: curl https://api.company.com/model-info

# References

documentation: https://wiki.company.com/ml-service
dashboards:
  grafana: https://grafana.company.com/d/ml-service
  prometheus: https://prometheus.company.com/graph
  logs: https://kibana.company.com/app/ml-service

contacts:
  team_lead: lead@company.com
  on_call: ml-team@company.com
  escalation: engineering@company.com
\end{lstlisting}

\section{Resource Lists}

\subsection{Essential Tools}

\textbf{Development}:
\begin{itemize}
    \item \textbf{IDEs}: VS Code, PyCharm, Jupyter Lab
    \item \textbf{Version Control}: Git, DVC, Git LFS
    \item \textbf{Package Management}: Poetry, Conda, pip-tools
    \item \textbf{Code Quality}: Black, flake8, mypy, pylint
\end{itemize}

\textbf{ML Frameworks}:
\begin{itemize}
    \item \textbf{Training}: PyTorch, TensorFlow, scikit-learn
    \item \textbf{Experiment Tracking}: MLflow, Weights \& Biases, Neptune
    \item \textbf{Model Serving}: TorchServe, TensorFlow Serving, BentoML
    \item \textbf{AutoML}: Auto-sklearn, TPOT, H2O AutoML
\end{itemize}

\textbf{Infrastructure}:
\begin{itemize}
    \item \textbf{Containerization}: Docker, Kubernetes, Helm
    \item \textbf{CI/CD}: GitHub Actions, GitLab CI, Jenkins
    \item \textbf{Orchestration}: Airflow, Prefect, Dagster
    \item \textbf{Cloud Platforms}: AWS SageMaker, Google AI Platform, Azure ML
\end{itemize}

\textbf{Monitoring}:
\begin{itemize}
    \item \textbf{Metrics}: Prometheus, Grafana, Datadog
    \item \textbf{Logging}: ELK Stack, Loki, CloudWatch
    \item \textbf{Tracing}: Jaeger, Zipkin, OpenTelemetry
    \item \textbf{APM}: New Relic, Datadog APM, Dynatrace
\end{itemize}

\subsection{Learning Resources}

\textbf{Books}:
\begin{itemize}
    \item \emph{Designing Machine Learning Systems} - Chip Huyen
    \item \emph{Machine Learning Engineering} - Andriy Burkov
    \item \emph{Building Machine Learning Powered Applications} - Emmanuel Ameisen
    \item \emph{Practical MLOps} - Noah Gift, Alfredo Deza
\end{itemize}

\textbf{Online Courses}:
\begin{itemize}
    \item MLOps Specialization (Coursera)
    \item Full Stack Deep Learning
    \item Made With ML
    \item Fast.ai Practical Deep Learning
\end{itemize}

\textbf{Communities}:
\begin{itemize}
    \item MLOps Community Slack
    \item r/MachineLearning
    \item Papers With Code
    \item Kaggle Forums
\end{itemize}

\section{Final Exercise: Complete Project Setup}

\subsection{Exercise: End-to-End ML Project}

Set up a complete production-ready ML project:

\textbf{Part 1: Project Initialization}
\begin{enumerate}
    \item Use ProjectTemplate to generate project structure
    \item Initialize Git repository with proper .gitignore
    \item Set up virtual environment and install dependencies
    \item Configure pre-commit hooks
\end{enumerate}

\textbf{Part 2: Development}
\begin{enumerate}
    \item Implement data pipeline with validation
    \item Train model with experiment tracking (MLflow)
    \item Add unit tests (target 80\% coverage)
    \item Implement API with FastAPI
    \item Add model card documentation
\end{enumerate}

\textbf{Part 3: Quality Assurance}
\begin{enumerate}
    \item Run HealthCheckFramework and fix issues
    \item Implement fairness evaluation
    \item Add monitoring instrumentation (Prometheus)
    \item Create Dockerfile and test locally
    \item Set up CI/CD pipeline
\end{enumerate}

\textbf{Part 4: Deployment}
\begin{enumerate}
    \item Complete pre-deployment checklist
    \item Deploy to staging environment
    \item Run integration tests
    \item Deploy to production with monitoring
    \item Complete post-deployment validation
\end{enumerate}

\textbf{Part 5: Operations}
\begin{enumerate}
    \item Create runbook for service
    \item Set up alerts and dashboards
    \item Document incident response procedures
    \item Conduct failure scenario testing
    \item Schedule first model retrain
\end{enumerate}

\subsection{Success Criteria}

\begin{itemize}
    \item Health check score $\geq$ 85/100
    \item All tests passing in CI/CD
    \item Service deployed and serving predictions
    \item Monitoring dashboard operational
    \item Documentation complete and reviewed
\end{itemize}

\section{Conclusion}

This handbook has covered the complete lifecycle of production ML systems—from reproducible research environments to ethical deployment at scale. The templates, frameworks, and checklists in this appendix accelerate the journey from prototype to production.

\textbf{Key Principles}:
\begin{itemize}
    \item \textbf{Automate Everything}: Manual processes don't scale
    \item \textbf{Measure Continuously}: Instrumentations enables optimization
    \item \textbf{Test Rigorously}: Failures are expensive in production
    \item \textbf{Document Thoroughly}: Future you will thank present you
    \item \textbf{Monitor Proactively}: Detect issues before users do
    \item \textbf{Iterate Systematically}: Small, validated improvements compound
\end{itemize}

Building reliable ML systems requires discipline, tooling, and continuous improvement. Use these resources to establish best practices in your organization and deliver ML systems that provide sustainable business value.
