\chapter{MLOps Automation and CI/CD}

\section{Introduction}

Manual ML deployments fail. A data scientist manually trains a model, copies files to production via SCP, restarts services, and hopes nothing breaks. Two weeks later, nobody remembers which model version is deployed, what data it was trained on, or how to rollback if issues arise. This is the reality for 60\% of ML teams—and why most ML projects never deliver sustained business value.

\subsection{The Manual Deployment Problem}

Consider a fraud detection team that manually deploys models weekly. One Friday afternoon, a data scientist deploys a new model that was accidentally trained on corrupted data. The model approves 95\% of transactions (baseline: 85\%), including fraudulent ones. By Monday morning, \$3M in fraudulent charges have been approved. The team spends 12 hours finding and reverting to the correct model version.

\subsection{Why MLOps Automation Matters}

ML systems differ from traditional software:

\begin{itemize}
    \item \textbf{Code + Data + Model}: Three components that must be versioned together
    \item \textbf{Experimental Nature}: Models evolve through experimentation, requiring rapid iteration
    \item \textbf{Performance Decay}: Models degrade over time, requiring automated retraining
    \item \textbf{Complex Dependencies}: Training environments differ from serving environments
    \item \textbf{Reproducibility}: Exact model recreation requires tracking all inputs and hyperparameters
    \item \textbf{Multiple Stages}: Dev, staging, production require different configurations
\end{itemize}

\subsection{The Cost of Manual MLOps}

Industry evidence shows:
\begin{itemize}
    \item \textbf{Manual deployments} take 4-6 hours and have 40\% failure rate
    \item \textbf{Configuration errors} cause 35\% of production ML incidents
    \item \textbf{Lack of automation} delays model updates by 2-4 weeks
    \item \textbf{Manual rollbacks} take 8-12 hours during incidents
\end{itemize}

\subsection{Chapter Overview}

This chapter provides production-grade MLOps automation:

\begin{enumerate}
    \item \textbf{CI/CD Pipelines}: Automated testing, building, and deployment
    \item \textbf{Training Automation}: Trigger-based retraining with validation
    \item \textbf{Infrastructure as Code}: Terraform/CloudFormation for ML infrastructure
    \item \textbf{Model Promotion}: Automated validation and approval workflows
    \item \textbf{Configuration Management}: Environment-specific settings and secrets
    \item \textbf{Rollback Automation}: Instant reversion to previous versions
    \item \textbf{GitOps}: Git as single source of truth for deployments
\end{enumerate}

\section{CI/CD Pipelines for ML}

Automated pipelines ensure every code and model change is tested, validated, and deployed consistently.

\subsection{CICDManager: Git-Integrated Pipeline}

\begin{lstlisting}[language=Python, caption={Comprehensive CI/CD Framework}]
from dataclasses import dataclass, field
from typing import Dict, List, Optional, Any, Callable
from enum import Enum
from pathlib import Path
from datetime import datetime
import subprocess
import logging
import yaml
import json

logger = logging.getLogger(__name__)

class PipelineStage(Enum):
    """CI/CD pipeline stages."""
    LINT = "lint"
    TEST = "test"
    BUILD = "build"
    SECURITY_SCAN = "security_scan"
    DEPLOY_STAGING = "deploy_staging"
    VALIDATE = "validate"
    DEPLOY_PROD = "deploy_prod"

class DeploymentStatus(Enum):
    """Deployment status."""
    PENDING = "pending"
    RUNNING = "running"
    SUCCESS = "success"
    FAILED = "failed"
    ROLLED_BACK = "rolled_back"

@dataclass
class PipelineConfig:
    """
    CI/CD pipeline configuration.

    Attributes:
        name: Pipeline identifier
        trigger_branch: Git branch that triggers pipeline
        stages: Ordered list of stages to execute
        auto_deploy_staging: Auto-deploy to staging on success
        auto_deploy_prod: Auto-deploy to prod (requires approval)
        slack_webhook: Slack webhook for notifications
        rollback_on_failure: Auto-rollback on validation failure
    """
    name: str
    trigger_branch: str = "main"
    stages: List[PipelineStage] = field(default_factory=list)
    auto_deploy_staging: bool = True
    auto_deploy_prod: bool = False
    slack_webhook: Optional[str] = None
    rollback_on_failure: bool = True

@dataclass
class DeploymentRecord:
    """
    Record of a deployment.

    Attributes:
        deployment_id: Unique identifier
        git_commit: Git commit SHA
        model_version: Model version deployed
        environment: Target environment
        status: Deployment status
        timestamp: When deployment occurred
        duration: Deployment duration in seconds
        artifacts: Deployed artifacts
        rollback_to: Previous version (for rollback)
    """
    deployment_id: str
    git_commit: str
    model_version: str
    environment: str
    status: DeploymentStatus
    timestamp: datetime
    duration: Optional[float] = None
    artifacts: Dict[str, str] = field(default_factory=dict)
    rollback_to: Optional[str] = None

class CICDManager:
    """
    CI/CD pipeline manager for ML projects.

    Integrates with Git, runs automated tests, validates models,
    and manages deployments across environments.

    Example:
        >>> cicd = CICDManager(config, repo_path=".")
        >>> cicd.run_pipeline()
    """

    def __init__(
        self,
        config: PipelineConfig,
        repo_path: str = ".",
        artifacts_path: str = "./artifacts"
    ):
        """
        Initialize CI/CD manager.

        Args:
            config: Pipeline configuration
            repo_path: Path to Git repository
            artifacts_path: Path for build artifacts
        """
        self.config = config
        self.repo_path = Path(repo_path)
        self.artifacts_path = Path(artifacts_path)

        # Create artifacts directory
        self.artifacts_path.mkdir(parents=True, exist_ok=True)

        # Deployment history
        self.deployments: List[DeploymentRecord] = []

        logger.info(f"Initialized CI/CD pipeline: {config.name}")

    def run_pipeline(
        self,
        skip_stages: Optional[List[PipelineStage]] = None
    ) -> bool:
        """
        Execute CI/CD pipeline.

        Args:
            skip_stages: Stages to skip

        Returns:
            True if pipeline succeeded
        """
        skip_stages = skip_stages or []
        start_time = datetime.now()

        logger.info(f"Starting CI/CD pipeline: {self.config.name}")

        # Get Git info
        git_commit = self._get_git_commit()
        git_branch = self._get_git_branch()

        logger.info(f"Git commit: {git_commit}, branch: {git_branch}")

        # Check if branch matches trigger
        if git_branch != self.config.trigger_branch:
            logger.info(
                f"Branch {git_branch} does not match trigger "
                f"{self.config.trigger_branch}, skipping"
            )
            return False

        try:
            # Execute stages
            for stage in self.config.stages:
                if stage in skip_stages:
                    logger.info(f"Skipping stage: {stage.value}")
                    continue

                logger.info(f"Running stage: {stage.value}")

                if stage == PipelineStage.LINT:
                    success = self._run_lint()
                elif stage == PipelineStage.TEST:
                    success = self._run_tests()
                elif stage == PipelineStage.BUILD:
                    success = self._run_build()
                elif stage == PipelineStage.SECURITY_SCAN:
                    success = self._run_security_scan()
                elif stage == PipelineStage.DEPLOY_STAGING:
                    success = self._deploy_staging(git_commit)
                elif stage == PipelineStage.VALIDATE:
                    success = self._validate_deployment()
                elif stage == PipelineStage.DEPLOY_PROD:
                    success = self._deploy_production(git_commit)
                else:
                    logger.warning(f"Unknown stage: {stage}")
                    success = True

                if not success:
                    logger.error(f"Stage {stage.value} failed")
                    self._notify_failure(stage, git_commit)
                    return False

            # Pipeline succeeded
            duration = (datetime.now() - start_time).total_seconds()
            logger.info(
                f"Pipeline completed successfully in {duration:.2f}s"
            )

            self._notify_success(git_commit)

            return True

        except Exception as e:
            logger.error(f"Pipeline failed with exception: {e}")
            self._notify_failure(None, git_commit, str(e))
            return False

    def _get_git_commit(self) -> str:
        """Get current Git commit SHA."""
        result = subprocess.run(
            ["git", "rev-parse", "HEAD"],
            cwd=self.repo_path,
            capture_output=True,
            text=True
        )
        return result.stdout.strip()

    def _get_git_branch(self) -> str:
        """Get current Git branch."""
        result = subprocess.run(
            ["git", "rev-parse", "--abbrev-ref", "HEAD"],
            cwd=self.repo_path,
            capture_output=True,
            text=True
        )
        return result.stdout.strip()

    def _run_lint(self) -> bool:
        """Run code linting."""
        logger.info("Running linters...")

        # Flake8 for Python
        result = subprocess.run(
            ["flake8", "src/", "--max-line-length=100"],
            cwd=self.repo_path,
            capture_output=True
        )

        if result.returncode != 0:
            logger.error(f"Flake8 failed:\n{result.stdout.decode()}")
            return False

        # Black for formatting
        result = subprocess.run(
            ["black", "--check", "src/"],
            cwd=self.repo_path,
            capture_output=True
        )

        if result.returncode != 0:
            logger.error("Code not formatted with Black")
            return False

        # MyPy for type checking
        result = subprocess.run(
            ["mypy", "src/", "--ignore-missing-imports"],
            cwd=self.repo_path,
            capture_output=True
        )

        if result.returncode != 0:
            logger.warning(f"Type checking issues:\n{result.stdout.decode()}")
            # Don't fail on type errors, just warn

        logger.info("Linting passed")
        return True

    def _run_tests(self) -> bool:
        """Run automated tests."""
        logger.info("Running tests...")

        # Unit tests
        result = subprocess.run(
            ["pytest", "tests/", "-v", "--tb=short", "--cov=src"],
            cwd=self.repo_path,
            capture_output=True,
            text=True
        )

        if result.returncode != 0:
            logger.error(f"Tests failed:\n{result.stdout}")
            return False

        # Parse coverage
        coverage_match = None
        for line in result.stdout.split("\n"):
            if "TOTAL" in line:
                coverage_match = line

        if coverage_match:
            logger.info(f"Coverage: {coverage_match}")

        logger.info("Tests passed")
        return True

    def _run_build(self) -> bool:
        """Build artifacts."""
        logger.info("Building artifacts...")

        # Build Docker image
        image_tag = f"ml-model:{self._get_git_commit()[:8]}"

        result = subprocess.run(
            ["docker", "build", "-t", image_tag, "."],
            cwd=self.repo_path,
            capture_output=True
        )

        if result.returncode != 0:
            logger.error(f"Docker build failed:\n{result.stderr.decode()}")
            return False

        # Save image tag
        artifacts = {
            'docker_image': image_tag,
            'timestamp': datetime.now().isoformat()
        }

        artifacts_file = self.artifacts_path / "build_artifacts.json"
        with open(artifacts_file, 'w') as f:
            json.dump(artifacts, f, indent=2)

        logger.info(f"Built Docker image: {image_tag}")
        return True

    def _run_security_scan(self) -> bool:
        """Run security scans."""
        logger.info("Running security scans...")

        # Scan Python dependencies
        result = subprocess.run(
            ["safety", "check", "--json"],
            cwd=self.repo_path,
            capture_output=True,
            text=True
        )

        if result.returncode != 0:
            try:
                vulnerabilities = json.loads(result.stdout)
                if vulnerabilities:
                    logger.error(
                        f"Found {len(vulnerabilities)} vulnerabilities"
                    )
                    return False
            except json.JSONDecodeError:
                logger.warning("Could not parse safety output")

        # Scan Docker image
        artifacts_file = self.artifacts_path / "build_artifacts.json"
        with open(artifacts_file) as f:
            artifacts = json.load(f)

        image_tag = artifacts['docker_image']

        result = subprocess.run(
            ["trivy", "image", "--severity", "HIGH,CRITICAL", image_tag],
            capture_output=True,
            text=True
        )

        if "Total: 0" not in result.stdout:
            logger.warning(f"Docker image vulnerabilities:\n{result.stdout}")
            # Don't fail on vulnerabilities, just warn

        logger.info("Security scan completed")
        return True

    def _deploy_staging(self, git_commit: str) -> bool:
        """Deploy to staging environment."""
        logger.info("Deploying to staging...")

        # Load artifacts
        artifacts_file = self.artifacts_path / "build_artifacts.json"
        with open(artifacts_file) as f:
            artifacts = json.load(f)

        # Create deployment record
        deployment = DeploymentRecord(
            deployment_id=f"staging-{git_commit[:8]}",
            git_commit=git_commit,
            model_version=artifacts.get('model_version', 'unknown'),
            environment="staging",
            status=DeploymentStatus.RUNNING,
            timestamp=datetime.now(),
            artifacts=artifacts
        )

        try:
            # Deploy to staging (implementation depends on infrastructure)
            # For demo, simulate deployment
            self._simulate_deployment("staging", artifacts)

            deployment.status = DeploymentStatus.SUCCESS
            deployment.duration = 30.0

            self.deployments.append(deployment)

            logger.info("Staging deployment successful")
            return True

        except Exception as e:
            logger.error(f"Staging deployment failed: {e}")
            deployment.status = DeploymentStatus.FAILED
            self.deployments.append(deployment)
            return False

    def _validate_deployment(self) -> bool:
        """Validate staging deployment."""
        logger.info("Validating deployment...")

        # Get latest staging deployment
        staging_deployments = [
            d for d in self.deployments
            if d.environment == "staging"
        ]

        if not staging_deployments:
            logger.error("No staging deployment found")
            return False

        latest = staging_deployments[-1]

        # Run validation tests
        # 1. Health check
        # 2. Smoke tests
        # 3. Performance tests

        # For demo, simulate validation
        validation_passed = True

        if validation_passed:
            logger.info("Validation passed")
            return True
        else:
            logger.error("Validation failed")

            if self.config.rollback_on_failure:
                self._rollback_deployment("staging")

            return False

    def _deploy_production(self, git_commit: str) -> bool:
        """Deploy to production."""
        if not self.config.auto_deploy_prod:
            logger.info("Production deployment requires manual approval")
            return True

        logger.info("Deploying to production...")

        # Similar to staging deployment
        artifacts_file = self.artifacts_path / "build_artifacts.json"
        with open(artifacts_file) as f:
            artifacts = json.load(f)

        deployment = DeploymentRecord(
            deployment_id=f"prod-{git_commit[:8]}",
            git_commit=git_commit,
            model_version=artifacts.get('model_version', 'unknown'),
            environment="production",
            status=DeploymentStatus.RUNNING,
            timestamp=datetime.now(),
            artifacts=artifacts
        )

        try:
            self._simulate_deployment("production", artifacts)

            deployment.status = DeploymentStatus.SUCCESS
            deployment.duration = 45.0

            self.deployments.append(deployment)

            logger.info("Production deployment successful")
            return True

        except Exception as e:
            logger.error(f"Production deployment failed: {e}")
            deployment.status = DeploymentStatus.FAILED
            self.deployments.append(deployment)

            # Auto-rollback on production failure
            self._rollback_deployment("production")

            return False

    def _simulate_deployment(self, environment: str, artifacts: Dict):
        """Simulate deployment (replace with actual deployment logic)."""
        logger.info(f"Deploying to {environment}: {artifacts}")
        # In production:
        # - Update Kubernetes deployment
        # - Update service mesh routing
        # - Update feature flags
        # - Drain old pods
        # - Monitor new pods
        pass

    def _rollback_deployment(self, environment: str):
        """Rollback to previous deployment."""
        logger.info(f"Rolling back {environment} deployment")

        # Get previous successful deployment
        env_deployments = [
            d for d in self.deployments
            if d.environment == environment and d.status == DeploymentStatus.SUCCESS
        ]

        if len(env_deployments) < 2:
            logger.error("No previous deployment to rollback to")
            return

        previous = env_deployments[-2]

        logger.info(
            f"Rolling back to deployment {previous.deployment_id}"
        )

        # Create rollback deployment record
        rollback = DeploymentRecord(
            deployment_id=f"rollback-{previous.deployment_id}",
            git_commit=previous.git_commit,
            model_version=previous.model_version,
            environment=environment,
            status=DeploymentStatus.RUNNING,
            timestamp=datetime.now(),
            artifacts=previous.artifacts,
            rollback_to=previous.deployment_id
        )

        try:
            self._simulate_deployment(environment, previous.artifacts)

            rollback.status = DeploymentStatus.ROLLED_BACK
            self.deployments.append(rollback)

            logger.info("Rollback successful")

        except Exception as e:
            logger.error(f"Rollback failed: {e}")
            rollback.status = DeploymentStatus.FAILED
            self.deployments.append(rollback)

    def _notify_success(self, git_commit: str):
        """Send success notification."""
        if not self.config.slack_webhook:
            return

        message = {
            "text": f"[SUCCESS] CI/CD Pipeline Success",
            "blocks": [
                {
                    "type": "section",
                    "text": {
                        "type": "mrkdwn",
                        "text": (
                            f"*Pipeline*: {self.config.name}\n"
                            f"*Commit*: `{git_commit[:8]}`\n"
                            f"*Status*: Success"
                        )
                    }
                }
            ]
        }

        # Send to Slack
        self._send_slack(message)

    def _notify_failure(
        self,
        stage: Optional[PipelineStage],
        git_commit: str,
        error: Optional[str] = None
    ):
        """Send failure notification."""
        if not self.config.slack_webhook:
            return

        stage_name = stage.value if stage else "Unknown"

        message = {
            "text": f"[FAILED] CI/CD Pipeline Failed",
            "blocks": [
                {
                    "type": "section",
                    "text": {
                        "type": "mrkdwn",
                        "text": (
                            f"*Pipeline*: {self.config.name}\n"
                            f"*Commit*: `{git_commit[:8]}`\n"
                            f"*Failed Stage*: {stage_name}\n"
                            f"*Error*: {error or 'See logs'}"
                        )
                    }
                }
            ]
        }

        self._send_slack(message)

    def _send_slack(self, message: Dict):
        """Send Slack notification."""
        import requests

        try:
            response = requests.post(
                self.config.slack_webhook,
                json=message
            )
            response.raise_for_status()
        except Exception as e:
            logger.error(f"Failed to send Slack notification: {e}")

    def get_deployment_history(
        self,
        environment: Optional[str] = None
    ) -> List[DeploymentRecord]:
        """
        Get deployment history.

        Args:
            environment: Filter by environment

        Returns:
            List of deployments
        """
        if environment:
            return [
                d for d in self.deployments
                if d.environment == environment
            ]

        return self.deployments
\end{lstlisting}

\subsection{GitHub Actions Integration}

\begin{lstlisting}[caption={.github/workflows/ml-cicd.yml}]
name: ML CI/CD Pipeline

on:
  push:
    branches: [main, develop]
  pull_request:
    branches: [main]

env:
  PYTHON_VERSION: '3.9'
  MODEL_REGISTRY: 'your-registry.azurecr.io'

jobs:
  lint-and-test:
    runs-on: ubuntu-latest

    steps:
      - uses: actions/checkout@v3

      - name: Set up Python
        uses: actions/setup-python@v4
        with:
          python-version: ${{ env.PYTHON_VERSION }}

      - name: Cache dependencies
        uses: actions/cache@v3
        with:
          path: ~/.cache/pip
          key: ${{ runner.os }}-pip-${{ hashFiles('requirements.txt') }}

      - name: Install dependencies
        run: |
          python -m pip install --upgrade pip
          pip install -r requirements.txt
          pip install -r requirements-dev.txt

      - name: Lint with flake8
        run: |
          flake8 src/ --count --select=E9,F63,F7,F82 --show-source --statistics
          flake8 src/ --count --max-line-length=100 --statistics

      - name: Check formatting with black
        run: black --check src/

      - name: Type check with mypy
        run: mypy src/ --ignore-missing-imports
        continue-on-error: true

      - name: Run tests
        run: |
          pytest tests/ -v --cov=src --cov-report=xml

      - name: Upload coverage
        uses: codecov/codecov-action@v3
        with:
          file: ./coverage.xml

  security-scan:
    runs-on: ubuntu-latest
    needs: lint-and-test

    steps:
      - uses: actions/checkout@v3

      - name: Run safety check
        run: |
          pip install safety
          safety check --json

      - name: Run bandit security linter
        run: |
          pip install bandit
          bandit -r src/ -f json

  build-and-push:
    runs-on: ubuntu-latest
    needs: [lint-and-test, security-scan]
    if: github.ref == 'refs/heads/main'

    steps:
      - uses: actions/checkout@v3

      - name: Set up Docker Buildx
        uses: docker/setup-buildx-action@v2

      - name: Log in to registry
        uses: docker/login-action@v2
        with:
          registry: ${{ env.MODEL_REGISTRY }}
          username: ${{ secrets.REGISTRY_USERNAME }}
          password: ${{ secrets.REGISTRY_PASSWORD }}

      - name: Build and push Docker image
        uses: docker/build-push-action@v4
        with:
          context: .
          push: true
          tags: |
            ${{ env.MODEL_REGISTRY }}/ml-model:${{ github.sha }}
            ${{ env.MODEL_REGISTRY }}/ml-model:latest
          cache-from: type=registry,ref=${{ env.MODEL_REGISTRY }}/ml-model:latest
          cache-to: type=inline

  deploy-staging:
    runs-on: ubuntu-latest
    needs: build-and-push
    if: github.ref == 'refs/heads/main'

    steps:
      - uses: actions/checkout@v3

      - name: Deploy to staging
        run: |
          # Update Kubernetes deployment
          kubectl set image deployment/ml-model \
            ml-model=${{ env.MODEL_REGISTRY }}/ml-model:${{ github.sha }} \
            -n staging

      - name: Wait for rollout
        run: |
          kubectl rollout status deployment/ml-model -n staging

      - name: Run smoke tests
        run: |
          python tests/smoke_tests.py --env staging

  deploy-production:
    runs-on: ubuntu-latest
    needs: deploy-staging
    if: github.ref == 'refs/heads/main'
    environment:
      name: production
      url: https://api.production.com

    steps:
      - uses: actions/checkout@v3

      - name: Deploy to production
        run: |
          kubectl set image deployment/ml-model \
            ml-model=${{ env.MODEL_REGISTRY }}/ml-model:${{ github.sha }} \
            -n production

      - name: Wait for rollout
        run: |
          kubectl rollout status deployment/ml-model -n production

      - name: Verify deployment
        run: |
          python tests/smoke_tests.py --env production

      - name: Notify Slack
        if: always()
        uses: 8398a7/action-slack@v3
        with:
          status: ${{ job.status }}
          text: 'Production deployment ${{ job.status }}'
          webhook_url: ${{ secrets.SLACK_WEBHOOK }}
\end{lstlisting}

\section{Model Training Automation}

Automated retraining ensures models stay current with changing data patterns.

\subsection{MLPipeline: Automated Training System}

\begin{lstlisting}[language=Python, caption={Automated ML Training Pipeline}]
from typing import Dict, List, Optional, Any, Callable
from dataclasses import dataclass, field
from datetime import datetime, timedelta
from pathlib import Path
import logging
import joblib
import json

logger = logging.getLogger(__name__)

class TriggerCondition(Enum):
    """Training trigger conditions."""
    SCHEDULED = "scheduled"
    PERFORMANCE_DEGRADATION = "performance_degradation"
    DATA_DRIFT = "data_drift"
    MANUAL = "manual"
    DATA_THRESHOLD = "data_threshold"

@dataclass
class TrainingConfig:
    """
    Training configuration.

    Attributes:
        model_name: Model identifier
        training_schedule: Cron expression for scheduled training
        performance_threshold: Min performance before retraining
        drift_threshold: Max drift before retraining
        min_training_samples: Minimum samples required
        validation_split: Validation set proportion
        hyperparameters: Model hyperparameters
    """
    model_name: str
    training_schedule: Optional[str] = "0 2 * * *"  # 2 AM daily
    performance_threshold: float = 0.85
    drift_threshold: float = 0.15
    min_training_samples: int = 10000
    validation_split: float = 0.2
    hyperparameters: Dict[str, Any] = field(default_factory=dict)

@dataclass
class TrainingRun:
    """
    Record of a training run.

    Attributes:
        run_id: Unique run identifier
        trigger: What triggered this run
        start_time: When training started
        end_time: When training completed
        status: Training status
        metrics: Evaluation metrics
        model_path: Path to trained model
        artifacts: Additional artifacts
    """
    run_id: str
    trigger: TriggerCondition
    start_time: datetime
    end_time: Optional[datetime] = None
    status: str = "running"
    metrics: Dict[str, float] = field(default_factory=dict)
    model_path: Optional[str] = None
    artifacts: Dict[str, str] = field(default_factory=dict)

class MLPipeline:
    """
    Automated ML training pipeline with triggers and validation.

    Handles data loading, training, evaluation, and model registration.

    Example:
        >>> pipeline = MLPipeline(config)
        >>> pipeline.check_triggers()
        >>> if pipeline.should_train():
        ...     pipeline.train()
    """

    def __init__(
        self,
        config: TrainingConfig,
        data_loader: Callable,
        model_factory: Callable,
        output_path: str = "./models"
    ):
        """
        Initialize ML pipeline.

        Args:
            config: Training configuration
            data_loader: Function to load training data
            model_factory: Function to create model instance
            output_path: Where to save trained models
        """
        self.config = config
        self.data_loader = data_loader
        self.model_factory = model_factory
        self.output_path = Path(output_path)

        # Create output directory
        self.output_path.mkdir(parents=True, exist_ok=True)

        # Training history
        self.training_runs: List[TrainingRun] = []

        # Current production model
        self.current_model = None
        self.current_metrics: Dict[str, float] = {}

        logger.info(f"Initialized ML pipeline: {config.model_name}")

    def check_triggers(self) -> List[TriggerCondition]:
        """
        Check if any training triggers are active.

        Returns:
            List of active triggers
        """
        active_triggers = []

        # Check scheduled trigger
        if self._should_train_scheduled():
            active_triggers.append(TriggerCondition.SCHEDULED)

        # Check performance degradation
        if self._has_performance_degraded():
            active_triggers.append(TriggerCondition.PERFORMANCE_DEGRADATION)

        # Check data drift
        if self._has_data_drifted():
            active_triggers.append(TriggerCondition.DATA_DRIFT)

        # Check data volume
        if self._has_sufficient_new_data():
            active_triggers.append(TriggerCondition.DATA_THRESHOLD)

        return active_triggers

    def should_train(self) -> bool:
        """
        Determine if training should be triggered.

        Returns:
            True if any trigger is active
        """
        triggers = self.check_triggers()

        if triggers:
            logger.info(f"Training triggers active: {triggers}")
            return True

        return False

    def train(
        self,
        trigger: TriggerCondition = TriggerCondition.MANUAL
    ) -> TrainingRun:
        """
        Execute training pipeline.

        Args:
            trigger: What triggered training

        Returns:
            Training run record
        """
        run_id = f"{self.config.model_name}_{datetime.now().strftime('%Y%m%d_%H%M%S')}"

        run = TrainingRun(
            run_id=run_id,
            trigger=trigger,
            start_time=datetime.now()
        )

        logger.info(f"Starting training run: {run_id}")

        try:
            # Load data
            logger.info("Loading training data")
            X_train, X_val, y_train, y_val = self._load_data()

            # Check minimum samples
            if len(X_train) < self.config.min_training_samples:
                raise ValueError(
                    f"Insufficient training samples: {len(X_train)} < "
                    f"{self.config.min_training_samples}"
                )

            # Create model
            logger.info("Creating model")
            model = self.model_factory(self.config.hyperparameters)

            # Train model
            logger.info("Training model")
            model.fit(X_train, y_train)

            # Evaluate model
            logger.info("Evaluating model")
            metrics = self._evaluate_model(model, X_val, y_val)

            run.metrics = metrics

            # Validate performance
            if not self._validate_performance(metrics):
                run.status = "failed_validation"
                logger.error("Model failed validation")
                return run

            # Save model
            model_path = self._save_model(model, run_id)
            run.model_path = str(model_path)

            # Save artifacts
            artifacts_path = self._save_artifacts(run, metrics)
            run.artifacts = {"metadata": str(artifacts_path)}

            run.status = "success"
            run.end_time = datetime.now()

            self.training_runs.append(run)

            logger.info(
                f"Training completed successfully. Metrics: {metrics}"
            )

            return run

        except Exception as e:
            logger.error(f"Training failed: {e}")
            run.status = "failed"
            run.end_time = datetime.now()
            self.training_runs.append(run)
            raise

    def _load_data(self):
        """Load and split training data."""
        # Load data using provided function
        data = self.data_loader()

        from sklearn.model_selection import train_test_split

        # Split features and target
        X = data.drop('target', axis=1)
        y = data['target']

        # Train/validation split
        X_train, X_val, y_train, y_val = train_test_split(
            X, y,
            test_size=self.config.validation_split,
            random_state=42,
            stratify=y
        )

        return X_train, X_val, y_train, y_val

    def _evaluate_model(self, model, X_val, y_val) -> Dict[str, float]:
        """Evaluate model performance."""
        from sklearn.metrics import (
            accuracy_score, precision_score, recall_score,
            f1_score, roc_auc_score
        )

        # Predictions
        y_pred = model.predict(X_val)
        y_prob = model.predict_proba(X_val)[:, 1]

        # Compute metrics
        metrics = {
            'accuracy': accuracy_score(y_val, y_pred),
            'precision': precision_score(y_val, y_pred),
            'recall': recall_score(y_val, y_pred),
            'f1': f1_score(y_val, y_pred),
            'auc': roc_auc_score(y_val, y_prob)
        }

        return metrics

    def _validate_performance(self, metrics: Dict[str, float]) -> bool:
        """Validate model meets minimum requirements."""
        # Check primary metric (accuracy)
        if metrics['accuracy'] < self.config.performance_threshold:
            logger.warning(
                f"Model accuracy {metrics['accuracy']:.3f} below "
                f"threshold {self.config.performance_threshold}"
            )
            return False

        # Check if better than current model
        if self.current_metrics:
            current_accuracy = self.current_metrics.get('accuracy', 0)

            if metrics['accuracy'] <= current_accuracy:
                logger.warning(
                    f"New model accuracy {metrics['accuracy']:.3f} not "
                    f"better than current {current_accuracy:.3f}"
                )
                # Still valid, just not an improvement
                # In production, might want to require improvement

        return True

    def _save_model(self, model, run_id: str) -> Path:
        """Save trained model."""
        model_path = self.output_path / f"{run_id}.pkl"
        joblib.dump(model, model_path)

        logger.info(f"Model saved to {model_path}")

        return model_path

    def _save_artifacts(
        self,
        run: TrainingRun,
        metrics: Dict[str, float]
    ) -> Path:
        """Save training artifacts."""
        artifacts = {
            'run_id': run.run_id,
            'trigger': run.trigger.value,
            'start_time': run.start_time.isoformat(),
            'metrics': metrics,
            'config': {
                'model_name': self.config.model_name,
                'hyperparameters': self.config.hyperparameters
            }
        }

        artifacts_path = self.output_path / f"{run.run_id}_metadata.json"

        with open(artifacts_path, 'w') as f:
            json.dump(artifacts, f, indent=2)

        return artifacts_path

    def _should_train_scheduled(self) -> bool:
        """Check if scheduled training is due."""
        if not self.config.training_schedule:
            return False

        # Check last training time
        if not self.training_runs:
            return True

        last_run = self.training_runs[-1]
        hours_since = (datetime.now() - last_run.start_time).total_seconds() / 3600

        # If using daily schedule and > 24 hours, retrain
        return hours_since >= 24

    def _has_performance_degraded(self) -> bool:
        """Check if model performance has degraded."""
        if not self.current_metrics:
            return False

        # In production, check recent performance metrics
        # For demo, simulate check
        recent_accuracy = 0.82  # Would come from monitoring

        return recent_accuracy < self.config.performance_threshold

    def _has_data_drifted(self) -> bool:
        """Check if data drift exceeds threshold."""
        # In production, check drift metrics from monitoring
        # For demo, simulate check
        drift_score = 0.10  # Would come from drift detector

        return drift_score > self.config.drift_threshold

    def _has_sufficient_new_data(self) -> bool:
        """Check if enough new data is available."""
        # In production, check data warehouse for new records
        # For demo, simulate check
        new_samples = 15000  # Would query data source

        return new_samples >= self.config.min_training_samples

    def promote_to_production(self, run_id: str):
        """
        Promote a trained model to production.

        Args:
            run_id: Training run to promote
        """
        # Find run
        run = next((r for r in self.training_runs if r.run_id == run_id), None)

        if not run:
            raise ValueError(f"Run {run_id} not found")

        if run.status != "success":
            raise ValueError(f"Run {run_id} did not succeed")

        # Load model
        model = joblib.load(run.model_path)

        # Update current model
        self.current_model = model
        self.current_metrics = run.metrics

        # Copy to production location
        prod_path = self.output_path / "production" / f"{self.config.model_name}.pkl"
        prod_path.parent.mkdir(parents=True, exist_ok=True)

        joblib.dump(model, prod_path)

        logger.info(f"Promoted model {run_id} to production")
\end{lstlisting}

\section{Infrastructure as Code}

IaC ensures consistent, version-controlled infrastructure across environments.

\subsection{Terraform Configuration for ML Infrastructure}

\begin{lstlisting}[language=Python, caption={Terraform Configuration Generator}]
from typing import Dict, List, Optional
from pathlib import Path
import logging

logger = logging.getLogger(__name__)

class InfrastructureManager:
    """
    Generate and manage infrastructure as code.

    Creates Terraform configurations for ML infrastructure.

    Example:
        >>> infra = InfrastructureManager("ml-platform")
        >>> infra.create_training_cluster(instance_type="n1-standard-8")
        >>> infra.create_serving_cluster(min_replicas=2)
        >>> infra.generate_terraform()
    """

    def __init__(self, project_name: str, output_path: str = "./terraform"):
        """
        Initialize infrastructure manager.

        Args:
            project_name: Project identifier
            output_path: Where to write Terraform files
        """
        self.project_name = project_name
        self.output_path = Path(output_path)

        # Infrastructure components
        self.resources: List[Dict] = []

        logger.info(f"Initialized infrastructure manager: {project_name}")

    def create_training_cluster(
        self,
        instance_type: str = "n1-standard-8",
        min_nodes: int = 1,
        max_nodes: int = 10
    ):
        """
        Add training cluster configuration.

        Args:
            instance_type: VM instance type
            min_nodes: Minimum cluster nodes
            max_nodes: Maximum cluster nodes
        """
        resource = {
            'type': 'google_container_cluster',
            'name': f'{self.project_name}-training',
            'config': {
                'name': f'{self.project_name}-training-cluster',
                'initial_node_count': min_nodes,
                'node_config': {
                    'machine_type': instance_type,
                    'disk_size_gb': 100,
                    'oauth_scopes': [
                        'https://www.googleapis.com/auth/cloud-platform'
                    ]
                },
                'autoscaling': {
                    'min_node_count': min_nodes,
                    'max_node_count': max_nodes
                }
            }
        }

        self.resources.append(resource)

    def create_serving_cluster(
        self,
        instance_type: str = "n1-standard-4",
        min_replicas: int = 2,
        max_replicas: int = 20
    ):
        """Add serving cluster configuration."""
        resource = {
            'type': 'google_container_cluster',
            'name': f'{self.project_name}-serving',
            'config': {
                'name': f'{self.project_name}-serving-cluster',
                'initial_node_count': min_replicas,
                'node_config': {
                    'machine_type': instance_type,
                    'disk_size_gb': 50
                },
                'autoscaling': {
                    'min_node_count': min_replicas,
                    'max_node_count': max_replicas
                }
            }
        }

        self.resources.append(resource)

    def create_feature_store(
        self,
        instance_type: str = "db-n1-standard-2"
    ):
        """Add feature store (database) configuration."""
        resource = {
            'type': 'google_sql_database_instance',
            'name': f'{self.project_name}-feature-store',
            'config': {
                'name': f'{self.project_name}-features',
                'database_version': 'POSTGRES_13',
                'tier': instance_type,
                'settings': {
                    'backup_configuration': {
                        'enabled': True,
                        'point_in_time_recovery_enabled': True
                    }
                }
            }
        }

        self.resources.append(resource)

    def generate_terraform(self):
        """Generate Terraform configuration files."""
        self.output_path.mkdir(parents=True, exist_ok=True)

        # Main configuration
        main_tf = self._generate_main_config()
        with open(self.output_path / "main.tf", 'w') as f:
            f.write(main_tf)

        # Variables
        variables_tf = self._generate_variables()
        with open(self.output_path / "variables.tf", 'w') as f:
            f.write(variables_tf)

        # Outputs
        outputs_tf = self._generate_outputs()
        with open(self.output_path / "outputs.tf", 'w') as f:
            f.write(outputs_tf)

        logger.info(f"Generated Terraform config in {self.output_path}")

    def _generate_main_config(self) -> str:
        """Generate main Terraform configuration."""
        lines = [
            'terraform {',
            '  required_version = ">= 1.0"',
            '  required_providers {',
            '    google = {',
            '      source  = "hashicorp/google"',
            '      version = "~> 4.0"',
            '    }',
            '  }',
            '}',
            '',
            'provider "google" {',
            '  project = var.project_id',
            '  region  = var.region',
            '}',
            ''
        ]

        # Add resources
        for resource in self.resources:
            lines.append(
                f'resource "{resource["type"]}" "{resource["name"]}" {{'
            )

            config = resource['config']
            for key, value in config.items():
                if isinstance(value, dict):
                    lines.append(f'  {key} {{')
                    for k2, v2 in value.items():
                        lines.append(f'    {k2} = {self._format_value(v2)}')
                    lines.append('  }')
                else:
                    lines.append(f'  {key} = {self._format_value(value)}')

            lines.append('}')
            lines.append('')

        return '\n'.join(lines)

    def _generate_variables(self) -> str:
        """Generate variables configuration."""
        return '''
variable "project_id" {
  description = "GCP project ID"
  type        = string
}

variable "region" {
  description = "GCP region"
  type        = string
  default     = "us-central1"
}

variable "environment" {
  description = "Environment (dev, staging, prod)"
  type        = string
}
'''

    def _generate_outputs(self) -> str:
        """Generate outputs configuration."""
        lines = []

        for resource in self.resources:
            name = resource['name']
            lines.append(f'output "{name}_id" {{')
            lines.append(f'  value = {resource["type"]}.{name}.id')
            lines.append('}')
            lines.append('')

        return '\n'.join(lines)

    def _format_value(self, value) -> str:
        """Format value for Terraform syntax."""
        if isinstance(value, str):
            return f'"{value}"'
        elif isinstance(value, list):
            items = [self._format_value(v) for v in value]
            return f'[{", ".join(items)}]'
        else:
            return str(value)
\end{lstlisting}

\section{Configuration Management}

Centralized configuration enables environment-specific settings without code changes.

\subsection{ConfigurationManager}

\begin{lstlisting}[language=Python, caption={Environment Configuration Management}]
from typing import Dict, Any, Optional
from pathlib import Path
from enum import Enum
import yaml
import os
import logging

logger = logging.getLogger(__name__)

class Environment(Enum):
    """Deployment environments."""
    DEVELOPMENT = "development"
    STAGING = "staging"
    PRODUCTION = "production"

class ConfigurationManager:
    """
    Manage environment-specific configurations.

    Loads configs from YAML files and environment variables.

    Example:
        >>> config_mgr = ConfigurationManager()
        >>> config = config_mgr.get_config(Environment.PRODUCTION)
        >>> model_path = config['model']['path']
    """

    def __init__(self, config_dir: str = "./config"):
        """
        Initialize configuration manager.

        Args:
            config_dir: Directory containing config files
        """
        self.config_dir = Path(config_dir)
        self.configs: Dict[Environment, Dict] = {}

        # Load all configs
        self._load_configs()

        logger.info("Initialized configuration manager")

    def _load_configs(self):
        """Load configuration files for all environments."""
        for env in Environment:
            config_file = self.config_dir / f"{env.value}.yaml"

            if config_file.exists():
                with open(config_file) as f:
                    config = yaml.safe_load(f)

                self.configs[env] = config
                logger.info(f"Loaded config for {env.value}")
            else:
                logger.warning(f"Config file not found: {config_file}")

    def get_config(
        self,
        environment: Optional[Environment] = None
    ) -> Dict[str, Any]:
        """
        Get configuration for environment.

        Args:
            environment: Target environment (auto-detect if None)

        Returns:
            Configuration dictionary
        """
        if environment is None:
            environment = self._detect_environment()

        config = self.configs.get(environment, {})

        # Overlay environment variables
        config = self._apply_env_overrides(config)

        return config

    def _detect_environment(self) -> Environment:
        """Auto-detect current environment."""
        env_var = os.getenv('ENVIRONMENT', 'development')

        try:
            return Environment(env_var.lower())
        except ValueError:
            logger.warning(
                f"Unknown environment {env_var}, defaulting to development"
            )
            return Environment.DEVELOPMENT

    def _apply_env_overrides(self, config: Dict) -> Dict:
        """Apply environment variable overrides."""
        # Check for environment-specific overrides
        # Format: APP_MODEL_PATH=/path/to/model

        import copy
        config = copy.deepcopy(config)

        prefix = "APP_"

        for key, value in os.environ.items():
            if not key.startswith(prefix):
                continue

            # Convert APP_MODEL_PATH to ['model', 'path']
            parts = key[len(prefix):].lower().split('_')

            # Set nested value
            current = config
            for part in parts[:-1]:
                if part not in current:
                    current[part] = {}
                current = current[part]

            current[parts[-1]] = value

        return config
\end{lstlisting}

\subsection{Example Configuration Files}

\begin{lstlisting}[caption={config/production.yaml}]
# Production configuration

model:
  name: "fraud-detector"
  version: "v2.1"
  path: "gs://models-prod/fraud-detector/v2.1"

serving:
  replicas: 5
  instance_type: "n1-standard-4"
  max_latency_ms: 100
  timeout_seconds: 30

database:
  host: "prod-db.example.com"
  port: 5432
  name: "ml_features"
  connection_pool_size: 20

feature_store:
  type: "feast"
  url: "feast-prod.example.com:443"

monitoring:
  enabled: true
  prometheus_endpoint: "http://prometheus-prod:9090"
  alert_webhook: "https://hooks.slack.com/services/XXX"

security:
  tls_enabled: true
  mtls_enabled: true
  api_key_required: true
\end{lstlisting}

\section{Real-World Scenario: Automation Preventing Disaster}

\subsection{The Problem}

A fintech company manually deployed ML models for loan approval. Their process:

\begin{enumerate}
    \item Data scientist trains model locally
    \item Emails model file (.pkl) to ops team
    \item Ops copies file to production server via SCP
    \item Ops manually restarts service
    \item No testing in staging
    \item No validation of model performance
    \item No rollback plan
\end{enumerate}

On a Friday deployment:
\begin{itemize}
    \item New model accidentally trained on 3-month-old data (stale features)
    \item Model approved 92\% of loans (baseline: 78\%)
    \item Weekend processing approved \$45M in loans, 40\% high-risk
    \item Monday morning: fraud alerts spike
    \item Tuesday: Model rolled back after 4-day impact
\end{itemize}

\textbf{Cost}: \$18M in bad loans, regulatory investigation, 2-month development freeze.

\subsection{The Solution}

Implementing full MLOps automation:

\begin{lstlisting}[language=Python, caption={Complete MLOps Automation}]
# 1. CI/CD Pipeline Configuration
pipeline_config = PipelineConfig(
    name="loan-approval-ml",
    trigger_branch="main",
    stages=[
        PipelineStage.LINT,
        PipelineStage.TEST,
        PipelineStage.BUILD,
        PipelineStage.SECURITY_SCAN,
        PipelineStage.DEPLOY_STAGING,
        PipelineStage.VALIDATE,
        PipelineStage.DEPLOY_PROD
    ],
    auto_deploy_staging=True,
    auto_deploy_prod=False,  # Requires approval
    rollback_on_failure=True
)

cicd = CICDManager(pipeline_config, repo_path=".")

# 2. Automated Training Pipeline
training_config = TrainingConfig(
    model_name="loan-approval",
    training_schedule="0 2 * * 0",  # Weekly Sunday 2 AM
    performance_threshold=0.82,
    drift_threshold=0.10,
    min_training_samples=50000,
    validation_split=0.2,
    hyperparameters={
        'max_depth': 8,
        'n_estimators': 200,
        'min_samples_split': 100
    }
)

ml_pipeline = MLPipeline(
    config=training_config,
    data_loader=load_loan_data,
    model_factory=create_loan_model
)

# 3. Automated Validation
class LoanModelValidator:
    """Validate loan approval models."""

    def validate(self, model, test_data) -> bool:
        """Run comprehensive validation."""
        X_test, y_test = test_data

        # Predictions
        y_pred = model.predict(X_test)
        y_prob = model.predict_proba(X_test)[:, 1]

        # Compute metrics
        from sklearn.metrics import roc_auc_score, precision_score

        auc = roc_auc_score(y_test, y_prob)
        precision = precision_score(y_test, y_pred)

        # Validation checks
        checks = []

        # 1. Minimum performance
        checks.append({
            'name': 'minimum_auc',
            'passed': auc >= 0.82,
            'value': auc,
            'threshold': 0.82
        })

        # 2. Precision (avoid approving bad loans)
        checks.append({
            'name': 'minimum_precision',
            'passed': precision >= 0.80,
            'value': precision,
            'threshold': 0.80
        })

        # 3. Approval rate check (catch data issues)
        approval_rate = y_pred.mean()
        checks.append({
            'name': 'approval_rate',
            'passed': 0.70 <= approval_rate <= 0.85,
            'value': approval_rate,
            'range': [0.70, 0.85]
        })

        # 4. Data freshness
        from datetime import datetime, timedelta
        max_age = datetime.now() - timedelta(days=7)

        data_timestamp = test_data.attrs.get('timestamp', datetime.now())
        checks.append({
            'name': 'data_freshness',
            'passed': data_timestamp >= max_age,
            'value': data_timestamp.isoformat(),
            'threshold': max_age.isoformat()
        })

        # Log results
        for check in checks:
            status = "PASS" if check['passed'] else "FAIL"
            logger.info(f"[{status}] {check['name']}: {check}")

        # Overall pass
        all_passed = all(c['passed'] for c in checks)

        if not all_passed:
            logger.error("Model validation failed")
            failed = [c['name'] for c in checks if not c['passed']]
            logger.error(f"Failed checks: {failed}")

        return all_passed

# 4. Automated Deployment Workflow
def automated_deployment_workflow():
    """Complete automated deployment workflow."""

    # Check training triggers
    triggers = ml_pipeline.check_triggers()

    if triggers:
        logger.info(f"Training triggered by: {triggers}")

        # Train model
        run = ml_pipeline.train(trigger=triggers[0])

        if run.status != "success":
            logger.error("Training failed, aborting deployment")
            return

        # Validate model
        validator = LoanModelValidator()
        model = joblib.load(run.model_path)

        test_data = load_test_data()
        if not validator.validate(model, test_data):
            logger.error("Validation failed, model not promoted")
            return

        # Promote to staging
        ml_pipeline.promote_to_production(run.run_id)

        # Trigger CI/CD for deployment
        cicd.run_pipeline()

        logger.info("Automated deployment completed")

# 5. Monitoring and Auto-Rollback
from monitoring import ModelMonitor, AlertSeverity

monitor = ModelMonitor("loan-approval-prod")

# Register key metrics
monitor.register_metric(MetricConfig(
    name="approval_rate",
    metric_type=MetricType.GAUGE,
    description="Rate of loan approvals",
    thresholds={
        AlertSeverity.WARNING: 0.85,  # Above 85% is suspicious
        AlertSeverity.CRITICAL: 0.90
    }
))

monitor.register_metric(MetricConfig(
    name="avg_confidence",
    metric_type=MetricType.GAUGE,
    description="Average prediction confidence",
    thresholds={
        AlertSeverity.WARNING: 0.60,  # Below 60% confidence
        AlertSeverity.CRITICAL: 0.50
    }
))

# Auto-rollback on critical alerts
def alert_handler(alert):
    """Handle monitoring alerts."""
    if alert.severity == AlertSeverity.CRITICAL:
        logger.critical(f"Critical alert: {alert.message}")

        # Trigger automatic rollback
        cicd._rollback_deployment("production")

        # Notify team
        notify_team(alert)

monitor.alert_callback = alert_handler

# 6. Scheduled Execution
import schedule

schedule.every().sunday.at("02:00").do(automated_deployment_workflow)
schedule.every(10).minutes.do(lambda: monitor.check_alerts())

# Run scheduler
while True:
    schedule.run_pending()
    time.sleep(60)
\end{lstlisting}

\subsection{Outcome}

With MLOps automation:
\begin{itemize}
    \item \textbf{Week 1}: Stale data model caught by freshness check in CI/CD
    \item \textbf{Week 2}: Model with 91\% approval rate failed validation
    \item \textbf{Week 3}: Deployed model triggered alert for 86\% approvals, auto-rollback in 2 minutes
    \item \textbf{6 Months}: Zero production incidents, 24 successful deployments
    \item \textbf{Impact}: Prevented \$18M+ in potential losses, reduced deployment time from 6 hours to 45 minutes
\end{itemize}

\section{Exercises}

\subsection{Exercise 1: Build CI/CD Pipeline}

Implement complete CI/CD pipeline:
\begin{itemize}
    \item Lint, test, build, security scan stages
    \item Automated deployment to staging
    \item Smoke tests and validation
    \item Manual approval gate for production
    \item Slack notifications on success/failure
\end{itemize}

\subsection{Exercise 2: Training Automation}

Create automated training system:
\begin{itemize}
    \item Scheduled weekly retraining
    \item Performance degradation triggers
    \item Data drift detection triggers
    \item Minimum data threshold checks
    \item Automated hyperparameter tuning
\end{itemize}

\subsection{Exercise 3: Infrastructure as Code}

Generate Terraform configuration for:
\begin{itemize}
    \item Kubernetes cluster for training (autoscaling 1-10 nodes)
    \item Kubernetes cluster for serving (autoscaling 2-20 nodes)
    \item PostgreSQL feature store with backups
    \item Object storage for models
    \item Monitoring stack (Prometheus + Grafana)
\end{itemize}

\subsection{Exercise 4: Model Validation Framework}

Build comprehensive validation:
\begin{itemize}
    \item Performance metrics (accuracy, precision, recall, AUC)
    \item Fairness checks across demographics
    \item Prediction distribution validation
    \item Data quality checks
    \item Business rule validation
\end{itemize}

\subsection{Exercise 5: Configuration Management}

Implement config system:
\begin{itemize}
    \item YAML configs for dev, staging, prod
    \item Environment variable overrides
    \item Secret management integration (Vault/KMS)
    \item Config validation on startup
    \item Hot-reload capability
\end{itemize}

\subsection{Exercise 6: Rollback Automation}

Create automated rollback:
\begin{itemize}
    \item Detect performance degradation in production
    \item Automatically revert to previous version
    \item Health check before completing rollback
    \item Notify team with rollback details
    \item Prevent re-deployment of bad version
\end{itemize}

\subsection{Exercise 7: GitOps Workflow}

Implement GitOps:
\begin{itemize}
    \item Git as single source of truth
    \item Pull-based deployment (ArgoCD/Flux)
    \item Automatic sync on Git changes
    \item Drift detection and correction
    \item Audit trail of all deployments
\end{itemize}

\section{Key Takeaways}

\begin{itemize}
    \item \textbf{Automate Everything}: Manual steps introduce errors and delays—automate testing, validation, deployment
    \item \textbf{Fail Fast}: Catch issues in CI/CD before production through comprehensive testing
    \item \textbf{Version Everything}: Code, data, models, configurations must be versioned together
    \item \textbf{Validate Rigorously}: Automated validation prevents bad models from reaching production
    \item \textbf{Infrastructure as Code}: Version-controlled infrastructure ensures consistency
    \item \textbf{Enable Rollback}: Every deployment must have instant rollback capability
    \item \textbf{Monitor Continuously}: Detect issues immediately and trigger automatic responses
\end{itemize}

MLOps automation transforms ML from a research project into a reliable production system. Investing in automation infrastructure pays dividends through faster iteration, fewer incidents, and confident deployments.
