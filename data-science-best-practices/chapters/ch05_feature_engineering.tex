\chapter{Systematic Feature Engineering and Selection}

\section{Introduction}

Feature engineering is often the difference between a mediocre model and a breakthrough solution. While modern machine learning algorithms can learn complex patterns, the quality and relevance of input features fundamentally determines model performance. This chapter presents a systematic approach to feature engineering that transforms raw data into predictive signals through domain knowledge, statistical rigor, and production-ready engineering practices.

\subsection{The Feature Engineering Challenge}

Raw data rarely arrives in an optimal format for machine learning. Consider a timestamp: as a Unix epoch, it offers little direct predictive value. However, extracted features like hour-of-day, day-of-week, or days-since-last-event can reveal crucial patterns. The challenge lies in systematically discovering, creating, validating, and maintaining such transformations at scale.

\subsection{Why Feature Engineering Matters}

Studies show that feature engineering can improve model performance by 20-50\% or more, often exceeding gains from hyperparameter tuning or algorithm selection. Yet many teams approach it ad-hoc, creating features without validation, monitoring, or versioning. This leads to:

\begin{itemize}
    \item \textbf{Inconsistent transformations} between training and production
    \item \textbf{Data leakage} through improper temporal ordering
    \item \textbf{Feature drift} going undetected in production
    \item \textbf{Irreproducible results} from undocumented transformations
\end{itemize}

\subsection{Chapter Overview}

This chapter provides a complete framework for systematic feature engineering:

\begin{enumerate}
    \item \textbf{Feature Engineering Pipeline}: Type-safe transformation framework with validation
    \item \textbf{Domain-Driven Feature Creation}: Temporal, categorical, and numerical transformations
    \item \textbf{Feature Selection}: Statistical tests, importance measures, and recursive elimination
    \item \textbf{Feature Validation}: Cross-validation stability and production readiness testing
    \item \textbf{Production Monitoring}: Drift detection and alerting for deployed features
    \item \textbf{Feature Store Integration}: Versioning and serving architecture
\end{enumerate}

\section{Advanced Feature Engineering Frameworks}

Modern feature engineering goes beyond manual transformations. This section presents automated and advanced techniques for systematic feature generation at scale.

\subsection{Automated Feature Generation with Genetic Programming}

Genetic programming can automatically discover complex feature transformations by evolving mathematical expressions that maximize predictive power.

\begin{lstlisting}[language=Python, caption={Automated Feature Generation Using Genetic Programming}]
from gplearn.genetic import SymbolicTransformer
from sklearn.preprocessing import StandardScaler
from typing import List, Callable
import numpy as np
import pandas as pd

class GeneticFeatureGenerator:
    """
    Generate features using genetic programming and symbolic regression.
    Automatically discovers mathematical transformations.
    """

    def __init__(self,
                 population_size: int = 1000,
                 generations: int = 20,
                 tournament_size: int = 20,
                 function_set: List[str] = None):
        """
        Args:
            population_size: Number of programs in each generation
            generations: Number of generations to evolve
            tournament_size: Selection tournament size
            function_set: Mathematical functions to use ('add', 'sub', 'mul',
                         'div', 'sqrt', 'log', 'abs', 'neg', 'inv', 'max', 'min')
        """
        if function_set is None:
            function_set = ['add', 'sub', 'mul', 'div', 'sqrt', 'log', 'abs']

        self.gp_transformer = SymbolicTransformer(
            population_size=population_size,
            generations=generations,
            tournament_size=tournament_size,
            function_set=function_set,
            parsimony_coefficient=0.001,  # Penalize complex expressions
            random_state=42,
            n_jobs=-1,
            verbose=1
        )
        self.feature_metadata: List[FeatureMetadata] = []

    def fit_transform(self, X: pd.DataFrame, y: pd.Series,
                     n_components: int = 10) -> pd.DataFrame:
        """
        Generate new features using genetic programming.

        Args:
            X: Input features
            y: Target variable
            n_components: Number of features to generate

        Returns:
            DataFrame with original and generated features
        """
        self.gp_transformer.n_components = n_components

        # Fit genetic programming transformer
        logger.info(f"Evolving {n_components} features over {self.gp_transformer.generations} generations...")
        X_gp = self.gp_transformer.fit_transform(X.values, y.values)

        # Create feature names and metadata
        result = X.copy()
        for i in range(n_components):
            feature_name = f"gp_feature_{i}"
            result[feature_name] = X_gp[:, i]

            # Get the evolved program (mathematical expression)
            program = self.gp_transformer._best_programs[i]

            self.feature_metadata.append(FeatureMetadata(
                name=feature_name,
                feature_type=FeatureType.NUMERICAL,
                source_columns=X.columns.tolist(),
                transformation=f"Genetic program: {str(program)}",
                scope=TransformationScope.ROW_LEVEL,
                created_at=datetime.now(),
                version="1.0.0",
                importance=float(program.fitness_),
                description=f"Auto-generated via genetic programming (fitness={program.fitness_:.4f})"
            ))

        logger.info(f"Generated {n_components} features with genetic programming")
        return result

    def get_metadata(self) -> List[FeatureMetadata]:
        return self.feature_metadata


class SymbolicRegressionFeatures:
    """
    Use symbolic regression to discover interpretable feature transformations.
    Useful for finding domain-meaningful formulas.
    """

    def __init__(self, max_complexity: int = 10):
        """
        Args:
            max_complexity: Maximum complexity of discovered formulas
        """
        self.max_complexity = max_complexity
        self.discovered_formulas: Dict[str, str] = {}
        self.metadata: List[FeatureMetadata] = []

    def discover_transformations(self, X: pd.DataFrame, y: pd.Series,
                                 columns: List[str] = None) -> pd.DataFrame:
        """
        Discover interpretable transformations for specific columns.

        Uses symbolic regression to find simple formulas that improve
        correlation with target variable.
        """
        if columns is None:
            columns = X.select_dtypes(include=[np.number]).columns.tolist()

        result = X.copy()

        for col in columns:
            if col not in X.columns:
                continue

            # Try common transformations and score them
            x_vals = X[col].values.reshape(-1, 1)

            transformations = {
                f"{col}_squared": lambda x: x ** 2,
                f"{col}_cubed": lambda x: x ** 3,
                f"{col}_sqrt": lambda x: np.sqrt(np.abs(x)),
                f"{col}_log1p": lambda x: np.log1p(np.abs(x)),
                f"{col}_exp": lambda x: np.exp(np.clip(x, -10, 10)),
                f"{col}_inv": lambda x: 1 / (x + 1e-10),
                f"{col}_sin": lambda x: np.sin(x),
                f"{col}_cos": lambda x: np.cos(x),
            }

            # Score each transformation by correlation with target
            for trans_name, trans_func in transformations.items():
                try:
                    transformed = trans_func(X[col].values)

                    # Calculate correlation with target
                    if not np.any(np.isnan(transformed)) and not np.any(np.isinf(transformed)):
                        corr = np.abs(np.corrcoef(transformed, y.values)[0, 1])

                        # Only keep if improves correlation significantly
                        original_corr = np.abs(np.corrcoef(X[col].values, y.values)[0, 1])

                        if corr > original_corr * 1.1:  # At least 10% improvement
                            result[trans_name] = transformed
                            self.discovered_formulas[trans_name] = trans_name.split('_', 1)[1]

                            self.metadata.append(FeatureMetadata(
                                name=trans_name,
                                feature_type=FeatureType.NUMERICAL,
                                source_columns=[col],
                                transformation=self.discovered_formulas[trans_name],
                                scope=TransformationScope.ROW_LEVEL,
                                created_at=datetime.now(),
                                version="1.0.0",
                                importance=float(corr),
                                description=f"Symbolic transformation (target_corr={corr:.4f})"
                            ))

                except Exception as e:
                    logger.debug(f"Transformation {trans_name} failed: {e}")
                    continue

        logger.info(f"Discovered {len(self.discovered_formulas)} beneficial transformations")
        return result

    def get_metadata(self) -> List[FeatureMetadata]:
        return self.metadata
\end{lstlisting}

\subsection{Domain-Specific Feature Libraries}

Different data types require specialized feature engineering approaches. Here we provide comprehensive libraries for time series, text, and graph data.

\begin{lstlisting}[language=Python, caption={Time Series Feature Library with Seasonality Detection}]
from statsmodels.tsa.seasonal import seasonal_decompose
from scipy.signal import find_peaks
from sklearn.decomposition import PCA
import warnings

class TimeSeriesFeatureLibrary:
    """
    Comprehensive time series feature extraction including:
    - Trend and seasonality decomposition
    - Autocorrelation features
    - Statistical moments over windows
    - Change point detection
    """

    def __init__(self, timestamp_col: str, value_col: str,
                 freq: str = 'D', seasonal_period: int = None):
        """
        Args:
            timestamp_col: Name of timestamp column
            value_col: Name of value column to extract features from
            freq: Frequency of time series ('D', 'H', 'M', etc.)
            seasonal_period: Period for seasonal decomposition (auto-detect if None)
        """
        self.timestamp_col = timestamp_col
        self.value_col = value_col
        self.freq = freq
        self.seasonal_period = seasonal_period
        self.metadata: List[FeatureMetadata] = []

    def extract_features(self, df: pd.DataFrame) -> pd.DataFrame:
        """Extract comprehensive time series features."""
        result = df.copy()
        result[self.timestamp_col] = pd.to_datetime(result[self.timestamp_col])
        result = result.sort_values(self.timestamp_col)

        # 1. Seasonal decomposition
        if len(result) >= 2 * (self.seasonal_period or 7):
            result = self._add_seasonal_features(result)

        # 2. Lag features with automatic selection
        optimal_lags = self._select_optimal_lags(result[self.value_col])
        result = self._add_lag_features(result, optimal_lags)

        # 3. Rolling statistics
        result = self._add_rolling_features(result)

        # 4. Autocorrelation features
        result = self._add_autocorrelation_features(result)

        # 5. Change point indicators
        result = self._add_changepoint_features(result)

        return result

    def _add_seasonal_features(self, df: pd.DataFrame) -> pd.DataFrame:
        """Decompose time series into trend, seasonal, and residual components."""
        try:
            ts = df.set_index(self.timestamp_col)[self.value_col]

            # Determine seasonal period if not provided
            if self.seasonal_period is None:
                period = self._detect_seasonality(ts)
            else:
                period = self.seasonal_period

            if period is None or period < 2:
                logger.info("No clear seasonality detected")
                return df

            # Perform decomposition
            with warnings.catch_warnings():
                warnings.simplefilter("ignore")
                decomposition = seasonal_decompose(ts, model='additive',
                                                   period=period, extrapolate_trend='freq')

            df[f'{self.value_col}_trend'] = decomposition.trend.values
            df[f'{self.value_col}_seasonal'] = decomposition.seasonal.values
            df[f'{self.value_col}_residual'] = decomposition.resid.values

            self.metadata.extend([
                FeatureMetadata(
                    name=f'{self.value_col}_trend',
                    feature_type=FeatureType.NUMERICAL,
                    source_columns=[self.value_col],
                    transformation=f"seasonal_decompose trend (period={period})",
                    scope=TransformationScope.GLOBAL_LEVEL,
                    created_at=datetime.now(),
                    version="1.0.0"
                ),
                FeatureMetadata(
                    name=f'{self.value_col}_seasonal',
                    feature_type=FeatureType.NUMERICAL,
                    source_columns=[self.value_col],
                    transformation=f"seasonal_decompose seasonal (period={period})",
                    scope=TransformationScope.GLOBAL_LEVEL,
                    created_at=datetime.now(),
                    version="1.0.0"
                )
            ])

            logger.info(f"Added seasonal decomposition features (period={period})")

        except Exception as e:
            logger.warning(f"Seasonal decomposition failed: {e}")

        return df

    def _detect_seasonality(self, ts: pd.Series, max_period: int = 365) -> Optional[int]:
        """Detect seasonality using autocorrelation."""
        from statsmodels.tsa.stattools import acf

        # Compute autocorrelation
        autocorr = acf(ts.dropna(), nlags=min(len(ts) // 2, max_period), fft=True)

        # Find peaks in autocorrelation
        peaks, properties = find_peaks(autocorr[1:], height=0.1)

        if len(peaks) > 0:
            # Return the most prominent peak
            peak_idx = peaks[np.argmax(properties['peak_heights'])]
            return peak_idx + 1

        return None

    def _select_optimal_lags(self, series: pd.Series, max_lags: int = 30) -> List[int]:
        """Select optimal lag values using PACF."""
        from statsmodels.tsa.stattools import pacf

        # Compute partial autocorrelation
        pacf_values = pacf(series.dropna(), nlags=max_lags)

        # Select lags with significant PACF (>0.1)
        significant_lags = [i for i in range(1, len(pacf_values))
                           if abs(pacf_values[i]) > 0.1]

        # Limit to top 5 lags
        return sorted(significant_lags[:5])

    def _add_lag_features(self, df: pd.DataFrame, lags: List[int]) -> pd.DataFrame:
        """Add lag features."""
        for lag in lags:
            col_name = f'{self.value_col}_lag_{lag}'
            df[col_name] = df[self.value_col].shift(lag)

            self.metadata.append(FeatureMetadata(
                name=col_name,
                feature_type=FeatureType.NUMERICAL,
                source_columns=[self.value_col],
                transformation=f"lag {lag} (auto-selected via PACF)",
                scope=TransformationScope.ROW_LEVEL,
                created_at=datetime.now(),
                version="1.0.0"
            ))

        return df

    def _add_rolling_features(self, df: pd.DataFrame,
                             windows: List[int] = [7, 14, 30]) -> pd.DataFrame:
        """Add rolling window statistics."""
        for window in windows:
            if window >= len(df):
                continue

            df[f'{self.value_col}_rolling_mean_{window}'] = \
                df[self.value_col].rolling(window, min_periods=1).mean()
            df[f'{self.value_col}_rolling_std_{window}'] = \
                df[self.value_col].rolling(window, min_periods=1).std()
            df[f'{self.value_col}_rolling_min_{window}'] = \
                df[self.value_col].rolling(window, min_periods=1).min()
            df[f'{self.value_col}_rolling_max_{window}'] = \
                df[self.value_col].rolling(window, min_periods=1).max()

            # Rolling rate of change
            df[f'{self.value_col}_rolling_roc_{window}'] = \
                df[self.value_col].pct_change(periods=window)

        return df

    def _add_autocorrelation_features(self, df: pd.DataFrame,
                                     lags: List[int] = [1, 7, 30]) -> pd.DataFrame:
        """Add autocorrelation at specific lags as features."""
        for lag in lags:
            df[f'{self.value_col}_autocorr_{lag}'] = \
                df[self.value_col].rolling(window=lag+10, min_periods=lag+1)\
                    .apply(lambda x: x.autocorr(lag=lag), raw=False)

        return df

    def _add_changepoint_features(self, df: pd.DataFrame) -> pd.DataFrame:
        """Detect and mark change points in the time series."""
        # Simple change point detection using rolling statistics
        window = min(30, len(df) // 4)

        rolling_mean = df[self.value_col].rolling(window, min_periods=1).mean()
        rolling_std = df[self.value_col].rolling(window, min_periods=1).std()

        # Detect points where value exceeds 2 standard deviations from rolling mean
        df[f'{self.value_col}_is_outlier'] = (
            np.abs(df[self.value_col] - rolling_mean) > 2 * rolling_std
        ).astype(int)

        # Detect trend changes (derivative changes sign)
        df[f'{self.value_col}_trend_change'] = (
            np.sign(df[self.value_col].diff()).diff().fillna(0) != 0
        ).astype(int)

        return df

    def get_metadata(self) -> List[FeatureMetadata]:
        return self.metadata
\end{lstlisting}

\subsection{Feature Interaction Discovery with Statistical Testing}

Manually specifying all feature interactions is infeasible for high-dimensional data. This class automatically discovers significant interactions.

\begin{lstlisting}[language=Python, caption={Automated Feature Interaction Discovery}]
from itertools import combinations
from scipy.stats import f_oneway, chi2_contingency
from statsmodels.stats.multitest import multipletests

class FeatureInteractionDiscovery:
    """
    Automatically discover significant feature interactions using:
    - Statistical tests (ANOVA, chi-squared)
    - Multiple testing corrections (Bonferroni, FDR)
    - Interaction strength scoring
    """

    def __init__(self, max_interactions: int = 100,
                 significance_level: float = 0.05,
                 correction_method: str = 'fdr_bh'):
        """
        Args:
            max_interactions: Maximum number of interactions to test
            significance_level: P-value threshold after correction
            correction_method: Multiple testing correction
                              ('bonferroni', 'fdr_bh', 'fdr_by')
        """
        self.max_interactions = max_interactions
        self.significance_level = significance_level
        self.correction_method = correction_method
        self.significant_interactions: List[tuple] = []
        self.metadata: List[FeatureMetadata] = []

    def discover_interactions(self, X: pd.DataFrame, y: pd.Series,
                             candidate_features: List[str] = None) -> pd.DataFrame:
        """
        Test all pairs of features for significant interactions with target.

        Returns DataFrame with original features plus significant interactions.
        """
        if candidate_features is None:
            candidate_features = X.select_dtypes(include=[np.number]).columns.tolist()

        # Limit candidates if too many
        if len(candidate_features) > 50:
            logger.warning(f"Too many candidates ({len(candidate_features)}), "
                          f"selecting top 50 by variance")
            variances = X[candidate_features].var()
            candidate_features = variances.nlargest(50).index.tolist()

        # Generate all pairs
        pairs = list(combinations(candidate_features, 2))

        # Limit number of tests
        if len(pairs) > self.max_interactions:
            pairs = pairs[:self.max_interactions]

        logger.info(f"Testing {len(pairs)} feature interactions...")

        # Test each interaction
        interaction_scores = []
        for feat1, feat2 in pairs:
            score, p_value = self._test_interaction(X[feat1], X[feat2], y)
            interaction_scores.append({
                'feature1': feat1,
                'feature2': feat2,
                'score': score,
                'p_value': p_value
            })

        # Apply multiple testing correction
        p_values = [item['p_value'] for item in interaction_scores]
        reject, corrected_pvals, _, _ = multipletests(
            p_values,
            alpha=self.significance_level,
            method=self.correction_method
        )

        # Keep only significant interactions
        result = X.copy()
        for i, (item, is_significant, corrected_p) in enumerate(
            zip(interaction_scores, reject, corrected_pvals)
        ):
            if is_significant:
                feat1 = item['feature1']
                feat2 = item['feature2']

                # Create interaction feature
                interaction_name = f"{feat1}_X_{feat2}"
                result[interaction_name] = X[feat1] * X[feat2]

                self.significant_interactions.append((feat1, feat2))

                self.metadata.append(FeatureMetadata(
                    name=interaction_name,
                    feature_type=FeatureType.NUMERICAL,
                    source_columns=[feat1, feat2],
                    transformation=f"multiplicative interaction (p={corrected_p:.4f})",
                    scope=TransformationScope.ROW_LEVEL,
                    created_at=datetime.now(),
                    version="1.0.0",
                    importance=item['score'],
                    description=f"Statistically significant interaction ({self.correction_method})"
                ))

        logger.info(f"Found {len(self.significant_interactions)} significant interactions "
                   f"(α={self.significance_level}, correction={self.correction_method})")

        return result

    def _test_interaction(self, feat1: pd.Series, feat2: pd.Series,
                         y: pd.Series) -> tuple:
        """
        Test if interaction between two features is significant.

        Returns (score, p_value) where higher score = stronger interaction.
        """
        # Create interaction term
        interaction = feat1 * feat2

        # Test if interaction improves prediction of target
        # For numerical target: use correlation
        # For categorical target: use ANOVA

        if pd.api.types.is_numeric_dtype(y):
            # Correlation of interaction with target
            corr = np.corrcoef(interaction.fillna(0), y)[0, 1]
            score = abs(corr)

            # Compute p-value using t-test
            n = len(interaction)
            t_stat = corr * np.sqrt(n - 2) / np.sqrt(1 - corr**2)
            from scipy.stats import t
            p_value = 2 * (1 - t.cdf(abs(t_stat), n - 2))
        else:
            # ANOVA for categorical target
            groups = [interaction[y == label].dropna()
                     for label in y.unique()]
            groups = [g for g in groups if len(g) > 0]

            if len(groups) < 2:
                return 0.0, 1.0

            f_stat, p_value = f_oneway(*groups)
            score = f_stat

        return score, p_value

    def get_metadata(self) -> List[FeatureMetadata]:
        return self.metadata
\end{lstlisting}

\subsection{Feature Embeddings for High-Cardinality Categoricals}

For categorical features with thousands of unique values, embeddings learned from neural networks can capture semantic relationships.

\begin{lstlisting}[language=Python, caption={Entity Embedding for High-Cardinality Features}]
import torch
import torch.nn as nn
from torch.utils.data import Dataset, DataLoader

class EntityEmbedding:
    """
    Learn dense embeddings for high-cardinality categorical features.

    Useful for features like:
    - Product IDs (10k+ unique products)
    - User IDs (100k+ users)
    - Store locations (1000+ stores)
    """

    def __init__(self, embedding_dim: int = 10, epochs: int = 50):
        """
        Args:
            embedding_dim: Dimension of learned embeddings
            epochs: Number of training epochs
        """
        self.embedding_dim = embedding_dim
        self.epochs = epochs
        self.embeddings_: Dict[str, nn.Embedding] = {}
        self.encoders_: Dict[str, Dict] = {}
        self.metadata: List[FeatureMetadata] = []

    def fit_transform(self, df: pd.DataFrame, categorical_cols: List[str],
                     target_col: str) -> pd.DataFrame:
        """
        Learn embeddings for categorical columns and transform data.

        Returns DataFrame with embedding dimensions as new features.
        """
        result = df.copy()

        for col in categorical_cols:
            if col not in df.columns:
                continue

            # Encode categories to integers
            unique_vals = df[col].unique()
            val_to_idx = {val: idx for idx, val in enumerate(unique_vals)}
            self.encoders_[col] = val_to_idx

            # Create embedding model
            n_categories = len(unique_vals)
            embedding = self._train_embedding(
                df[col], df[target_col], n_categories
            )
            self.embeddings_[col] = embedding

            # Transform data using learned embeddings
            encoded = df[col].map(val_to_idx).fillna(0).astype(int)
            embedding_values = embedding(torch.LongTensor(encoded.values)).detach().numpy()

            # Add embedding dimensions as features
            for dim in range(self.embedding_dim):
                feat_name = f"{col}_emb_{dim}"
                result[feat_name] = embedding_values[:, dim]

                self.metadata.append(FeatureMetadata(
                    name=feat_name,
                    feature_type=FeatureType.EMBEDDING,
                    source_columns=[col],
                    transformation=f"neural embedding dimension {dim}/{self.embedding_dim}",
                    scope=TransformationScope.GLOBAL_LEVEL,
                    created_at=datetime.now(),
                    version="1.0.0",
                    description=f"Learned embedding for {col} (n_categories={n_categories})"
                ))

            logger.info(f"Created {self.embedding_dim}-dim embedding for '{col}' "
                       f"({n_categories} categories)")

        return result

    def _train_embedding(self, categorical_series: pd.Series,
                        target: pd.Series, n_categories: int) -> nn.Embedding:
        """Train embedding using simple neural network."""

        class EmbeddingModel(nn.Module):
            def __init__(self, n_categories, embedding_dim):
                super().__init__()
                self.embedding = nn.Embedding(n_categories, embedding_dim)
                self.fc = nn.Linear(embedding_dim, 1)

            def forward(self, x):
                embedded = self.embedding(x)
                out = self.fc(embedded)
                return out.squeeze()

        # Prepare data
        encoder = self.encoders_[categorical_series.name]
        X_encoded = categorical_series.map(encoder).fillna(0).astype(int).values
        y_values = target.values if pd.api.types.is_numeric_dtype(target) \
                   else pd.factorize(target)[0]

        # Create model
        model = EmbeddingModel(n_categories, self.embedding_dim)
        optimizer = torch.optim.Adam(model.parameters(), lr=0.01)
        criterion = nn.MSELoss()

        # Training loop
        X_tensor = torch.LongTensor(X_encoded)
        y_tensor = torch.FloatTensor(y_values)

        model.train()
        for epoch in range(self.epochs):
            optimizer.zero_grad()
            outputs = model(X_tensor)
            loss = criterion(outputs, y_tensor)
            loss.backward()
            optimizer.step()

        # Return learned embedding layer
        return model.embedding

    def get_metadata(self) -> List[FeatureMetadata]:
        return self.metadata
\end{lstlisting}

\section{Feature Engineering Pipeline Framework}

A robust feature engineering pipeline must ensure transformations are reproducible, validated, and production-ready. We'll build a framework that tracks every transformation, validates feature quality, and prevents common pitfalls like data leakage.

\subsection{Core Pipeline Architecture}

\begin{lstlisting}[language=Python, caption={Feature Engineering Pipeline Framework}]
from dataclasses import dataclass, field
from typing import Protocol, List, Dict, Any, Optional, Callable
from enum import Enum
import pandas as pd
import numpy as np
from datetime import datetime
import logging
from pathlib import Path
import json
import hashlib

logger = logging.getLogger(__name__)

class FeatureType(Enum):
    """Types of features for tracking and validation."""
    NUMERICAL = "numerical"
    CATEGORICAL = "categorical"
    TEMPORAL = "temporal"
    BOOLEAN = "boolean"
    TEXT = "text"
    EMBEDDING = "embedding"

class TransformationScope(Enum):
    """Scope of feature transformation."""
    ROW_LEVEL = "row_level"  # Operates on individual rows
    GROUP_LEVEL = "group_level"  # Requires grouping (e.g., mean by category)
    GLOBAL_LEVEL = "global_level"  # Requires full dataset (e.g., normalization)

@dataclass
class FeatureMetadata:
    """Metadata about a generated feature."""
    name: str
    feature_type: FeatureType
    source_columns: List[str]
    transformation: str
    scope: TransformationScope
    created_at: datetime
    version: str
    importance: Optional[float] = None
    description: str = ""

    def to_dict(self) -> Dict[str, Any]:
        """Convert to dictionary for serialization."""
        return {
            "name": self.name,
            "feature_type": self.feature_type.value,
            "source_columns": self.source_columns,
            "transformation": self.transformation,
            "scope": self.scope.value,
            "created_at": self.created_at.isoformat(),
            "version": self.version,
            "importance": self.importance,
            "description": self.description
        }

@dataclass
class FeatureValidationResult:
    """Results from feature validation checks."""
    feature_name: str
    is_valid: bool
    checks_passed: List[str]
    checks_failed: List[str]
    warnings: List[str]
    quality_score: float  # 0-100

    def __str__(self) -> str:
        status = "VALID" if self.is_valid else "INVALID"
        return (f"Feature '{self.feature_name}': {status} "
                f"(Quality: {self.quality_score:.1f}/100)")

class FeatureTransformer(Protocol):
    """Protocol for feature transformation functions."""

    def transform(self, df: pd.DataFrame) -> pd.DataFrame:
        """Transform dataframe to create new features."""
        ...

    def get_metadata(self) -> List[FeatureMetadata]:
        """Return metadata for created features."""
        ...

@dataclass
class TransformationStep:
    """A single step in the feature engineering pipeline."""
    name: str
    transformer: FeatureTransformer
    enabled: bool = True
    metadata: List[FeatureMetadata] = field(default_factory=list)

    def execute(self, df: pd.DataFrame) -> pd.DataFrame:
        """Execute transformation if enabled."""
        if not self.enabled:
            logger.info(f"Skipping disabled transformation: {self.name}")
            return df

        logger.info(f"Executing transformation: {self.name}")
        try:
            result = self.transformer.transform(df)
            self.metadata = self.transformer.get_metadata()
            return result
        except Exception as e:
            logger.error(f"Transformation '{self.name}' failed: {e}")
            raise

class FeatureEngineeringPipeline:
    """
    Comprehensive feature engineering pipeline with validation,
    versioning, and production-ready transformations.
    """

    def __init__(self, name: str, version: str = "1.0.0"):
        self.name = name
        self.version = version
        self.steps: List[TransformationStep] = []
        self.feature_metadata: Dict[str, FeatureMetadata] = {}
        self.execution_history: List[Dict[str, Any]] = []

    def add_step(self, name: str, transformer: FeatureTransformer) -> None:
        """Add a transformation step to the pipeline."""
        step = TransformationStep(name=name, transformer=transformer)
        self.steps.append(step)
        logger.info(f"Added transformation step: {name}")

    def fit_transform(self, df: pd.DataFrame,
                      validate: bool = True) -> pd.DataFrame:
        """
        Execute all transformation steps and optionally validate.

        Args:
            df: Input dataframe
            validate: Whether to validate features after creation

        Returns:
            Transformed dataframe with new features
        """
        result = df.copy()
        start_time = datetime.now()

        logger.info(f"Starting pipeline '{self.name}' v{self.version}")
        logger.info(f"Input shape: {result.shape}")

        for step in self.steps:
            step_start = datetime.now()
            result = step.execute(result)
            step_duration = (datetime.now() - step_start).total_seconds()

            # Update feature metadata
            for metadata in step.metadata:
                self.feature_metadata[metadata.name] = metadata

            logger.info(f"Step '{step.name}' completed in {step_duration:.2f}s")
            logger.info(f"Output shape: {result.shape}")

        duration = (datetime.now() - start_time).total_seconds()

        # Record execution
        self.execution_history.append({
            "timestamp": start_time.isoformat(),
            "duration_seconds": duration,
            "input_shape": df.shape,
            "output_shape": result.shape,
            "features_created": len(self.feature_metadata)
        })

        logger.info(f"Pipeline completed in {duration:.2f}s")
        logger.info(f"Created {len(self.feature_metadata)} features")

        if validate:
            validation_results = self.validate_features(result)
            self._log_validation_results(validation_results)

        return result

    def validate_features(self, df: pd.DataFrame) -> List[FeatureValidationResult]:
        """
        Validate all created features for quality and correctness.

        Checks:
        - No constant features (zero variance)
        - No features with excessive missing values (>50%)
        - Numerical features have reasonable distributions
        - No infinite or NaN values after transformation
        """
        results = []

        for feature_name, metadata in self.feature_metadata.items():
            if feature_name not in df.columns:
                results.append(FeatureValidationResult(
                    feature_name=feature_name,
                    is_valid=False,
                    checks_passed=[],
                    checks_failed=["Feature not found in dataframe"],
                    warnings=[],
                    quality_score=0.0
                ))
                continue

            series = df[feature_name]
            checks_passed = []
            checks_failed = []
            warnings = []

            # Check 1: Missing values
            missing_pct = series.isna().sum() / len(series) * 100
            if missing_pct <= 50:
                checks_passed.append(f"Missing values: {missing_pct:.1f}%")
            else:
                checks_failed.append(f"Excessive missing values: {missing_pct:.1f}%")

            if 20 < missing_pct <= 50:
                warnings.append(f"High missing rate: {missing_pct:.1f}%")

            # Check 2: Constant features
            if metadata.feature_type == FeatureType.NUMERICAL:
                variance = series.var()
                if variance > 0:
                    checks_passed.append(f"Non-constant (var={variance:.4f})")
                else:
                    checks_failed.append("Zero variance (constant feature)")

            # Check 3: Infinite values
            if metadata.feature_type == FeatureType.NUMERICAL:
                inf_count = np.isinf(series).sum()
                if inf_count == 0:
                    checks_passed.append("No infinite values")
                else:
                    checks_failed.append(f"Contains {inf_count} infinite values")

            # Check 4: Cardinality (for categorical)
            if metadata.feature_type == FeatureType.CATEGORICAL:
                cardinality = series.nunique()
                if cardinality < len(series) * 0.95:
                    checks_passed.append(f"Reasonable cardinality: {cardinality}")
                else:
                    warnings.append(f"High cardinality: {cardinality}")

            # Calculate quality score
            total_checks = len(checks_passed) + len(checks_failed)
            quality_score = (len(checks_passed) / total_checks * 100) if total_checks > 0 else 0

            is_valid = len(checks_failed) == 0

            results.append(FeatureValidationResult(
                feature_name=feature_name,
                is_valid=is_valid,
                checks_passed=checks_passed,
                checks_failed=checks_failed,
                warnings=warnings,
                quality_score=quality_score
            ))

        return results

    def _log_validation_results(self, results: List[FeatureValidationResult]) -> None:
        """Log validation results."""
        valid_count = sum(1 for r in results if r.is_valid)
        logger.info(f"Validation: {valid_count}/{len(results)} features valid")

        for result in results:
            if not result.is_valid:
                logger.warning(f"Invalid feature: {result}")
                for failure in result.checks_failed:
                    logger.warning(f"  - {failure}")

    def get_feature_lineage(self, feature_name: str) -> Optional[Dict[str, Any]]:
        """Get the lineage (source and transformations) of a feature."""
        if feature_name not in self.feature_metadata:
            return None

        metadata = self.feature_metadata[feature_name]
        return {
            "feature": feature_name,
            "source_columns": metadata.source_columns,
            "transformation": metadata.transformation,
            "scope": metadata.scope.value,
            "created_at": metadata.created_at.isoformat(),
            "version": metadata.version
        }

    def export_metadata(self, output_path: Path) -> None:
        """Export all feature metadata to JSON."""
        metadata_dict = {
            "pipeline_name": self.name,
            "pipeline_version": self.version,
            "features": {
                name: meta.to_dict()
                for name, meta in self.feature_metadata.items()
            },
            "execution_history": self.execution_history
        }

        with open(output_path, 'w') as f:
            json.dump(metadata_dict, f, indent=2)

        logger.info(f"Exported metadata to {output_path}")

    def compute_pipeline_hash(self) -> str:
        """Compute hash of pipeline configuration for versioning."""
        config = {
            "name": self.name,
            "version": self.version,
            "steps": [
                {
                    "name": step.name,
                    "enabled": step.enabled,
                    "transformer": step.transformer.__class__.__name__
                }
                for step in self.steps
            ]
        }

        config_str = json.dumps(config, sort_keys=True)
        return hashlib.sha256(config_str.encode()).hexdigest()[:16]
\end{lstlisting}

\subsection{Pipeline Usage Example}

\begin{lstlisting}[language=Python, caption={Using the Feature Engineering Pipeline}]
# Example: Create a pipeline for customer churn prediction
pipeline = FeatureEngineeringPipeline(
    name="customer_churn_features",
    version="1.0.0"
)

# Add transformation steps (transformers defined in next sections)
pipeline.add_step("temporal_features", TemporalFeatureExtractor())
pipeline.add_step("categorical_encoding", CategoricalEncoder())
pipeline.add_step("numerical_transformations", NumericalTransformer())

# Execute pipeline with validation
df_transformed = pipeline.fit_transform(df_raw, validate=True)

# Export metadata for reproducibility
pipeline.export_metadata(Path("feature_metadata.json"))

# Check specific feature lineage
lineage = pipeline.get_feature_lineage("days_since_last_purchase")
print(lineage)
# Output: {
#   'feature': 'days_since_last_purchase',
#   'source_columns': ['last_purchase_date'],
#   'transformation': 'days_since',
#   'scope': 'row_level',
#   ...
# }
\end{lstlisting}

\section{Domain-Driven Feature Creation}

Feature engineering should be driven by domain knowledge and statistical principles. This section presents systematic approaches for temporal, categorical, and numerical feature extraction.

\subsection{Temporal Feature Extraction}

Time-based features often provide strong predictive signals. We'll extract cyclic patterns, trends, and event-based features.

\begin{lstlisting}[language=Python, caption={Temporal Feature Extraction}]
from typing import List
import pandas as pd
import numpy as np
from datetime import datetime

class TemporalFeatureExtractor:
    """Extract temporal features from datetime columns."""

    def __init__(self, datetime_columns: List[str],
                 reference_date: Optional[datetime] = None):
        self.datetime_columns = datetime_columns
        self.reference_date = reference_date or datetime.now()
        self.metadata: List[FeatureMetadata] = []

    def transform(self, df: pd.DataFrame) -> pd.DataFrame:
        """Extract temporal features."""
        result = df.copy()
        self.metadata = []

        for col in self.datetime_columns:
            if col not in df.columns:
                logger.warning(f"Column '{col}' not found, skipping")
                continue

            # Ensure datetime type
            dt_series = pd.to_datetime(result[col], errors='coerce')

            # Cyclic features: hour, day of week, month
            result[f"{col}_hour"] = dt_series.dt.hour
            result[f"{col}_hour_sin"] = np.sin(2 * np.pi * result[f"{col}_hour"] / 24)
            result[f"{col}_hour_cos"] = np.cos(2 * np.pi * result[f"{col}_hour"] / 24)

            self._add_metadata(
                name=f"{col}_hour_sin",
                feature_type=FeatureType.NUMERICAL,
                source_columns=[col],
                transformation="sin(2*pi*hour/24) - cyclic hour encoding"
            )

            result[f"{col}_dayofweek"] = dt_series.dt.dayofweek
            result[f"{col}_dayofweek_sin"] = np.sin(
                2 * np.pi * result[f"{col}_dayofweek"] / 7
            )
            result[f"{col}_dayofweek_cos"] = np.cos(
                2 * np.pi * result[f"{col}_dayofweek"] / 7
            )

            self._add_metadata(
                name=f"{col}_dayofweek_sin",
                feature_type=FeatureType.NUMERICAL,
                source_columns=[col],
                transformation="sin(2*pi*dayofweek/7) - cyclic day encoding"
            )

            result[f"{col}_month"] = dt_series.dt.month
            result[f"{col}_month_sin"] = np.sin(2 * np.pi * result[f"{col}_month"] / 12)
            result[f"{col}_month_cos"] = np.cos(2 * np.pi * result[f"{col}_month"] / 12)

            # Boolean flags
            result[f"{col}_is_weekend"] = dt_series.dt.dayofweek.isin([5, 6]).astype(int)
            result[f"{col}_is_month_start"] = dt_series.dt.is_month_start.astype(int)
            result[f"{col}_is_month_end"] = dt_series.dt.is_month_end.astype(int)

            self._add_metadata(
                name=f"{col}_is_weekend",
                feature_type=FeatureType.BOOLEAN,
                source_columns=[col],
                transformation="is_weekend flag (Saturday/Sunday)"
            )

            # Days since reference date
            days_since = (self.reference_date - dt_series).dt.days
            result[f"{col}_days_since"] = days_since

            self._add_metadata(
                name=f"{col}_days_since",
                feature_type=FeatureType.NUMERICAL,
                source_columns=[col],
                transformation=f"days since {self.reference_date.date()}"
            )

            # Quarter
            result[f"{col}_quarter"] = dt_series.dt.quarter

        return result

    def _add_metadata(self, name: str, feature_type: FeatureType,
                     source_columns: List[str], transformation: str) -> None:
        """Add metadata for a created feature."""
        self.metadata.append(FeatureMetadata(
            name=name,
            feature_type=feature_type,
            source_columns=source_columns,
            transformation=transformation,
            scope=TransformationScope.ROW_LEVEL,
            created_at=datetime.now(),
            version="1.0.0"
        ))

    def get_metadata(self) -> List[FeatureMetadata]:
        """Return metadata for all created features."""
        return self.metadata


class LagFeatureCreator:
    """Create lag features for time series data."""

    def __init__(self, columns: List[str], lags: List[int],
                 group_by: Optional[List[str]] = None):
        self.columns = columns
        self.lags = lags
        self.group_by = group_by
        self.metadata: List[FeatureMetadata] = []

    def transform(self, df: pd.DataFrame) -> pd.DataFrame:
        """Create lag features."""
        result = df.copy()
        self.metadata = []

        for col in self.columns:
            if col not in df.columns:
                continue

            for lag in self.lags:
                if self.group_by:
                    # Group-wise lags (e.g., per customer)
                    lag_col = f"{col}_lag_{lag}"
                    result[lag_col] = result.groupby(self.group_by)[col].shift(lag)
                    scope = TransformationScope.GROUP_LEVEL
                else:
                    # Global lags
                    lag_col = f"{col}_lag_{lag}"
                    result[lag_col] = result[col].shift(lag)
                    scope = TransformationScope.ROW_LEVEL

                self.metadata.append(FeatureMetadata(
                    name=lag_col,
                    feature_type=FeatureType.NUMERICAL,
                    source_columns=[col] + (self.group_by or []),
                    transformation=f"lag {lag} periods",
                    scope=scope,
                    created_at=datetime.now(),
                    version="1.0.0"
                ))

        return result

    def get_metadata(self) -> List[FeatureMetadata]:
        return self.metadata


class RollingFeatureCreator:
    """Create rolling window statistics."""

    def __init__(self, columns: List[str], windows: List[int],
                 statistics: List[str] = ['mean', 'std', 'min', 'max'],
                 group_by: Optional[List[str]] = None):
        self.columns = columns
        self.windows = windows
        self.statistics = statistics
        self.group_by = group_by
        self.metadata: List[FeatureMetadata] = []

    def transform(self, df: pd.DataFrame) -> pd.DataFrame:
        """Create rolling window features."""
        result = df.copy()
        self.metadata = []

        for col in self.columns:
            if col not in df.columns:
                continue

            for window in self.windows:
                for stat in self.statistics:
                    feature_name = f"{col}_rolling_{window}_{stat}"

                    if self.group_by:
                        # Group-wise rolling (e.g., per customer)
                        result[feature_name] = (
                            result.groupby(self.group_by)[col]
                            .transform(lambda x: x.rolling(window, min_periods=1)
                                      .agg(stat))
                        )
                        scope = TransformationScope.GROUP_LEVEL
                    else:
                        # Global rolling
                        result[feature_name] = (
                            result[col].rolling(window, min_periods=1).agg(stat)
                        )
                        scope = TransformationScope.GLOBAL_LEVEL

                    self.metadata.append(FeatureMetadata(
                        name=feature_name,
                        feature_type=FeatureType.NUMERICAL,
                        source_columns=[col] + (self.group_by or []),
                        transformation=f"rolling {stat} over {window} periods",
                        scope=scope,
                        created_at=datetime.now(),
                        version="1.0.0"
                    ))

        return result

    def get_metadata(self) -> List[FeatureMetadata]:
        return self.metadata
\end{lstlisting}

\subsection{Categorical Feature Encoding}

Categorical variables require special handling, especially high-cardinality features. We'll implement multiple encoding strategies with automatic cardinality detection.

\begin{lstlisting}[language=Python, caption={Categorical Feature Encoding with High-Cardinality Handling}]
from sklearn.preprocessing import LabelEncoder
from typing import Dict, Optional
import category_encoders as ce  # pip install category-encoders

class EncodingStrategy(Enum):
    """Encoding strategies for categorical variables."""
    ONE_HOT = "one_hot"
    LABEL = "label"
    TARGET = "target"  # Mean target encoding
    FREQUENCY = "frequency"
    ORDINAL = "ordinal"

class CategoricalEncoder:
    """
    Encode categorical features with automatic strategy selection
    based on cardinality.
    """

    def __init__(self,
                 target_column: Optional[str] = None,
                 max_cardinality_onehot: int = 10,
                 min_samples_target_encode: int = 5):
        """
        Args:
            target_column: Target for target encoding
            max_cardinality_onehot: Max unique values for one-hot encoding
            min_samples_target_encode: Min samples per category for target encoding
        """
        self.target_column = target_column
        self.max_cardinality_onehot = max_cardinality_onehot
        self.min_samples_target_encode = min_samples_target_encode
        self.metadata: List[FeatureMetadata] = []
        self.encoders: Dict[str, Any] = {}
        self.strategies: Dict[str, EncodingStrategy] = {}

    def transform(self, df: pd.DataFrame) -> pd.DataFrame:
        """Encode categorical columns."""
        result = df.copy()
        self.metadata = []

        # Identify categorical columns
        categorical_cols = result.select_dtypes(
            include=['object', 'category']
        ).columns.tolist()

        # Remove target from encoding
        if self.target_column and self.target_column in categorical_cols:
            categorical_cols.remove(self.target_column)

        for col in categorical_cols:
            cardinality = result[col].nunique()

            # Choose encoding strategy
            if cardinality <= self.max_cardinality_onehot:
                strategy = EncodingStrategy.ONE_HOT
                result = self._one_hot_encode(result, col)
            elif cardinality > 100:
                # High cardinality: use target or frequency encoding
                if self.target_column and self.target_column in result.columns:
                    strategy = EncodingStrategy.TARGET
                    result = self._target_encode(result, col)
                else:
                    strategy = EncodingStrategy.FREQUENCY
                    result = self._frequency_encode(result, col)
            else:
                # Medium cardinality: label encoding
                strategy = EncodingStrategy.LABEL
                result = self._label_encode(result, col)

            self.strategies[col] = strategy
            logger.info(f"Encoded '{col}' (cardinality={cardinality}) "
                       f"using {strategy.value}")

        return result

    def _one_hot_encode(self, df: pd.DataFrame, col: str) -> pd.DataFrame:
        """One-hot encode a categorical column."""
        dummies = pd.get_dummies(df[col], prefix=col, drop_first=True)

        for dummy_col in dummies.columns:
            self.metadata.append(FeatureMetadata(
                name=dummy_col,
                feature_type=FeatureType.BOOLEAN,
                source_columns=[col],
                transformation=f"one-hot encoding of {col}",
                scope=TransformationScope.ROW_LEVEL,
                created_at=datetime.now(),
                version="1.0.0"
            ))

        result = pd.concat([df.drop(columns=[col]), dummies], axis=1)
        return result

    def _label_encode(self, df: pd.DataFrame, col: str) -> pd.DataFrame:
        """Label encode a categorical column."""
        encoder = LabelEncoder()
        encoded_col = f"{col}_label"

        df[encoded_col] = encoder.fit_transform(df[col].astype(str))
        self.encoders[col] = encoder

        self.metadata.append(FeatureMetadata(
            name=encoded_col,
            feature_type=FeatureType.NUMERICAL,
            source_columns=[col],
            transformation=f"label encoding of {col}",
            scope=TransformationScope.GLOBAL_LEVEL,
            created_at=datetime.now(),
            version="1.0.0",
            description=f"Mapping: {dict(enumerate(encoder.classes_))}"
        ))

        return df.drop(columns=[col])

    def _target_encode(self, df: pd.DataFrame, col: str) -> pd.DataFrame:
        """
        Target encode using mean of target variable.
        Includes smoothing to handle low-frequency categories.
        """
        if not self.target_column or self.target_column not in df.columns:
            logger.warning(f"Target column not available, using frequency encoding")
            return self._frequency_encode(df, col)

        # Calculate global mean
        global_mean = df[self.target_column].mean()

        # Calculate category means with counts
        stats = df.groupby(col)[self.target_column].agg(['mean', 'count'])

        # Smoothing: blend category mean with global mean based on count
        # More samples = more weight on category mean
        alpha = 1 / (1 + np.exp(-(stats['count'] - self.min_samples_target_encode)))
        stats['smoothed_mean'] = alpha * stats['mean'] + (1 - alpha) * global_mean

        # Map to dataframe
        encoded_col = f"{col}_target"
        df[encoded_col] = df[col].map(stats['smoothed_mean'])

        # Handle unseen categories
        df[encoded_col].fillna(global_mean, inplace=True)

        self.encoders[col] = stats['smoothed_mean'].to_dict()

        self.metadata.append(FeatureMetadata(
            name=encoded_col,
            feature_type=FeatureType.NUMERICAL,
            source_columns=[col, self.target_column],
            transformation=f"target encoding with smoothing (alpha-based)",
            scope=TransformationScope.GLOBAL_LEVEL,
            created_at=datetime.now(),
            version="1.0.0",
            description=f"Smoothed mean of {self.target_column} by {col}"
        ))

        return df.drop(columns=[col])

    def _frequency_encode(self, df: pd.DataFrame, col: str) -> pd.DataFrame:
        """Encode by frequency of occurrence."""
        freq = df[col].value_counts(normalize=True).to_dict()

        encoded_col = f"{col}_freq"
        df[encoded_col] = df[col].map(freq)

        self.encoders[col] = freq

        self.metadata.append(FeatureMetadata(
            name=encoded_col,
            feature_type=FeatureType.NUMERICAL,
            source_columns=[col],
            transformation=f"frequency encoding of {col}",
            scope=TransformationScope.GLOBAL_LEVEL,
            created_at=datetime.now(),
            version="1.0.0",
            description=f"Normalized frequency of occurrence"
        ))

        return df.drop(columns=[col])

    def get_metadata(self) -> List[FeatureMetadata]:
        return self.metadata
\end{lstlisting}

\subsection{Numerical Feature Transformations}

Numerical features often benefit from transformations to handle skewness, outliers, and scale.

\begin{lstlisting}[language=Python, caption={Numerical Feature Transformations}]
from sklearn.preprocessing import StandardScaler, RobustScaler, PowerTransformer
from scipy import stats

class NumericalTransformer:
    """Transform numerical features for better model performance."""

    def __init__(self,
                 columns: Optional[List[str]] = None,
                 auto_transform: bool = True,
                 skew_threshold: float = 1.0):
        """
        Args:
            columns: Specific columns to transform (None = all numerical)
            auto_transform: Automatically apply transformations based on distribution
            skew_threshold: Skewness threshold for log/power transforms
        """
        self.columns = columns
        self.auto_transform = auto_transform
        self.skew_threshold = skew_threshold
        self.metadata: List[FeatureMetadata] = []
        self.scalers: Dict[str, Any] = {}

    def transform(self, df: pd.DataFrame) -> pd.DataFrame:
        """Transform numerical features."""
        result = df.copy()
        self.metadata = []

        # Identify numerical columns
        if self.columns is None:
            numerical_cols = result.select_dtypes(
                include=[np.number]
            ).columns.tolist()
        else:
            numerical_cols = self.columns

        for col in numerical_cols:
            if col not in result.columns:
                continue

            series = result[col]

            # Skip if all NaN
            if series.isna().all():
                continue

            # Calculate skewness
            skewness = stats.skew(series.dropna())

            if self.auto_transform and abs(skewness) > self.skew_threshold:
                # Apply log transform for positive skewed data
                if skewness > self.skew_threshold and (series > 0).all():
                    result[f"{col}_log"] = np.log1p(series)
                    self.metadata.append(FeatureMetadata(
                        name=f"{col}_log",
                        feature_type=FeatureType.NUMERICAL,
                        source_columns=[col],
                        transformation=f"log1p transform (original skew={skewness:.2f})",
                        scope=TransformationScope.ROW_LEVEL,
                        created_at=datetime.now(),
                        version="1.0.0"
                    ))

                # Square root for moderate positive skew
                elif 0 < skewness <= self.skew_threshold and (series >= 0).all():
                    result[f"{col}_sqrt"] = np.sqrt(series)
                    self.metadata.append(FeatureMetadata(
                        name=f"{col}_sqrt",
                        feature_type=FeatureType.NUMERICAL,
                        source_columns=[col],
                        transformation=f"sqrt transform (original skew={skewness:.2f})",
                        scope=TransformationScope.ROW_LEVEL,
                        created_at=datetime.now(),
                        version="1.0.0"
                    ))

            # Robust scaling (median and IQR, resistant to outliers)
            scaler = RobustScaler()
            result[f"{col}_robust_scaled"] = scaler.fit_transform(
                series.values.reshape(-1, 1)
            )
            self.scalers[col] = scaler

            self.metadata.append(FeatureMetadata(
                name=f"{col}_robust_scaled",
                feature_type=FeatureType.NUMERICAL,
                source_columns=[col],
                transformation="robust scaling (median, IQR)",
                scope=TransformationScope.GLOBAL_LEVEL,
                created_at=datetime.now(),
                version="1.0.0"
            ))

            # Create binned version for categorical interactions
            result[f"{col}_binned"] = pd.qcut(
                series, q=5, labels=['very_low', 'low', 'medium', 'high', 'very_high'],
                duplicates='drop'
            )

            self.metadata.append(FeatureMetadata(
                name=f"{col}_binned",
                feature_type=FeatureType.CATEGORICAL,
                source_columns=[col],
                transformation="quintile binning (5 bins)",
                scope=TransformationScope.GLOBAL_LEVEL,
                created_at=datetime.now(),
                version="1.0.0"
            ))

        return result

    def get_metadata(self) -> List[FeatureMetadata]:
        return self.metadata


class InteractionFeatureCreator:
    """Create interaction features between numerical columns."""

    def __init__(self, column_pairs: List[tuple]):
        """
        Args:
            column_pairs: List of (col1, col2) tuples to create interactions
        """
        self.column_pairs = column_pairs
        self.metadata: List[FeatureMetadata] = []

    def transform(self, df: pd.DataFrame) -> pd.DataFrame:
        """Create interaction features."""
        result = df.copy()
        self.metadata = []

        for col1, col2 in self.column_pairs:
            if col1 not in df.columns or col2 not in df.columns:
                logger.warning(f"Columns '{col1}' or '{col2}' not found")
                continue

            # Multiplicative interaction
            mult_col = f"{col1}_x_{col2}"
            result[mult_col] = result[col1] * result[col2]

            self.metadata.append(FeatureMetadata(
                name=mult_col,
                feature_type=FeatureType.NUMERICAL,
                source_columns=[col1, col2],
                transformation="multiplicative interaction",
                scope=TransformationScope.ROW_LEVEL,
                created_at=datetime.now(),
                version="1.0.0"
            ))

            # Ratio (if col2 non-zero)
            if (result[col2] != 0).all():
                ratio_col = f"{col1}_div_{col2}"
                result[ratio_col] = result[col1] / result[col2]

                self.metadata.append(FeatureMetadata(
                    name=ratio_col,
                    feature_type=FeatureType.NUMERICAL,
                    source_columns=[col1, col2],
                    transformation="ratio feature",
                    scope=TransformationScope.ROW_LEVEL,
                    created_at=datetime.now(),
                    version="1.0.0"
                ))

        return result

    def get_metadata(self) -> List[FeatureMetadata]:
        return self.metadata
\end{lstlisting}

\section{Feature Selection}

Not all engineered features improve model performance. Systematic feature selection identifies the most predictive features while removing redundant or noisy ones.

\subsection{Statistical Feature Selection}

\begin{lstlisting}[language=Python, caption={Statistical Feature Selection Methods}]
from sklearn.feature_selection import (
    SelectKBest, f_classif, f_regression, mutual_info_classif,
    mutual_info_regression, RFE
)
from sklearn.ensemble import RandomForestClassifier, RandomForestRegressor
from typing import Union

@dataclass
class FeatureSelectionResult:
    """Results from feature selection."""
    selected_features: List[str]
    feature_scores: Dict[str, float]
    method: str
    threshold: Optional[float] = None

    def get_top_k(self, k: int) -> List[str]:
        """Get top k features by score."""
        sorted_features = sorted(
            self.feature_scores.items(),
            key=lambda x: x[1],
            reverse=True
        )
        return [f for f, _ in sorted_features[:k]]

class FeatureSelector:
    """
    Comprehensive feature selection using multiple methods:
    - Statistical tests (ANOVA F-test, chi-squared)
    - Information theory (mutual information)
    - Model-based importance (Random Forest)
    - Recursive feature elimination (RFE)
    """

    def __init__(self, task_type: str = "classification"):
        """
        Args:
            task_type: 'classification' or 'regression'
        """
        if task_type not in ["classification", "regression"]:
            raise ValueError("task_type must be 'classification' or 'regression'")

        self.task_type = task_type
        self.selection_results: Dict[str, FeatureSelectionResult] = {}

    def select_by_statistical_test(
        self,
        X: pd.DataFrame,
        y: pd.Series,
        k: int = 10
    ) -> FeatureSelectionResult:
        """
        Select features using statistical tests.
        - Classification: ANOVA F-test
        - Regression: F-test for regression
        """
        if self.task_type == "classification":
            selector = SelectKBest(score_func=f_classif, k=min(k, X.shape[1]))
        else:
            selector = SelectKBest(score_func=f_regression, k=min(k, X.shape[1]))

        selector.fit(X, y)

        # Get scores for all features
        scores = dict(zip(X.columns, selector.scores_))

        # Get selected features
        selected_mask = selector.get_support()
        selected_features = X.columns[selected_mask].tolist()

        result = FeatureSelectionResult(
            selected_features=selected_features,
            feature_scores=scores,
            method=f"statistical_test_{self.task_type}"
        )

        self.selection_results['statistical_test'] = result
        logger.info(f"Statistical test selected {len(selected_features)} features")

        return result

    def select_by_mutual_information(
        self,
        X: pd.DataFrame,
        y: pd.Series,
        k: int = 10
    ) -> FeatureSelectionResult:
        """
        Select features using mutual information.
        Captures both linear and non-linear relationships.
        """
        if self.task_type == "classification":
            score_func = mutual_info_classif
        else:
            score_func = mutual_info_regression

        selector = SelectKBest(score_func=score_func, k=min(k, X.shape[1]))
        selector.fit(X, y)

        scores = dict(zip(X.columns, selector.scores_))
        selected_mask = selector.get_support()
        selected_features = X.columns[selected_mask].tolist()

        result = FeatureSelectionResult(
            selected_features=selected_features,
            feature_scores=scores,
            method=f"mutual_information_{self.task_type}"
        )

        self.selection_results['mutual_information'] = result
        logger.info(f"Mutual information selected {len(selected_features)} features")

        return result

    def select_by_model_importance(
        self,
        X: pd.DataFrame,
        y: pd.Series,
        threshold: float = 0.01
    ) -> FeatureSelectionResult:
        """
        Select features using Random Forest feature importance.

        Args:
            threshold: Minimum importance score (0-1)
        """
        if self.task_type == "classification":
            model = RandomForestClassifier(n_estimators=100, random_state=42, n_jobs=-1)
        else:
            model = RandomForestRegressor(n_estimators=100, random_state=42, n_jobs=-1)

        model.fit(X, y)

        # Get feature importances
        importances = dict(zip(X.columns, model.feature_importances_))

        # Select features above threshold
        selected_features = [
            feature for feature, importance in importances.items()
            if importance >= threshold
        ]

        result = FeatureSelectionResult(
            selected_features=selected_features,
            feature_scores=importances,
            method=f"random_forest_{self.task_type}",
            threshold=threshold
        )

        self.selection_results['model_importance'] = result
        logger.info(f"Model importance selected {len(selected_features)} features "
                   f"(threshold={threshold})")

        return result

    def select_by_rfe(
        self,
        X: pd.DataFrame,
        y: pd.Series,
        n_features: int = 10,
        step: int = 1
    ) -> FeatureSelectionResult:
        """
        Recursive Feature Elimination (RFE).
        Iteratively removes least important features.

        Args:
            n_features: Number of features to select
            step: Number of features to remove at each iteration
        """
        if self.task_type == "classification":
            estimator = RandomForestClassifier(n_estimators=50, random_state=42, n_jobs=-1)
        else:
            estimator = RandomForestRegressor(n_estimators=50, random_state=42, n_jobs=-1)

        rfe = RFE(estimator=estimator, n_features_to_select=n_features, step=step)
        rfe.fit(X, y)

        # Get ranking (1 = selected, higher = eliminated earlier)
        rankings = dict(zip(X.columns, rfe.ranking_))

        # Convert ranking to scores (inverse ranking)
        max_rank = max(rankings.values())
        scores = {
            feature: (max_rank - rank + 1) / max_rank
            for feature, rank in rankings.items()
        }

        # Get selected features
        selected_mask = rfe.get_support()
        selected_features = X.columns[selected_mask].tolist()

        result = FeatureSelectionResult(
            selected_features=selected_features,
            feature_scores=scores,
            method=f"rfe_{self.task_type}"
        )

        self.selection_results['rfe'] = result
        logger.info(f"RFE selected {len(selected_features)} features")

        return result

    def get_consensus_features(
        self,
        min_methods: int = 2
    ) -> List[str]:
        """
        Get features selected by at least min_methods different methods.
        Provides robust feature selection through consensus.
        """
        if not self.selection_results:
            logger.warning("No selection results available")
            return []

        # Count how many methods selected each feature
        feature_counts: Dict[str, int] = {}

        for result in self.selection_results.values():
            for feature in result.selected_features:
                feature_counts[feature] = feature_counts.get(feature, 0) + 1

        # Filter by minimum methods
        consensus_features = [
            feature for feature, count in feature_counts.items()
            if count >= min_methods
        ]

        logger.info(f"Consensus: {len(consensus_features)} features selected by "
                   f">={min_methods} methods")

        return consensus_features

    def get_feature_selection_report(self) -> pd.DataFrame:
        """Generate a report comparing all selection methods."""
        if not self.selection_results:
            return pd.DataFrame()

        # Create report dataframe
        all_features = set()
        for result in self.selection_results.values():
            all_features.update(result.feature_scores.keys())

        report_data = []
        for feature in sorted(all_features):
            row = {"feature": feature}

            for method, result in self.selection_results.items():
                row[f"{method}_score"] = result.feature_scores.get(feature, 0.0)
                row[f"{method}_selected"] = feature in result.selected_features

            # Count selections
            row["num_selections"] = sum(
                1 for result in self.selection_results.values()
                if feature in result.selected_features
            )

            report_data.append(row)

        df = pd.DataFrame(report_data)
        df = df.sort_values("num_selections", ascending=False)

        return df
\end{lstlisting}

\section{Feature Validation}

Selected features must be validated for stability, robustness, and production readiness.

\begin{lstlisting}[language=Python, caption={Feature Validation Framework}]
from sklearn.model_selection import cross_val_score, KFold
from sklearn.linear_model import LogisticRegression, Ridge
from typing import List, Dict
import warnings

@dataclass
class FeatureStabilityResult:
    """Results from feature stability analysis."""
    feature_name: str
    stability_score: float  # 0-1, higher is more stable
    cv_scores: List[float]
    mean_cv_score: float
    std_cv_score: float
    is_stable: bool  # True if std/mean < threshold

    def __str__(self) -> str:
        return (f"Feature '{self.feature_name}': "
                f"Stability={self.stability_score:.3f}, "
                f"CV={self.mean_cv_score:.3f} +/- {self.std_cv_score:.3f}")

class FeatureValidator:
    """
    Validate features for production readiness:
    - Cross-validation stability
    - Correlation with target
    - Redundancy detection
    - Production compatibility checks
    """

    def __init__(self, task_type: str = "classification", n_folds: int = 5):
        self.task_type = task_type
        self.n_folds = n_folds

    def validate_feature_stability(
        self,
        X: pd.DataFrame,
        y: pd.Series,
        features: Optional[List[str]] = None,
        stability_threshold: float = 0.2
    ) -> List[FeatureStabilityResult]:
        """
        Validate feature stability across cross-validation folds.

        A stable feature maintains consistent importance across different
        data subsets, indicating robustness.

        Args:
            stability_threshold: Max coefficient of variation (std/mean)
        """
        if features is None:
            features = X.columns.tolist()

        results = []
        kfold = KFold(n_splits=self.n_folds, shuffle=True, random_state=42)

        if self.task_type == "classification":
            base_model = LogisticRegression(max_iter=1000, random_state=42)
        else:
            base_model = Ridge(random_state=42)

        for feature in features:
            if feature not in X.columns:
                continue

            X_feature = X[[feature]].values

            # Get cross-validation scores
            with warnings.catch_warnings():
                warnings.simplefilter("ignore")
                cv_scores = cross_val_score(
                    base_model, X_feature, y,
                    cv=kfold,
                    scoring='accuracy' if self.task_type == 'classification' else 'r2',
                    n_jobs=-1
                )

            mean_score = np.mean(cv_scores)
            std_score = np.std(cv_scores)

            # Calculate stability (inverse of coefficient of variation)
            if mean_score != 0:
                cv_coefficient = std_score / abs(mean_score)
                stability_score = 1 / (1 + cv_coefficient)
            else:
                stability_score = 0.0

            is_stable = cv_coefficient < stability_threshold if mean_score != 0 else False

            results.append(FeatureStabilityResult(
                feature_name=feature,
                stability_score=stability_score,
                cv_scores=cv_scores.tolist(),
                mean_cv_score=mean_score,
                std_cv_score=std_score,
                is_stable=is_stable
            ))

        # Sort by stability score
        results.sort(key=lambda x: x.stability_score, reverse=True)

        stable_count = sum(1 for r in results if r.is_stable)
        logger.info(f"Feature stability: {stable_count}/{len(results)} features stable")

        return results

    def detect_redundant_features(
        self,
        X: pd.DataFrame,
        correlation_threshold: float = 0.95
    ) -> List[tuple]:
        """
        Detect highly correlated (redundant) feature pairs.

        Returns:
            List of (feature1, feature2, correlation) tuples
        """
        # Calculate correlation matrix
        corr_matrix = X.corr().abs()

        # Find feature pairs with correlation above threshold
        redundant_pairs = []

        for i in range(len(corr_matrix.columns)):
            for j in range(i + 1, len(corr_matrix.columns)):
                if corr_matrix.iloc[i, j] >= correlation_threshold:
                    redundant_pairs.append((
                        corr_matrix.columns[i],
                        corr_matrix.columns[j],
                        corr_matrix.iloc[i, j]
                    ))

        logger.info(f"Found {len(redundant_pairs)} redundant feature pairs "
                   f"(threshold={correlation_threshold})")

        return redundant_pairs

    def check_production_readiness(
        self,
        df: pd.DataFrame,
        features: List[str]
    ) -> Dict[str, List[str]]:
        """
        Check if features are ready for production deployment.

        Checks:
        - No NaN or Inf values
        - Reasonable value ranges
        - Consistent dtypes
        """
        issues = {
            "nan_features": [],
            "inf_features": [],
            "constant_features": [],
            "warnings": []
        }

        for feature in features:
            if feature not in df.columns:
                issues["warnings"].append(f"Feature '{feature}' not found")
                continue

            series = df[feature]

            # Check for NaN
            if series.isna().any():
                nan_pct = series.isna().sum() / len(series) * 100
                issues["nan_features"].append(f"{feature} ({nan_pct:.1f}% NaN)")

            # Check for Inf
            if pd.api.types.is_numeric_dtype(series):
                if np.isinf(series).any():
                    issues["inf_features"].append(feature)

                # Check for constant
                if series.nunique() == 1:
                    issues["constant_features"].append(feature)

        # Log summary
        total_issues = (len(issues["nan_features"]) +
                       len(issues["inf_features"]) +
                       len(issues["constant_features"]))

        if total_issues == 0:
            logger.info(f"All {len(features)} features are production-ready")
        else:
            logger.warning(f"Found {total_issues} production readiness issues")
            for issue_type, issue_list in issues.items():
                if issue_list:
                    logger.warning(f"{issue_type}: {issue_list}")

        return issues
\end{lstlisting}

\section{Production Feature Monitoring}

Features can drift in production due to changing data distributions, upstream pipeline changes, or real-world concept drift. Continuous monitoring is essential.

\begin{lstlisting}[language=Python, caption={Production Feature Monitoring with Drift Detection}]
from scipy.stats import ks_2samp, chi2_contingency
from datetime import datetime, timedelta
import sqlite3

@dataclass
class FeatureDriftAlert:
    """Alert for detected feature drift."""
    feature_name: str
    drift_score: float
    p_value: float
    test_method: str
    timestamp: datetime
    severity: str  # 'low', 'medium', 'high'
    reference_stats: Dict[str, float]
    current_stats: Dict[str, float]

    def __str__(self) -> str:
        return (f"DRIFT ALERT [{self.severity.upper()}]: {self.feature_name} - "
                f"Score={self.drift_score:.3f}, p={self.p_value:.4f} ({self.test_method})")

class FeatureMonitor:
    """
    Monitor features in production for drift and anomalies.

    Tracks:
    - Distribution drift (KS test for numerical, chi-squared for categorical)
    - Statistical moments (mean, std, skewness, kurtosis)
    - Value range changes
    - Missing value patterns
    """

    def __init__(self, db_path: Path, p_value_threshold: float = 0.05):
        """
        Args:
            db_path: Path to SQLite database for storing metrics
            p_value_threshold: P-value threshold for drift detection
        """
        self.db_path = db_path
        self.p_value_threshold = p_value_threshold
        self.reference_distributions: Dict[str, pd.Series] = {}
        self._init_database()

    def _init_database(self) -> None:
        """Initialize SQLite database schema."""
        conn = sqlite3.connect(self.db_path)
        cursor = conn.cursor()

        # Feature metrics table
        cursor.execute('''
            CREATE TABLE IF NOT EXISTS feature_metrics (
                id INTEGER PRIMARY KEY AUTOINCREMENT,
                feature_name TEXT NOT NULL,
                timestamp DATETIME NOT NULL,
                mean REAL,
                std REAL,
                min REAL,
                max REAL,
                missing_pct REAL,
                skewness REAL,
                kurtosis REAL
            )
        ''')

        # Drift alerts table
        cursor.execute('''
            CREATE TABLE IF NOT EXISTS drift_alerts (
                id INTEGER PRIMARY KEY AUTOINCREMENT,
                feature_name TEXT NOT NULL,
                timestamp DATETIME NOT NULL,
                drift_score REAL NOT NULL,
                p_value REAL NOT NULL,
                test_method TEXT NOT NULL,
                severity TEXT NOT NULL,
                reference_stats TEXT,
                current_stats TEXT
            )
        ''')

        # Create indices
        cursor.execute('''
            CREATE INDEX IF NOT EXISTS idx_feature_metrics_name_time
            ON feature_metrics(feature_name, timestamp)
        ''')

        cursor.execute('''
            CREATE INDEX IF NOT EXISTS idx_drift_alerts_name_time
            ON drift_alerts(feature_name, timestamp)
        ''')

        conn.commit()
        conn.close()

        logger.info(f"Initialized feature monitoring database: {self.db_path}")

    def set_reference_distribution(self, feature_name: str,
                                   reference_data: pd.Series) -> None:
        """Set reference distribution for a feature (baseline)."""
        self.reference_distributions[feature_name] = reference_data.copy()
        logger.info(f"Set reference distribution for '{feature_name}' "
                   f"(n={len(reference_data)})")

    def monitor_batch(self, df: pd.DataFrame,
                     timestamp: Optional[datetime] = None) -> List[FeatureDriftAlert]:
        """
        Monitor a batch of production data for drift.

        Args:
            df: Production data batch
            timestamp: Timestamp for this batch (default: now)

        Returns:
            List of drift alerts
        """
        if timestamp is None:
            timestamp = datetime.now()

        alerts = []

        for feature_name in df.columns:
            # Record metrics
            self._record_feature_metrics(df[feature_name], feature_name, timestamp)

            # Check for drift if reference exists
            if feature_name in self.reference_distributions:
                alert = self._check_drift(
                    reference=self.reference_distributions[feature_name],
                    current=df[feature_name],
                    feature_name=feature_name,
                    timestamp=timestamp
                )

                if alert:
                    alerts.append(alert)
                    self._record_drift_alert(alert)

        if alerts:
            logger.warning(f"Detected {len(alerts)} drift alerts")
            for alert in alerts:
                logger.warning(str(alert))
        else:
            logger.info(f"No drift detected in {len(df.columns)} features")

        return alerts

    def _record_feature_metrics(self, series: pd.Series,
                               feature_name: str, timestamp: datetime) -> None:
        """Record feature statistics to database."""
        conn = sqlite3.connect(self.db_path)
        cursor = conn.cursor()

        # Calculate statistics (only for numerical features)
        if pd.api.types.is_numeric_dtype(series):
            stats = {
                "mean": series.mean(),
                "std": series.std(),
                "min": series.min(),
                "max": series.max(),
                "missing_pct": series.isna().sum() / len(series) * 100,
                "skewness": stats.skew(series.dropna()) if len(series.dropna()) > 0 else None,
                "kurtosis": stats.kurtosis(series.dropna()) if len(series.dropna()) > 0 else None
            }
        else:
            stats = {
                "mean": None,
                "std": None,
                "min": None,
                "max": None,
                "missing_pct": series.isna().sum() / len(series) * 100,
                "skewness": None,
                "kurtosis": None
            }

        cursor.execute('''
            INSERT INTO feature_metrics
            (feature_name, timestamp, mean, std, min, max, missing_pct, skewness, kurtosis)
            VALUES (?, ?, ?, ?, ?, ?, ?, ?, ?)
        ''', (
            feature_name,
            timestamp.isoformat(),
            stats["mean"],
            stats["std"],
            stats["min"],
            stats["max"],
            stats["missing_pct"],
            stats["skewness"],
            stats["kurtosis"]
        ))

        conn.commit()
        conn.close()

    def _check_drift(self, reference: pd.Series, current: pd.Series,
                    feature_name: str, timestamp: datetime) -> Optional[FeatureDriftAlert]:
        """Check for distribution drift using statistical tests."""
        # Remove NaN values
        ref_clean = reference.dropna()
        curr_clean = current.dropna()

        if len(ref_clean) == 0 or len(curr_clean) == 0:
            return None

        # Choose test based on data type
        if pd.api.types.is_numeric_dtype(reference):
            # Kolmogorov-Smirnov test for numerical features
            statistic, p_value = ks_2samp(ref_clean, curr_clean)
            test_method = "ks_test"

            ref_stats = {
                "mean": float(ref_clean.mean()),
                "std": float(ref_clean.std())
            }
            curr_stats = {
                "mean": float(curr_clean.mean()),
                "std": float(curr_clean.std())
            }
        else:
            # Chi-squared test for categorical features
            # Create contingency table
            ref_counts = reference.value_counts()
            curr_counts = current.value_counts()

            # Align categories
            all_categories = set(ref_counts.index) | set(curr_counts.index)
            ref_aligned = [ref_counts.get(cat, 0) for cat in all_categories]
            curr_aligned = [curr_counts.get(cat, 0) for cat in all_categories]

            contingency_table = np.array([ref_aligned, curr_aligned])
            statistic, p_value, _, _ = chi2_contingency(contingency_table)
            test_method = "chi2_test"

            ref_stats = {"top_categories": ref_counts.head(5).to_dict()}
            curr_stats = {"top_categories": curr_counts.head(5).to_dict()}

        # Determine if drift detected
        if p_value < self.p_value_threshold:
            # Determine severity based on p-value
            if p_value < 0.001:
                severity = "high"
            elif p_value < 0.01:
                severity = "medium"
            else:
                severity = "low"

            return FeatureDriftAlert(
                feature_name=feature_name,
                drift_score=float(statistic),
                p_value=float(p_value),
                test_method=test_method,
                timestamp=timestamp,
                severity=severity,
                reference_stats=ref_stats,
                current_stats=curr_stats
            )

        return None

    def _record_drift_alert(self, alert: FeatureDriftAlert) -> None:
        """Record drift alert to database."""
        conn = sqlite3.connect(self.db_path)
        cursor = conn.cursor()

        cursor.execute('''
            INSERT INTO drift_alerts
            (feature_name, timestamp, drift_score, p_value, test_method,
             severity, reference_stats, current_stats)
            VALUES (?, ?, ?, ?, ?, ?, ?, ?)
        ''', (
            alert.feature_name,
            alert.timestamp.isoformat(),
            alert.drift_score,
            alert.p_value,
            alert.test_method,
            alert.severity,
            json.dumps(alert.reference_stats),
            json.dumps(alert.current_stats)
        ))

        conn.commit()
        conn.close()

    def get_drift_history(self, feature_name: str,
                         days: int = 30) -> pd.DataFrame:
        """Get drift alert history for a feature."""
        conn = sqlite3.connect(self.db_path)

        cutoff_date = datetime.now() - timedelta(days=days)

        query = '''
            SELECT * FROM drift_alerts
            WHERE feature_name = ? AND timestamp >= ?
            ORDER BY timestamp DESC
        '''

        df = pd.read_sql_query(query, conn, params=(feature_name, cutoff_date.isoformat()))
        conn.close()

        return df

    def get_metrics_history(self, feature_name: str,
                           days: int = 30) -> pd.DataFrame:
        """Get metrics history for a feature."""
        conn = sqlite3.connect(self.db_path)

        cutoff_date = datetime.now() - timedelta(days=days)

        query = '''
            SELECT * FROM feature_metrics
            WHERE feature_name = ? AND timestamp >= ?
            ORDER BY timestamp DESC
        '''

        df = pd.read_sql_query(query, conn, params=(feature_name, cutoff_date.isoformat()))
        conn.close()

        return df
\end{lstlisting}

\section{Real-World Scenario: Feature Engineering Impact}

\subsection{The TechVentures Recommendation Engine}

TechVentures, a fast-growing e-commerce platform, struggled with poor click-through rates (CTR) on their product recommendations. Their baseline model used only 5 simple features: user age, product price, category, time of day, and previous purchase count. CTR hovered at 2.1\%, well below the industry benchmark of 4-5\%.

\subsection{The Feature Engineering Initiative}

The data science team, led by Maya Chen, launched a systematic feature engineering initiative following the framework from this chapter.

\textbf{Week 1-2: Feature Discovery}

Maya's team implemented the FeatureEngineeringPipeline and created 47 new features:

\begin{itemize}
    \item \textbf{Temporal features}: Time since last purchase, hour-of-day cyclic encoding, day-of-week patterns, month seasonality
    \item \textbf{Behavioral features}: 7-day/30-day rolling purchase frequency, category affinity scores, price sensitivity (ratio features)
    \item \textbf{Contextual features}: Product popularity (frequency encoding), user-product category interaction features
    \item \textbf{Engagement features}: Session duration binned, pages viewed (log-transformed due to right skew)
\end{itemize}

\textbf{Week 3: Feature Selection}

Using the FeatureSelector with four methods (statistical tests, mutual information, Random Forest importance, and RFE), the team identified 18 consensus features that all methods ranked highly. These included:

\begin{itemize}
    \item Days since last purchase (ranked \#1 by 3/4 methods)
    \item Category affinity score (ranked \#2)
    \item Price ratio to user's average purchase
    \item 30-day rolling purchase frequency
    \item Hour-of-day cyclic features
\end{itemize}

\textbf{Week 4: Validation}

The FeatureValidator revealed stability issues with 3 features that showed high variance across cross-validation folds. These were removed. The remaining 15 features passed all production readiness checks.

\subsection{The Results}

After deploying the new model with engineered features:

\begin{itemize}
    \item \textbf{CTR improved from 2.1\% to 4.8\%} (129\% relative improvement)
    \item \textbf{Revenue per user increased by 34\%}
    \item \textbf{Model AUC improved from 0.72 to 0.86}
\end{itemize}

\subsection{Production Monitoring Saves the Day}

Two months post-deployment, the FeatureMonitor detected drift in the "days\_since\_last\_purchase" feature (p-value = 0.003, KS statistic = 0.21). Investigation revealed that a marketing campaign had significantly changed purchase frequency patterns.

The team retrained the model with updated reference distributions and prevented a potential 15\% drop in CTR that would have occurred if the drift had gone undetected.

\subsection{Key Lessons}

\begin{enumerate}
    \item \textbf{Systematic > Ad-hoc}: The structured pipeline prevented common pitfalls like data leakage and ensured reproducibility
    \item \textbf{Selection matters}: Of 47 created features, only 15 were stable and valuable. Without rigorous selection, model complexity would have increased with no benefit
    \item \textbf{Monitoring is essential}: Production drift is inevitable; automated monitoring enabled proactive response
    \item \textbf{Documentation pays off}: The FeatureMetadata system made it trivial to understand feature lineage when debugging issues
\end{enumerate}

\section{Industry Scenarios: Feature Engineering Failures and Successes}

\subsection{Scenario 1: The Feature Engineering Arms Race}

FinanceAI, a quantitative trading firm, fell into a classic trap: believing more features automatically meant better performance. Their initial model used 50 carefully engineered features and achieved 63\% accuracy on stock movement prediction.

\textbf{The Escalation}

The team started generating features aggressively:
\begin{itemize}
    \item \textbf{Week 1}: Added polynomial combinations of existing features (200 new features)
    \item \textbf{Week 2}: Created rolling windows at 12 different time intervals for each feature (600 more features)
    \item \textbf{Week 3}: Added interaction terms between all pairs (10,000+ features)
    \item \textbf{Week 4}: Generated automatic transformations (log, sqrt, square) for all numerical features (15,000+ total)
\end{itemize}

Training accuracy improved to 78\%, and the team celebrated. However, validation accuracy dropped to 54\% -- worse than the original model.

\textbf{The Problems}

\begin{enumerate}
    \item \textbf{Massive overfitting}: With 15,000 features and only 50,000 training samples, the model memorized noise
    \item \textbf{Training time explosion}: Model training went from 5 minutes to 8 hours
    \item \textbf{Production inference latency}: Feature computation at prediction time exceeded acceptable 100ms SLA, taking 2.3 seconds
    \item \textbf{Feature maintenance nightmare}: Nobody understood what most features meant or where they came from
    \item \textbf{Data leakage}: Several rolling window features accidentally included future information
\end{enumerate}

\textbf{The Solution}

Senior ML engineer Dr. James Park intervened:
\begin{itemize}
    \item Implemented rigorous feature selection using mutual information and stability analysis
    \item Applied L1 regularization to identify truly important features
    \item Reduced feature count to 85 features (vs. original 50)
    \item Implemented FeatureMetadata tracking for lineage and documentation
    \item Added automated leakage detection tests in CI/CD
\end{itemize}

\textbf{Results}: Validation accuracy improved to 67\% (4 percentage points better than baseline), training time reduced to 12 minutes, inference latency at 45ms. Most importantly, features were interpretable and maintainable.

\textbf{Key Lesson}: Feature quantity is not quality. Systematic selection and validation are essential. Every feature should have a hypothesis and be validated for stability and non-redundancy.

\subsection{Scenario 2: The Real-Time Serving Nightmare}

RetailStream built a sophisticated customer churn prediction model using batch-computed features with impressive offline metrics: 0.91 AUC, 85\% precision at 70\% recall. The team was confident as they moved to production real-time serving.

\textbf{The Training Setup}

Features were computed using Spark batch jobs running nightly:
\begin{itemize}
    \item 30-day rolling aggregations (purchase frequency, average order value, category diversity)
    \item User-to-cohort similarity scores (computed via matrix factorization on full user base)
    \item Relative features (user's metrics vs. cohort averages)
    \item Complex graph features (social network connectivity, influence scores)
\end{itemize}

\textbf{The Production Reality}

When deployed for real-time predictions, disasters struck:
\begin{enumerate}
    \item \textbf{Latency catastrophe}: Rolling aggregations required querying 30 days of transaction history per user. p99 latency: 4.7 seconds (SLA: 200ms)
    \item \textbf{Data inconsistency}: Batch features used different data sources than production databases, causing training/serving skew
    \item \textbf{Cold start failure}: New users had no historical data; features returned NULL, causing model crashes
    \item \textbf{Resource exhaustion}: Peak traffic (Black Friday) overwhelmed the database with feature computation queries
    \item \textbf{Staleness}: Batch features updated nightly; intraday user behavior changes were invisible
\end{enumerate}

Customer-facing applications timed out, causing \$2.3M in lost revenue over a three-day period before the system was rolled back.

\textbf{The Architectural Redesign}

The team rebuilt with production-first thinking:
\begin{itemize}
    \item \textbf{Feature Store with dual serving}: Batch features pre-computed and cached in low-latency store (Redis); streaming features computed in real-time from Kafka
    \item \textbf{Feature complexity tiers}:
    \begin{itemize}
        \item \textit{Tier 1 (real-time)}: Simple aggregations from last 24 hours, cached user profiles
        \item \textit{Tier 2 (near-real-time)}: Updated hourly, acceptable 1-hour staleness
        \item \textit{Tier 3 (batch)}: Complex features updated nightly, used only for non-critical paths
    \end{itemize}
    \item \textbf{Fallback mechanisms}: Default values for cold start scenarios; graceful degradation if feature computation exceeds latency budget
    \item \textbf{Training/serving parity}: Shared feature computation code between batch training and online serving
\end{itemize}

\textbf{Results}: p99 latency reduced to 87ms, zero production incidents over 6 months, model performance maintained at 0.89 AUC (slight drop acceptable for reliability).

\textbf{Key Lesson}: Design features with production constraints from day one. Latency, scalability, and consistency are not afterthoughts. The best feature is useless if it cannot be served reliably.

\subsection{Scenario 3: The Data Leakage Disaster}

MedPredict developed a hospital readmission prediction model with remarkable offline performance: 0.94 AUC. Hospitals were eager to adopt it. However, after deployment, the model's real-world accuracy plummeted to barely better than random chance.

\textbf{The Hidden Leakage}

After weeks of investigation, the team discovered multiple leakage sources:

\begin{enumerate}
    \item \textbf{Temporal leakage}: Features included "total medications prescribed" -- but this was computed from the ENTIRE hospital stay, including medications prescribed AFTER the readmission decision point
    \item \textbf{Lab result leakage}: "Average lab result stability" was calculated over a 7-day window that included days AFTER discharge
    \item \textbf{Implicit target leakage}: "Number of follow-up appointments scheduled" was highly predictive -- because patients likely to be readmitted were proactively scheduled for more follow-ups by doctors
    \item \textbf{Data preprocessing leakage}: Missing value imputation used mean from entire dataset, including future test set patients
    \item \textbf{Feature selection leakage}: Feature importance was calculated on the full dataset before train/test split
\end{enumerate}

The model had essentially learned to predict the past from the future -- useless in production where only historical data exists at prediction time.

\textbf{The Consequences}

\begin{itemize}
    \item Three hospitals had already integrated the model; all had to be notified and systems reverted
    \item Regulatory scrutiny: FDA raised questions about ML validation processes
    \item Reputation damage: Published case studies had to be retracted
    \item Six months of development time lost
\end{itemize}

\textbf{The Systematic Fix}

The team implemented rigorous anti-leakage protocols:

\begin{lstlisting}[language=Python, caption={Automated Temporal Leakage Detection}]
class TemporalLeakageDetector:
    """Detect potential temporal leakage in feature engineering."""

    def __init__(self, prediction_time_col: str,
                 event_time_col: str):
        self.prediction_time = prediction_time_col
        self.event_time = event_time_col

    def validate_features(self, df: pd.DataFrame,
                         feature_metadata: List[FeatureMetadata]) -> Dict[str, str]:
        """
        Validate that features only use data available before prediction time.

        Returns:
            Dictionary of {feature_name: violation_description}
        """
        violations = {}

        for feature in feature_metadata:
            # Check if feature uses future data
            if feature.scope == TransformationScope.AGGREGATE:
                # Validate aggregation windows
                if hasattr(feature, 'window_end'):
                    future_data = df[
                        df[feature.window_end] > df[self.prediction_time]
                    ]
                    if len(future_data) > 0:
                        violations[feature.name] = (
                            f"Aggregation window extends beyond prediction time "
                            f"for {len(future_data)} rows"
                        )

            # Check for target leakage in source columns
            suspicious_keywords = ['after', 'post', 'future', 'outcome',
                                  'result', 'followup', 'readmit']
            for col in feature.source_columns:
                if any(keyword in col.lower() for keyword in suspicious_keywords):
                    violations[feature.name] = (
                        f"Source column '{col}' contains suspicious "
                        f"keyword indicating potential leakage"
                    )

        return violations

    def test_feature_validity(self, X: pd.DataFrame, y: pd.Series,
                             feature_cols: List[str]) -> Dict[str, float]:
        """
        Test for leakage by checking if features predict target TOO well.
        Suspiciously high correlation may indicate leakage.
        """
        suspicious_features = {}

        for col in feature_cols:
            if col not in X.columns:
                continue

            # Calculate correlation with target
            if X[col].dtype in ['float64', 'int64']:
                corr = np.abs(X[col].corr(y))

                # Single feature with correlation > 0.9 is suspicious
                if corr > 0.9:
                    suspicious_features[col] = corr

        return suspicious_features
\end{lstlisting}

Additional safeguards:
\begin{itemize}
    \item \textbf{Strict temporal splits}: Test data only includes patients admitted AFTER all training data
    \item \textbf{Feature cutoff times}: All features explicitly tagged with knowledge cutoff timestamp
    \item \textbf{Code review protocol}: Two senior reviewers must approve all new features
    \item \textbf{Automated leakage tests}: CI/CD pipeline includes temporal validation checks
    \item \textbf{Documentation requirements}: Every feature must document its temporal validity
\end{itemize}

\textbf{Results}: The rebuilt model achieved 0.78 AUC (lower but honest), maintained performance in production, and passed regulatory audit.

\textbf{Key Lesson}: Data leakage is insidious and often invisible in offline metrics. Implement automated leakage detection, enforce strict temporal discipline, and always validate with forward-looking temporal splits.

\subsection{Scenario 4: The Feature Store Migration}

GlobalBank operated 47 different ML models across fraud detection, credit risk, customer segmentation, and recommendation systems. Each team independently engineered features, leading to massive duplication, inconsistency, and waste.

\textbf{The Chaos}

\begin{itemize}
    \item "Customer lifetime value" was computed 12 different ways by 12 teams
    \item "Days since last transaction" existed in 8 variants with subtle differences
    \item Shared features (e.g., "account age") were recomputed in each model's pipeline
    \item No versioning: Features changed without notice, breaking downstream models
    \item Total compute cost for redundant feature computation: \$380K/year
\end{itemize}

\textbf{The Migration Initiative}

The ML Platform team launched a centralized feature store migration:

\begin{lstlisting}[language=Python, caption={Enterprise Feature Store Architecture}]
class EnterpriseFeatureStore:
    """
    Production feature store with versioning, governance, and monitoring.
    Supports both batch training and real-time serving.
    """

    def __init__(self,
                 offline_store: str,  # S3, BigQuery, etc.
                 online_store: str):  # Redis, DynamoDB, etc.
        self.offline_store = offline_store
        self.online_store = online_store
        self.feature_registry: Dict[str, FeatureDefinition] = {}

    def register_feature_group(self,
                              name: str,
                              features: List[FeatureDefinition],
                              owner: str,
                              compute_mode: str = "batch") -> None:
        """
        Register a group of related features with governance metadata.

        Args:
            name: Feature group name (e.g., "customer_demographics")
            features: List of feature definitions
            owner: Team/individual responsible for maintaining
            compute_mode: "batch", "streaming", or "on-demand"
        """
        feature_group = FeatureGroup(
            name=name,
            features=features,
            owner=owner,
            compute_mode=compute_mode,
            created_at=datetime.now(),
            documentation_url=f"https://wiki.company.com/features/{name}"
        )

        # Validate no naming conflicts
        for feature in features:
            if feature.name in self.feature_registry:
                existing = self.feature_registry[feature.name]
                raise ValueError(
                    f"Feature '{feature.name}' already exists "
                    f"(owner: {existing.owner})"
                )

        # Register each feature
        for feature in features:
            self.feature_registry[feature.name] = feature
            logger.info(f"Registered feature: {feature.name} (group: {name})")

    def get_historical_features(self,
                               entity_df: pd.DataFrame,
                               features: List[str],
                               feature_version: str = "latest") -> pd.DataFrame:
        """
        Get point-in-time correct features for training.
        Ensures no data leakage by respecting event timestamps.
        """
        result = entity_df.copy()

        for feature_name in features:
            if feature_name not in self.feature_registry:
                raise ValueError(f"Feature '{feature_name}' not found in registry")

            feature_def = self.feature_registry[feature_name]

            # Retrieve from offline store with point-in-time join
            feature_values = self._point_in_time_join(
                entity_df=entity_df,
                feature_def=feature_def,
                version=feature_version
            )

            result[feature_name] = feature_values

        return result

    def get_online_features(self,
                           entity_ids: List[str],
                           features: List[str]) -> pd.DataFrame:
        """
        Get features for real-time serving (low latency).
        Retrieves from online store (Redis, DynamoDB, etc.)
        """
        # Query online store for low-latency retrieval
        feature_vectors = []

        for entity_id in entity_ids:
            entity_features = {}
            for feature_name in features:
                # Retrieve from online store
                cache_key = f"{entity_id}:{feature_name}"
                value = self._get_from_online_store(cache_key)
                entity_features[feature_name] = value

            feature_vectors.append(entity_features)

        return pd.DataFrame(feature_vectors)

    def materialize_features(self,
                            feature_group: str,
                            start_time: datetime,
                            end_time: datetime) -> None:
        """
        Compute and materialize features for a time range.
        Stores in both offline (training) and online (serving) stores.
        """
        logger.info(f"Materializing feature group '{feature_group}' "
                   f"from {start_time} to {end_time}")

        # Compute features (batch job)
        features_df = self._compute_feature_group(
            feature_group, start_time, end_time
        )

        # Write to offline store (for training)
        self._write_offline_store(feature_group, features_df)

        # Write to online store (for serving)
        self._write_online_store(feature_group, features_df)

        logger.info(f"Materialized {len(features_df)} rows for {feature_group}")
\end{lstlisting}

\textbf{The Migration Challenges}

\begin{enumerate}
    \item \textbf{Reconciling definitions}: "Customer lifetime value" had 12 implementations; teams had to agree on canonical version
    \item \textbf{Backward compatibility}: Couldn't break existing models; needed versioning and migration paths
    \item \textbf{Performance regression}: Some teams had optimized local pipelines; feature store added latency
    \item \textbf{Cultural resistance}: Teams reluctant to give up control and depend on shared infrastructure
    \item \textbf{Governance overhead}: Establishing ownership, documentation, and approval processes
\end{enumerate}

\textbf{The Phased Approach}

\begin{itemize}
    \item \textbf{Phase 1 (Months 1-3)}: Pilot with fraud detection team; prove value and iron out issues
    \item \textbf{Phase 2 (Months 4-6)}: Migrate high-value shared features (customer demographics, transaction aggregates)
    \item \textbf{Phase 3 (Months 7-12)}: Gradual migration of remaining teams; deprecate redundant pipelines
    \item \textbf{Phase 4 (Ongoing)}: Establish feature governance: approval process, documentation standards, monitoring
\end{itemize}

\textbf{Results After 12 Months}

\begin{itemize}
    \item 347 features registered in feature store (from 2,000+ redundant features)
    \item Compute cost reduced by 62\% (\$235K/year savings)
    \item Model development time reduced by 40\% (reusing existing features)
    \item Feature consistency across models improved data governance compliance
    \item Real-time serving latency: p95 = 23ms, p99 = 47ms
    \item 23 models successfully migrated; 24 more in progress
\end{itemize}

\textbf{Key Lessons}

\begin{enumerate}
    \item \textbf{Start small}: Pilot with one team to prove value and learn
    \item \textbf{Incentivize adoption}: Show clear benefits (reduced development time, cost savings)
    \item \textbf{Maintain backward compatibility}: Support gradual migration, not forced big-bang
    \item \textbf{Governance is essential}: Without ownership and documentation, feature store becomes a dumping ground
    \item \textbf{Monitor performance}: Track serving latency, feature freshness, and compute costs
\end{enumerate}

\section{Feature Store Integration}

For organizations with multiple ML systems, a feature store provides centralized feature management, versioning, and serving.

\subsection{Feature Store Concepts}

\begin{lstlisting}[language=Python, caption={Feature Store Integration Pattern}]
from typing import Protocol
from datetime import datetime

class FeatureStore(Protocol):
    """Protocol for feature store implementations (e.g., Feast, Tecton)."""

    def register_features(self, feature_metadata: List[FeatureMetadata]) -> None:
        """Register features in the feature store."""
        ...

    def get_online_features(self, entity_ids: List[str],
                           feature_names: List[str]) -> pd.DataFrame:
        """Retrieve features for online serving (low latency)."""
        ...

    def get_historical_features(self, entity_df: pd.DataFrame,
                               feature_names: List[str]) -> pd.DataFrame:
        """Retrieve features for training (point-in-time correct)."""
        ...

@dataclass
class FeatureVersion:
    """Version information for features."""
    version_id: str
    pipeline_hash: str
    created_at: datetime
    features: List[FeatureMetadata]
    performance_metrics: Optional[Dict[str, float]] = None

    def is_compatible_with(self, other: 'FeatureVersion') -> bool:
        """Check if two feature versions are compatible."""
        self_features = set(f.name for f in self.features)
        other_features = set(f.name for f in other.features)
        return self_features == other_features

class FeatureVersionManager:
    """Manage feature versions for reproducibility."""

    def __init__(self, storage_path: Path):
        self.storage_path = storage_path
        self.storage_path.mkdir(parents=True, exist_ok=True)

    def save_version(self, pipeline: FeatureEngineeringPipeline,
                    performance_metrics: Optional[Dict[str, float]] = None) -> FeatureVersion:
        """Save a feature version."""
        version_id = datetime.now().strftime("%Y%m%d_%H%M%S")
        pipeline_hash = pipeline.compute_pipeline_hash()

        version = FeatureVersion(
            version_id=version_id,
            pipeline_hash=pipeline_hash,
            created_at=datetime.now(),
            features=list(pipeline.feature_metadata.values()),
            performance_metrics=performance_metrics
        )

        # Save to disk
        version_file = self.storage_path / f"feature_version_{version_id}.json"
        with open(version_file, 'w') as f:
            json.dump({
                "version_id": version.version_id,
                "pipeline_hash": version.pipeline_hash,
                "created_at": version.created_at.isoformat(),
                "features": [f.to_dict() for f in version.features],
                "performance_metrics": version.performance_metrics
            }, f, indent=2)

        logger.info(f"Saved feature version: {version_id}")
        return version

    def load_version(self, version_id: str) -> FeatureVersion:
        """Load a feature version."""
        version_file = self.storage_path / f"feature_version_{version_id}.json"

        with open(version_file, 'r') as f:
            data = json.load(f)

        return FeatureVersion(
            version_id=data["version_id"],
            pipeline_hash=data["pipeline_hash"],
            created_at=datetime.fromisoformat(data["created_at"]),
            features=[FeatureMetadata(**f) for f in data["features"]],
            performance_metrics=data.get("performance_metrics")
        )

    def list_versions(self) -> List[str]:
        """List all available versions."""
        version_files = self.storage_path.glob("feature_version_*.json")
        return sorted([f.stem.replace("feature_version_", "") for f in version_files])
\end{lstlisting}

\section{Enterprise Feature Management}

At scale, feature engineering becomes an organizational challenge requiring governance, cost optimization, and real-time serving capabilities.

\subsection{Production Feature Store Implementation}

\begin{lstlisting}[language=Python, caption={Production-Ready Feature Store with Real-Time Serving}]
from abc import ABC, abstractmethod
from typing import Optional, Dict, List
import redis
import pyarrow.parquet as pq
from concurrent.futures import ThreadPoolExecutor
import asyncio

class FeatureStore(ABC):
    """
    Abstract base class for feature stores supporting:
    - Online serving (low-latency, real-time)
    - Offline serving (batch, point-in-time correct)
    - Feature versioning and lineage
    - Consistency guarantees
    """

    @abstractmethod
    def write_features(self, entity_key: str, features: Dict[str, Any],
                      timestamp: datetime, feature_view: str) -> None:
        """Write features to both online and offline stores."""
        pass

    @abstractmethod
    def get_online_features(self, entity_keys: List[str],
                           feature_names: List[str]) -> pd.DataFrame:
        """Retrieve features for real-time inference (< 10ms)."""
        pass

    @abstractmethod
    def get_historical_features(self, entity_df: pd.DataFrame,
                               feature_names: List[str],
                               timestamp_column: str) -> pd.DataFrame:
        """Retrieve point-in-time correct features for training."""
        pass


class RedisParquetFeatureStore(FeatureStore):
    """
    Feature store implementation using:
    - Redis for online serving (low latency)
    - Parquet files for offline serving (historical accuracy)
    - Dual-write for consistency
    """

    def __init__(self, redis_host: str, redis_port: int,
                 offline_storage_path: Path,
                 ttl_seconds: int = 86400 * 30):  # 30 days default
        """
        Args:
            redis_host: Redis server host
            redis_port: Redis server port
            offline_storage_path: Path for Parquet files
            ttl_seconds: TTL for online features
        """
        self.redis_client = redis.Redis(
            host=redis_host,
            port=redis_port,
            decode_responses=False  # Store binary data
        )
        self.offline_storage_path = offline_storage_path
        self.offline_storage_path.mkdir(parents=True, exist_ok=True)
        self.ttl_seconds = ttl_seconds
        self.executor = ThreadPoolExecutor(max_workers=10)

    def write_features(self, entity_key: str, features: Dict[str, Any],
                      timestamp: datetime, feature_view: str) -> None:
        """
        Dual-write to online (Redis) and offline (Parquet) stores.

        Ensures eventual consistency between serving paths.
        """
        # Write to online store (Redis)
        redis_key = f"{feature_view}:{entity_key}"
        feature_data = {
            'timestamp': timestamp.isoformat(),
            **features
        }

        # Serialize to JSON and store
        import json
        self.redis_client.set(
            redis_key,
            json.dumps(feature_data),
            ex=self.ttl_seconds
        )

        # Async write to offline store
        self.executor.submit(
            self._write_to_offline_store,
            entity_key, features, timestamp, feature_view
        )

        logger.debug(f"Wrote features for {entity_key} to {feature_view}")

    def _write_to_offline_store(self, entity_key: str, features: Dict[str, Any],
                                timestamp: datetime, feature_view: str) -> None:
        """Append features to Parquet file for offline storage."""
        # Create DataFrame row
        row_data = {
            'entity_key': entity_key,
            'timestamp': timestamp,
            **features
        }
        df_row = pd.DataFrame([row_data])

        # Partition by date for efficient retrieval
        date_partition = timestamp.strftime("%Y-%m-%d")
        partition_path = (
            self.offline_storage_path / feature_view / date_partition
        )
        partition_path.mkdir(parents=True, exist_ok=True)

        # Append to Parquet file
        parquet_file = partition_path / f"{entity_key[:2]}.parquet"

        if parquet_file.exists():
            # Append to existing file
            existing_df = pd.read_parquet(parquet_file)
            combined_df = pd.concat([existing_df, df_row], ignore_index=True)
            combined_df.to_parquet(parquet_file, index=False)
        else:
            # Create new file
            df_row.to_parquet(parquet_file, index=False)

    def get_online_features(self, entity_keys: List[str],
                           feature_names: List[str],
                           feature_view: str = "default") -> pd.DataFrame:
        """
        Retrieve features from Redis for real-time serving.

        Optimized for low latency (<10ms for 100 entities).
        """
        import json

        # Batch retrieve from Redis using pipeline
        pipeline = self.redis_client.pipeline()
        for entity_key in entity_keys:
            redis_key = f"{feature_view}:{entity_key}"
            pipeline.get(redis_key)

        results = pipeline.execute()

        # Parse results
        rows = []
        for entity_key, redis_data in zip(entity_keys, results):
            if redis_data is None:
                # Feature not found, use nulls
                row = {'entity_key': entity_key}
                row.update({feat: None for feat in feature_names})
            else:
                feature_data = json.loads(redis_data)
                row = {'entity_key': entity_key}
                row.update({
                    feat: feature_data.get(feat)
                    for feat in feature_names
                })

            rows.append(row)

        return pd.DataFrame(rows)

    def get_historical_features(self, entity_df: pd.DataFrame,
                               feature_names: List[str],
                               timestamp_column: str,
                               feature_view: str = "default") -> pd.DataFrame:
        """
        Retrieve point-in-time correct features for training.

        Ensures no data leakage by only using features available
        at the specified timestamp.
        """
        # Determine date range
        min_date = entity_df[timestamp_column].min()
        max_date = entity_df[timestamp_column].max()

        # Load relevant Parquet partitions
        date_range = pd.date_range(start=min_date, end=max_date, freq='D')

        all_features = []
        for date in date_range:
            date_str = date.strftime("%Y-%m-%d")
            partition_path = (
                self.offline_storage_path / feature_view / date_str
            )

            if not partition_path.exists():
                continue

            # Read all Parquet files in partition
            for parquet_file in partition_path.glob("*.parquet"):
                df_partition = pd.read_parquet(parquet_file)
                all_features.append(df_partition)

        if not all_features:
            logger.warning(f"No historical features found for {feature_view}")
            return entity_df

        # Combine all partitions
        features_df = pd.concat(all_features, ignore_index=True)

        # Point-in-time join: for each entity at each timestamp,
        # get the most recent feature values before that timestamp
        result_rows = []

        for _, row in entity_df.iterrows():
            entity_key = row['entity_key']
            timestamp = row[timestamp_column]

            # Filter features for this entity before timestamp
            entity_features = features_df[
                (features_df['entity_key'] == entity_key) &
                (features_df['timestamp'] <= timestamp)
            ]

            if len(entity_features) == 0:
                # No historical features available
                feature_row = {feat: None for feat in feature_names}
            else:
                # Get most recent feature values
                latest = entity_features.sort_values('timestamp').iloc[-1]
                feature_row = {feat: latest.get(feat) for feat in feature_names}

            result_rows.append({**row.to_dict(), **feature_row})

        return pd.DataFrame(result_rows)


class FeatureGovernance:
    """
    Feature governance framework tracking:
    - Feature ownership
    - Documentation and lineage
    - Access controls
    - Data quality SLAs
    """

    def __init__(self, governance_db_path: Path):
        self.db_path = governance_db_path
        self._init_database()

    def _init_database(self) -> None:
        """Initialize governance database."""
        conn = sqlite3.connect(self.db_path)
        cursor = conn.cursor()

        # Feature registry
        cursor.execute('''
            CREATE TABLE IF NOT EXISTS feature_registry (
                feature_id TEXT PRIMARY KEY,
                feature_name TEXT NOT NULL,
                feature_view TEXT NOT NULL,
                owner TEXT NOT NULL,
                description TEXT,
                created_at DATETIME NOT NULL,
                last_updated DATETIME,
                status TEXT DEFAULT 'active',
                sla_freshness_minutes INTEGER,
                sla_completeness_pct REAL
            )
        ''')

        # Access log
        cursor.execute('''
            CREATE TABLE IF NOT EXISTS feature_access_log (
                id INTEGER PRIMARY KEY AUTOINCREMENT,
                feature_id TEXT NOT NULL,
                accessed_by TEXT NOT NULL,
                access_type TEXT NOT NULL,
                timestamp DATETIME NOT NULL,
                row_count INTEGER,
                FOREIGN KEY(feature_id) REFERENCES feature_registry(feature_id)
            )
        ''')

        # Data quality metrics
        cursor.execute('''
            CREATE TABLE IF NOT EXISTS feature_quality_metrics (
                id INTEGER PRIMARY KEY AUTOINCREMENT,
                feature_id TEXT NOT NULL,
                timestamp DATETIME NOT NULL,
                completeness_pct REAL,
                uniqueness_pct REAL,
                validity_pct REAL,
                freshness_minutes INTEGER,
                FOREIGN KEY(feature_id) REFERENCES feature_registry(feature_id)
            )
        ''')

        conn.commit()
        conn.close()

    def register_feature(self, feature_name: str, feature_view: str,
                        owner: str, description: str,
                        sla_freshness_minutes: Optional[int] = None,
                        sla_completeness_pct: float = 95.0) -> str:
        """Register a new feature with governance metadata."""
        feature_id = hashlib.sha256(
            f"{feature_view}:{feature_name}".encode()
        ).hexdigest()[:16]

        conn = sqlite3.connect(self.db_path)
        cursor = conn.cursor()

        cursor.execute('''
            INSERT OR REPLACE INTO feature_registry
            (feature_id, feature_name, feature_view, owner, description,
             created_at, last_updated, sla_freshness_minutes, sla_completeness_pct)
            VALUES (?, ?, ?, ?, ?, ?, ?, ?, ?)
        ''', (
            feature_id, feature_name, feature_view, owner, description,
            datetime.now().isoformat(), datetime.now().isoformat(),
            sla_freshness_minutes, sla_completeness_pct
        ))

        conn.commit()
        conn.close()

        logger.info(f"Registered feature: {feature_name} (owner: {owner})")
        return feature_id

    def log_access(self, feature_id: str, accessed_by: str,
                  access_type: str, row_count: int = 0) -> None:
        """Log feature access for audit trail."""
        conn = sqlite3.connect(self.db_path)
        cursor = conn.cursor()

        cursor.execute('''
            INSERT INTO feature_access_log
            (feature_id, accessed_by, access_type, timestamp, row_count)
            VALUES (?, ?, ?, ?, ?)
        ''', (
            feature_id, accessed_by, access_type,
            datetime.now().isoformat(), row_count
        ))

        conn.commit()
        conn.close()

    def check_sla_compliance(self, feature_id: str) -> Dict[str, bool]:
        """Check if feature meets its SLAs."""
        conn = sqlite3.connect(self.db_path)

        # Get SLA requirements
        sla_query = '''
            SELECT sla_freshness_minutes, sla_completeness_pct
            FROM feature_registry WHERE feature_id = ?
        '''
        sla_df = pd.read_sql_query(sla_query, conn, params=(feature_id,))

        if len(sla_df) == 0:
            return {"error": "Feature not found"}

        sla_freshness = sla_df.iloc[0]['sla_freshness_minutes']
        sla_completeness = sla_df.iloc[0]['sla_completeness_pct']

        # Get latest quality metrics
        metrics_query = '''
            SELECT * FROM feature_quality_metrics
            WHERE feature_id = ?
            ORDER BY timestamp DESC LIMIT 1
        '''
        metrics_df = pd.read_sql_query(metrics_query, conn, params=(feature_id,))

        conn.close()

        if len(metrics_df) == 0:
            return {"compliance": False, "reason": "No metrics available"}

        latest_metrics = metrics_df.iloc[0]

        # Check compliance
        freshness_ok = (
            sla_freshness is None or
            latest_metrics['freshness_minutes'] <= sla_freshness
        )
        completeness_ok = latest_metrics['completeness_pct'] >= sla_completeness

        return {
            "compliance": freshness_ok and completeness_ok,
            "freshness_ok": freshness_ok,
            "completeness_ok": completeness_ok,
            "metrics": latest_metrics.to_dict()
        }
\end{lstlisting}

\subsection{Feature Cost Optimization}

Feature computation and storage have real costs. This class optimizes the cost-performance trade-off.

\begin{lstlisting}[language=Python, caption={Feature Cost-Performance Optimization}]
from dataclasses import dataclass
from typing import List, Dict
import time

@dataclass
class FeatureCostMetrics:
    """Cost and performance metrics for a feature."""
    feature_name: str
    compute_time_ms: float
    storage_bytes: int
    query_time_ms: float
    importance_score: float
    cost_per_prediction: float  # Estimated cost
    efficiency_score: float  # Importance / cost ratio

class FeatureOptimizer:
    """
    Optimize feature set for cost-performance trade-offs:
    - Identify redundant expensive features
    - Balance compute cost vs. model performance
    - Recommend feature caching strategies
    """

    def __init__(self, cost_per_cpu_hour: float = 0.05,
                 cost_per_gb_month: float = 0.02):
        """
        Args:
            cost_per_cpu_hour: Compute cost per CPU hour
            cost_per_gb_month: Storage cost per GB per month
        """
        self.cost_per_cpu_hour = cost_per_cpu_hour
        self.cost_per_gb_month = cost_per_gb_month
        self.feature_metrics: Dict[str, FeatureCostMetrics] = {}

    def profile_features(self, feature_pipeline: FeatureEngineeringPipeline,
                        X_sample: pd.DataFrame, y_sample: pd.Series,
                        n_runs: int = 10) -> List[FeatureCostMetrics]:
        """
        Profile computational cost of each feature.

        Measures:
        - Compute time (transformation)
        - Storage size
        - Query time (retrieval)
        - Feature importance
        """
        metrics = []

        for step in feature_pipeline.steps:
            for metadata in step.metadata:
                feature_name = metadata.name

                # Measure compute time
                compute_times = []
                for _ in range(n_runs):
                    start = time.perf_counter()
                    step.execute(X_sample)
                    compute_times.append((time.perf_counter() - start) * 1000)

                compute_time_ms = np.mean(compute_times)

                # Measure storage size
                if feature_name in X_sample.columns:
                    storage_bytes = X_sample[feature_name].memory_usage(deep=True)
                else:
                    storage_bytes = 0

                # Estimate query time (simplified)
                query_time_ms = storage_bytes / 1_000_000  # Rough estimate

                # Get importance score (from metadata or compute)
                importance_score = metadata.importance or 0.0

                # Calculate cost per prediction
                # Cost = (compute_time * cpu_cost) + (storage * storage_cost)
                compute_cost = (compute_time_ms / 1000 / 3600) * self.cost_per_cpu_hour
                storage_cost = (storage_bytes / 1e9 / 30) * self.cost_per_gb_month
                cost_per_prediction = compute_cost + storage_cost

                # Efficiency score
                efficiency_score = (
                    importance_score / cost_per_prediction
                    if cost_per_prediction > 0 else 0.0
                )

                metric = FeatureCostMetrics(
                    feature_name=feature_name,
                    compute_time_ms=compute_time_ms,
                    storage_bytes=storage_bytes,
                    query_time_ms=query_time_ms,
                    importance_score=importance_score,
                    cost_per_prediction=cost_per_prediction,
                    efficiency_score=efficiency_score
                )

                metrics.append(metric)
                self.feature_metrics[feature_name] = metric

        # Sort by efficiency score
        metrics.sort(key=lambda x: x.efficiency_score, reverse=True)

        return metrics

    def recommend_optimizations(self, metrics: List[FeatureCostMetrics],
                               performance_threshold: float = 0.95) -> Dict[str, List[str]]:
        """
        Recommend cost optimizations while maintaining performance.

        Args:
            performance_threshold: Fraction of total importance to retain (0-1)

        Returns:
            Dictionary with optimization recommendations
        """
        # Sort by importance
        sorted_metrics = sorted(metrics, key=lambda x: x.importance_score, reverse=True)

        total_importance = sum(m.importance_score for m in metrics)
        target_importance = total_importance * performance_threshold

        # Greedy selection: keep features until reaching target importance
        cumulative_importance = 0
        keep_features = []
        drop_features = []

        for metric in sorted_metrics:
            if cumulative_importance < target_importance:
                keep_features.append(metric.feature_name)
                cumulative_importance += metric.importance_score
            else:
                drop_features.append(metric.feature_name)

        # Identify expensive features to cache
        cache_candidates = [
            m.feature_name for m in metrics
            if m.compute_time_ms > 100 and m.feature_name in keep_features
        ]

        # Identify features for batch precomputation
        precompute_candidates = [
            m.feature_name for m in metrics
            if m.compute_time_ms > 50 and m.query_time_ms < 10
        ]

        recommendations = {
            "keep": keep_features,
            "drop": drop_features,
            "cache": cache_candidates,
            "precompute": precompute_candidates,
            "summary": {
                "features_kept": len(keep_features),
                "features_dropped": len(drop_features),
                "importance_retained": cumulative_importance / total_importance,
                "estimated_cost_savings": sum(
                    self.feature_metrics[f].cost_per_prediction
                    for f in drop_features
                )
            }
        }

        return recommendations

    def generate_cost_report(self) -> pd.DataFrame:
        """Generate detailed cost-performance report."""
        if not self.feature_metrics:
            return pd.DataFrame()

        rows = []
        for metric in self.feature_metrics.values():
            rows.append({
                'feature': metric.feature_name,
                'compute_ms': metric.compute_time_ms,
                'storage_mb': metric.storage_bytes / 1e6,
                'importance': metric.importance_score,
                'cost_per_pred': metric.cost_per_prediction,
                'efficiency': metric.efficiency_score
            })

        df = pd.DataFrame(rows)
        df = df.sort_values('efficiency', ascending=False)

        return df
\end{lstlisting}

\section{Mathematical Foundations for Feature Engineering}

Rigorous statistical and information-theoretic foundations ensure features are meaningful and statistically sound.

\subsection{Information Theory Metrics}

\begin{lstlisting}[language=Python, caption={Information Theory for Feature Selection}]
from scipy.stats import entropy
from sklearn.metrics import mutual_info_score, normalized_mutual_info_score

class InformationTheoryFeatureSelector:
    """
    Feature selection using information theory metrics:
    - Mutual information with target
    - Conditional mutual information
    - Information gain ratio
    - Redundancy analysis
    """

    def __init__(self, n_bins: int = 10):
        """
        Args:
            n_bins: Number of bins for discretizing continuous variables
        """
        self.n_bins = n_bins
        self.feature_mi_scores: Dict[str, float] = {}
        self.redundancy_matrix: Optional[pd.DataFrame] = None

    def compute_mutual_information(self, X: pd.DataFrame, y: pd.Series,
                                   features: List[str] = None) -> Dict[str, float]:
        """
        Compute mutual information between each feature and target.

        MI(X;Y) = H(X) + H(Y) - H(X,Y)

        Measures reduction in uncertainty about Y when X is known.
        """
        if features is None:
            features = X.columns.tolist()

        mi_scores = {}

        for feature in features:
            if feature not in X.columns:
                continue

            x = X[feature]

            # Discretize if continuous
            if pd.api.types.is_numeric_dtype(x) and x.nunique() > self.n_bins:
                x_discrete = pd.qcut(x, q=self.n_bins, labels=False,
                                    duplicates='drop')
            else:
                x_discrete = x

            # Compute mutual information
            mi = mutual_info_score(x_discrete.fillna(-1), y)
            mi_scores[feature] = mi

        self.feature_mi_scores = mi_scores

        logger.info(f"Computed MI for {len(mi_scores)} features")
        return mi_scores

    def compute_redundancy(self, X: pd.DataFrame,
                          features: List[str] = None) -> pd.DataFrame:
        """
        Compute pairwise feature redundancy using mutual information.

        High MI between features indicates redundancy.
        """
        if features is None:
            features = X.columns.tolist()

        n = len(features)
        redundancy_matrix = np.zeros((n, n))

        for i, feat1 in enumerate(features):
            for j, feat2 in enumerate(features):
                if i >= j:
                    continue

                x1 = X[feat1]
                x2 = X[feat2]

                # Discretize if needed
                if pd.api.types.is_numeric_dtype(x1) and x1.nunique() > self.n_bins:
                    x1 = pd.qcut(x1, q=self.n_bins, labels=False, duplicates='drop')

                if pd.api.types.is_numeric_dtype(x2) and x2.nunique() > self.n_bins:
                    x2 = pd.qcut(x2, q=self.n_bins, labels=False, duplicates='drop')

                # Normalized MI (0-1)
                nmi = normalized_mutual_info_score(
                    x1.fillna(-1),
                    x2.fillna(-1)
                )

                redundancy_matrix[i, j] = nmi
                redundancy_matrix[j, i] = nmi

        self.redundancy_matrix = pd.DataFrame(
            redundancy_matrix,
            index=features,
            columns=features
        )

        return self.redundancy_matrix

    def select_mrmr_features(self, X: pd.DataFrame, y: pd.Series,
                            k: int = 10) -> List[str]:
        """
        Select features using mRMR (minimum Redundancy Maximum Relevance).

        Balances:
        - Relevance: High MI with target
        - Redundancy: Low MI with already selected features
        """
        if not self.feature_mi_scores:
            self.compute_mutual_information(X, y)

        if self.redundancy_matrix is None:
            self.compute_redundancy(X)

        selected_features = []
        candidate_features = list(self.feature_mi_scores.keys())

        # Start with feature with highest MI
        first_feature = max(candidate_features,
                           key=lambda f: self.feature_mi_scores[f])
        selected_features.append(first_feature)
        candidate_features.remove(first_feature)

        # Iteratively add features
        while len(selected_features) < k and candidate_features:
            mrmr_scores = {}

            for candidate in candidate_features:
                # Relevance: MI with target
                relevance = self.feature_mi_scores[candidate]

                # Redundancy: average MI with selected features
                redundancies = [
                    self.redundancy_matrix.loc[candidate, selected]
                    for selected in selected_features
                ]
                redundancy = np.mean(redundancies) if redundancies else 0

                # mRMR score
                mrmr_scores[candidate] = relevance - redundancy

            # Select feature with highest mRMR score
            best_candidate = max(mrmr_scores.items(), key=lambda x: x[1])[0]
            selected_features.append(best_candidate)
            candidate_features.remove(best_candidate)

        logger.info(f"Selected {len(selected_features)} features using mRMR")
        return selected_features

    def compute_conditional_mi(self, X: pd.DataFrame, y: pd.Series,
                              feature: str, condition_feature: str) -> float:
        """
        Compute conditional mutual information: MI(feature; y | condition).

        Measures information about y provided by feature,
        given that condition_feature is already known.
        """
        # This is a simplified implementation
        # Full implementation would require proper discretization and counting

        x = X[feature].fillna(-1)
        z = X[condition_feature].fillna(-1)

        # Discretize
        if pd.api.types.is_numeric_dtype(x) and x.nunique() > self.n_bins:
            x = pd.qcut(x, q=self.n_bins, labels=False, duplicates='drop')

        if pd.api.types.is_numeric_dtype(z) and z.nunique() > self.n_bins:
            z = pd.qcut(z, q=self.n_bins, labels=False, duplicates='drop')

        # CMI(X;Y|Z) = MI(X;Y,Z) - MI(X;Z)
        # Simplified: compute for each value of Z and average
        cmi_values = []

        for z_val in z.unique():
            mask = (z == z_val)
            if mask.sum() < 10:  # Skip small groups
                continue

            x_subset = x[mask]
            y_subset = y[mask]

            mi = mutual_info_score(x_subset, y_subset)
            cmi_values.append(mi)

        return np.mean(cmi_values) if cmi_values else 0.0
\end{lstlisting}

\subsection{Causal Feature Selection}

\begin{lstlisting}[language=Python, caption={Causal Feature Selection with Confounding Adjustment}]
from scipy.stats import pearsonr, spearmanr
import networkx as nx

class CausalFeatureSelector:
    """
    Select features based on causal relationships rather than just correlation.

    Addresses:
    - Confounding variables
    - Spurious correlations
    - Causal discovery
    """

    def __init__(self):
        self.causal_graph: Optional[nx.DiGraph] = None
        self.adjustment_sets: Dict[str, List[str]] = {}

    def discover_causal_structure(self, X: pd.DataFrame,
                                  y: pd.Series,
                                  alpha: float = 0.05) -> nx.DiGraph:
        """
        Discover causal structure using PC algorithm (simplified).

        Returns directed acyclic graph representing causal relationships.
        """
        from pgmpy.estimators import PC

        # Combine X and y
        data = X.copy()
        data['target'] = y

        # Run PC algorithm for causal discovery
        pc = PC(data)
        causal_graph = pc.estimate(significance_level=alpha)

        self.causal_graph = causal_graph

        logger.info(f"Discovered causal graph with {len(causal_graph.nodes())} nodes "
                   f"and {len(causal_graph.edges())} edges")

        return causal_graph

    def identify_confounders(self, feature: str, target: str = 'target') -> List[str]:
        """
        Identify confounding variables for feature -> target relationship.

        A confounder causes both the feature and the target.
        """
        if self.causal_graph is None:
            raise ValueError("Call discover_causal_structure first")

        confounders = []

        for node in self.causal_graph.nodes():
            if node == feature or node == target:
                continue

            # Check if node causes both feature and target
            causes_feature = self.causal_graph.has_edge(node, feature)
            causes_target = self.causal_graph.has_edge(node, target)

            if causes_feature and causes_target:
                confounders.append(node)

        return confounders

    def compute_adjusted_correlation(self, X: pd.DataFrame, y: pd.Series,
                                    feature: str,
                                    confounders: List[str] = None) -> float:
        """
        Compute correlation between feature and target,
        adjusted for confounding variables.

        Uses partial correlation to control for confounders.
        """
        if confounders is None or len(confounders) == 0:
            # No adjustment needed
            return abs(pearsonr(X[feature], y)[0])

        # Perform linear regression to remove confounding effects
        from sklearn.linear_model import LinearRegression

        # Regress feature on confounders
        model_feature = LinearRegression()
        model_feature.fit(X[confounders], X[feature])
        residual_feature = X[feature] - model_feature.predict(X[confounders])

        # Regress target on confounders
        model_target = LinearRegression()
        model_target.fit(X[confounders], y)
        residual_target = y - model_target.predict(X[confounders])

        # Partial correlation = correlation of residuals
        partial_corr = abs(pearsonr(residual_feature, residual_target)[0])

        return partial_corr

    def select_causal_features(self, X: pd.DataFrame, y: pd.Series,
                              threshold: float = 0.1) -> List[str]:
        """
        Select features based on adjusted causal effect on target.

        Returns features with significant causal relationship to target.
        """
        if self.causal_graph is None:
            self.discover_causal_structure(X, y)

        causal_features = []

        for feature in X.columns:
            # Identify confounders
            confounders = self.identify_confounders(feature)

            # Compute adjusted correlation
            adj_corr = self.compute_adjusted_correlation(
                X, y, feature, confounders
            )

            if adj_corr >= threshold:
                causal_features.append(feature)
                self.adjustment_sets[feature] = confounders

        logger.info(f"Selected {len(causal_features)} features "
                   f"with causal relationship (threshold={threshold})")

        return causal_features
\end{lstlisting}

\section{Advanced Feature Engineering Techniques}

This section covers cutting-edge feature engineering methods that go beyond traditional approaches, incorporating automated feature selection with regularization, deep feature synthesis, neural feature learning, multi-modal fusion, and adaptive online learning.

\subsection{Automated Feature Selection with Regularization Paths}

Regularization paths provide a systematic way to understand feature importance across different regularization strengths, enabling robust feature selection.

\begin{lstlisting}[language=Python, caption={Feature Selection Using Regularization Paths}]
from sklearn.linear_model import LassoCV, ElasticNetCV
from sklearn.preprocessing import StandardScaler
import matplotlib.pyplot as plt

class RegularizationPathSelector:
    """
    Use regularization paths to identify stable, important features.
    Analyzes feature coefficients across different regularization strengths.
    """

    def __init__(self, method: str = "lasso", n_alphas: int = 100):
        """
        Args:
            method: "lasso", "elasticnet", or "ridge"
            n_alphas: Number of regularization strengths to test
        """
        self.method = method
        self.n_alphas = n_alphas
        self.scaler = StandardScaler()
        self.coefficients_path: Dict[str, np.ndarray] = {}
        self.selected_features: List[str] = []

    def fit(self, X: pd.DataFrame, y: pd.Series,
           stability_threshold: float = 0.8) -> List[str]:
        """
        Fit regularization path and select stable features.

        Args:
            X: Feature matrix
            y: Target variable
            stability_threshold: Fraction of alphas where feature must be nonzero

        Returns:
            List of selected feature names
        """
        # Standardize features
        X_scaled = self.scaler.fit_transform(X)

        # Fit model with cross-validated regularization
        if self.method == "lasso":
            alphas = np.logspace(-4, 1, self.n_alphas)
            model = LassoCV(alphas=alphas, cv=5, random_state=42, n_jobs=-1)
        elif self.method == "elasticnet":
            alphas = np.logspace(-4, 1, self.n_alphas)
            model = ElasticNetCV(alphas=alphas, cv=5, random_state=42, n_jobs=-1)
        else:
            raise ValueError(f"Method {self.method} not supported")

        model.fit(X_scaled, y)

        # Get regularization path
        # For each alpha, train model and record coefficients
        feature_stability = {col: 0 for col in X.columns}
        self.coefficients_path = {col: [] for col in X.columns}

        for alpha in alphas:
            if self.method == "lasso":
                from sklearn.linear_model import Lasso
                model_alpha = Lasso(alpha=alpha, random_state=42)
            else:
                from sklearn.linear_model import ElasticNet
                model_alpha = ElasticNet(alpha=alpha, random_state=42)

            model_alpha.fit(X_scaled, y)

            # Record coefficients
            for idx, col in enumerate(X.columns):
                coef = model_alpha.coef_[idx]
                self.coefficients_path[col].append(coef)

                # Track if feature is active (non-zero)
                if abs(coef) > 1e-10:
                    feature_stability[col] += 1

        # Select features that are stable across regularization path
        for col, count in feature_stability.items():
            stability = count / len(alphas)
            if stability >= stability_threshold:
                self.selected_features.append(col)

        logger.info(f"Selected {len(self.selected_features)} stable features "
                   f"(stability >= {stability_threshold})")

        return self.selected_features

    def plot_regularization_path(self, top_n: int = 20) -> None:
        """Plot coefficient paths for top features."""
        if not self.coefficients_path:
            raise ValueError("Must fit() before plotting")

        # Get features with highest coefficient magnitude at optimal alpha
        feature_importance = {
            col: max(abs(np.array(coefs)))
            for col, coefs in self.coefficients_path.items()
        }
        top_features = sorted(
            feature_importance.items(),
            key=lambda x: x[1],
            reverse=True
        )[:top_n]

        plt.figure(figsize=(12, 6))
        alphas = np.logspace(-4, 1, self.n_alphas)

        for feature_name, _ in top_features:
            coef_path = self.coefficients_path[feature_name]
            plt.plot(alphas, coef_path, label=feature_name)

        plt.xscale('log')
        plt.xlabel('Regularization strength (alpha)')
        plt.ylabel('Coefficient value')
        plt.title(f'{self.method.capitalize()} Regularization Path')
        plt.legend(bbox_to_anchor=(1.05, 1), loc='upper left')
        plt.grid(True, alpha=0.3)
        plt.tight_layout()


class StabilitySelectionMethod:
    """
    Stability selection: Run feature selection on bootstrap samples
    and select features that appear frequently.
    """

    def __init__(self, base_estimator=None, n_bootstrap: int = 100,
                 threshold: float = 0.6, sample_fraction: float = 0.5):
        """
        Args:
            base_estimator: Feature selection method (default: LassoCV)
            n_bootstrap: Number of bootstrap iterations
            threshold: Minimum selection frequency to keep feature
            sample_fraction: Fraction of data to sample in each iteration
        """
        if base_estimator is None:
            base_estimator = LassoCV(cv=5, random_state=42)

        self.base_estimator = base_estimator
        self.n_bootstrap = n_bootstrap
        self.threshold = threshold
        self.sample_fraction = sample_fraction
        self.feature_scores: Dict[str, float] = {}

    def fit(self, X: pd.DataFrame, y: pd.Series) -> List[str]:
        """
        Perform stability selection via bootstrap sampling.

        Returns:
            Features selected in at least threshold fraction of iterations
        """
        feature_selection_count = {col: 0 for col in X.columns}
        n_samples = int(len(X) * self.sample_fraction)

        for iteration in range(self.n_bootstrap):
            # Bootstrap sample
            indices = np.random.choice(len(X), size=n_samples, replace=False)
            X_boot = X.iloc[indices]
            y_boot = y.iloc[indices]

            # Fit feature selector
            self.base_estimator.fit(X_boot, y_boot)

            # Get selected features (non-zero coefficients)
            if hasattr(self.base_estimator, 'coef_'):
                selected_mask = np.abs(self.base_estimator.coef_) > 1e-10
                selected_features = X.columns[selected_mask].tolist()
            elif hasattr(self.base_estimator, 'feature_importances_'):
                # For tree-based methods
                threshold_val = np.median(self.base_estimator.feature_importances_)
                selected_mask = self.base_estimator.feature_importances_ > threshold_val
                selected_features = X.columns[selected_mask].tolist()
            else:
                continue

            # Increment counts
            for feature in selected_features:
                feature_selection_count[feature] += 1

        # Calculate stability scores
        self.feature_scores = {
            col: count / self.n_bootstrap
            for col, count in feature_selection_count.items()
        }

        # Select stable features
        stable_features = [
            col for col, score in self.feature_scores.items()
            if score >= self.threshold
        ]

        logger.info(f"Stability selection: {len(stable_features)} features "
                   f"selected in >={self.threshold*100}% of iterations")

        return stable_features

    def get_stability_scores(self) -> pd.Series:
        """Get stability score for each feature."""
        return pd.Series(self.feature_scores).sort_values(ascending=False)
\end{lstlisting}

\subsection{Deep Feature Synthesis with Automated Primitive Composition}

Deep feature synthesis automatically generates features by composing transformation primitives, discovering complex feature interactions.

\begin{lstlisting}[language=Python, caption={Deep Feature Synthesis Implementation}]
from typing import Callable, List, Dict
from itertools import combinations
import featuretools as ft

class DeepFeatureSynthesis:
    """
    Automated deep feature synthesis using primitive composition.
    Generates features by stacking transformation primitives.
    """

    def __init__(self, max_depth: int = 2,
                 primitives: List[str] = None):
        """
        Args:
            max_depth: Maximum depth of feature stacking
            primitives: List of primitives to use (default: common set)
        """
        self.max_depth = max_depth

        if primitives is None:
            # Default primitive set
            self.primitives = [
                "add", "subtract", "multiply", "divide",
                "log", "sqrt", "square", "abs",
                "mean", "sum", "std", "min", "max",
                "count", "mode"
            ]
        else:
            self.primitives = primitives

        self.synthesized_features: List[FeatureMetadata] = []

    def synthesize_features(self, df: pd.DataFrame,
                           entity_columns: Dict[str, str],
                           target_entity: str,
                           agg_primitives: List[str] = None,
                           trans_primitives: List[str] = None) -> pd.DataFrame:
        """
        Synthesize features using featuretools deep feature synthesis.

        Args:
            df: Input dataframe
            entity_columns: Dict mapping entity names to their ID columns
            target_entity: Entity to generate features for
            agg_primitives: Aggregation primitives (sum, mean, count, etc.)
            trans_primitives: Transform primitives (log, sqrt, etc.)

        Returns:
            DataFrame with original and synthesized features
        """
        if agg_primitives is None:
            agg_primitives = ["sum", "mean", "std", "count", "max", "min"]

        if trans_primitives is None:
            trans_primitives = ["divide", "multiply", "add", "subtract"]

        # Create entity set
        es = ft.EntitySet(id="data")

        # Add entities
        for entity_name, id_column in entity_columns.items():
            es = es.add_dataframe(
                dataframe_name=entity_name,
                dataframe=df,
                index=id_column
            )

        # Run deep feature synthesis
        feature_matrix, feature_defs = ft.dfs(
            entityset=es,
            target_dataframe_name=target_entity,
            agg_primitives=agg_primitives,
            trans_primitives=trans_primitives,
            max_depth=self.max_depth,
            verbose=True
        )

        # Create metadata for synthesized features
        for feature_def in feature_defs:
            metadata = FeatureMetadata(
                name=str(feature_def.get_name()),
                feature_type=FeatureType.NUMERICAL,
                source_columns=feature_def.get_feature_names(),
                transformation=str(feature_def),
                scope=TransformationScope.AGGREGATE,
                created_at=datetime.now(),
                version="1.0.0",
                description=f"Deep feature synthesis: {str(feature_def)}"
            )
            self.synthesized_features.append(metadata)

        logger.info(f"Synthesized {len(feature_defs)} features using DFS "
                   f"(max_depth={self.max_depth})")

        return feature_matrix

    def get_feature_lineage(self) -> pd.DataFrame:
        """Get lineage information for synthesized features."""
        lineage = []
        for feature in self.synthesized_features:
            lineage.append({
                'feature_name': feature.name,
                'source_columns': ', '.join(feature.source_columns),
                'transformation': feature.transformation,
                'depth': feature.transformation.count('(')
            })

        return pd.DataFrame(lineage)
\end{lstlisting}

\subsection{Feature Learning from Neural Networks}

Extract learned representations from neural networks as features, combining deep learning's representation power with traditional ML's interpretability.

\begin{lstlisting}[language=Python, caption={Neural Feature Learning with Interpretability}]
from sklearn.decomposition import PCA
from tensorflow import keras
from tensorflow.keras import layers
import shap

class NeuralFeatureLearner:
    """
    Learn features using neural networks and extract interpretable representations.
    """

    def __init__(self, encoding_dim: int = 32,
                 hidden_layers: List[int] = None):
        """
        Args:
            encoding_dim: Dimension of learned feature representation
            hidden_layers: Sizes of hidden layers (default: [128, 64])
        """
        self.encoding_dim = encoding_dim
        self.hidden_layers = hidden_layers or [128, 64]
        self.autoencoder = None
        self.encoder = None
        self.feature_importance: Optional[np.ndarray] = None

    def build_autoencoder(self, input_dim: int) -> None:
        """Build autoencoder architecture for feature learning."""
        # Encoder
        input_layer = keras.Input(shape=(input_dim,))
        encoded = input_layer

        for hidden_size in self.hidden_layers:
            encoded = layers.Dense(hidden_size, activation='relu')(encoded)
            encoded = layers.Dropout(0.2)(encoded)

        # Bottleneck (learned features)
        encoded = layers.Dense(self.encoding_dim, activation='relu',
                              name='bottleneck')(encoded)

        # Decoder
        decoded = encoded
        for hidden_size in reversed(self.hidden_layers):
            decoded = layers.Dense(hidden_size, activation='relu')(decoded)

        decoded = layers.Dense(input_dim, activation='linear')(decoded)

        # Models
        self.autoencoder = keras.Model(input_layer, decoded)
        self.encoder = keras.Model(input_layer, encoded)

        self.autoencoder.compile(optimizer='adam', loss='mse')

    def fit(self, X: pd.DataFrame, epochs: int = 50,
           batch_size: int = 256) -> None:
        """
        Train autoencoder to learn feature representations.
        """
        if self.autoencoder is None:
            self.build_autoencoder(X.shape[1])

        # Train autoencoder
        self.autoencoder.fit(
            X.values, X.values,
            epochs=epochs,
            batch_size=batch_size,
            validation_split=0.2,
            verbose=1
        )

        logger.info(f"Trained autoencoder with {self.encoding_dim}-dim encoding")

    def transform(self, X: pd.DataFrame) -> pd.DataFrame:
        """
        Transform data to learned feature representation.

        Returns:
            DataFrame with learned features (encoding_dim columns)
        """
        if self.encoder is None:
            raise ValueError("Must fit() before transform()")

        # Get encoded representation
        encoded = self.encoder.predict(X.values)

        # Create DataFrame with learned features
        feature_cols = [f"neural_feature_{i}" for i in range(self.encoding_dim)]
        result = pd.DataFrame(encoded, columns=feature_cols, index=X.index)

        return result

    def explain_features(self, X: pd.DataFrame,
                        original_feature_names: List[str]) -> pd.DataFrame:
        """
        Use SHAP to explain which original features contribute to learned features.

        Returns:
            DataFrame with feature importance for each learned feature
        """
        if self.encoder is None:
            raise ValueError("Must fit() before explaining")

        # Use SHAP to explain encoder
        explainer = shap.DeepExplainer(self.encoder, X.values[:100])
        shap_values = explainer.shap_values(X.values[:100])

        # Calculate average absolute SHAP value for each original feature
        # across all learned features
        importance = np.abs(shap_values).mean(axis=0)

        importance_df = pd.DataFrame({
            'original_feature': original_feature_names,
            'importance': importance.mean(axis=1)  # Average across learned features
        }).sort_values('importance', ascending=False)

        return importance_df
\end{lstlisting}

\subsection{Multi-Modal Feature Fusion}

Combine features from different data modalities (tabular, text, images) using attention mechanisms for unified representations.

\begin{lstlisting}[language=Python, caption={Multi-Modal Feature Fusion with Attention}]
class MultiModalFeatureFusion:
    """
    Fuse features from multiple modalities (tabular, text, images)
    using attention mechanisms.
    """

    def __init__(self, fusion_dim: int = 64):
        """
        Args:
            fusion_dim: Dimension of fused feature representation
        """
        self.fusion_dim = fusion_dim
        self.modality_encoders: Dict[str, keras.Model] = {}
        self.fusion_model: Optional[keras.Model] = None

    def add_tabular_encoder(self, input_dim: int) -> None:
        """Add encoder for tabular features."""
        inputs = keras.Input(shape=(input_dim,))
        x = layers.Dense(128, activation='relu')(inputs)
        x = layers.Dropout(0.3)(x)
        x = layers.Dense(self.fusion_dim, activation='relu')(x)

        self.modality_encoders['tabular'] = keras.Model(inputs, x)

    def add_text_encoder(self, vocab_size: int, max_length: int) -> None:
        """Add encoder for text features (embeddings + LSTM)."""
        inputs = keras.Input(shape=(max_length,))
        x = layers.Embedding(vocab_size, 128)(inputs)
        x = layers.LSTM(64, return_sequences=False)(x)
        x = layers.Dense(self.fusion_dim, activation='relu')(x)

        self.modality_encoders['text'] = keras.Model(inputs, x)

    def build_fusion_model(self, modalities: List[str]) -> None:
        """
        Build attention-based fusion model for specified modalities.

        Args:
            modalities: List of modality names to fuse (e.g., ['tabular', 'text'])
        """
        # Collect modality embeddings
        modality_inputs = {}
        modality_embeddings = []

        for modality in modalities:
            if modality not in self.modality_encoders:
                raise ValueError(f"Encoder for modality '{modality}' not found")

            encoder = self.modality_encoders[modality]
            modality_inputs[modality] = encoder.input
            embedding = encoder.output
            modality_embeddings.append(embedding)

        # Stack embeddings
        stacked = layers.concatenate(modality_embeddings)
        stacked = layers.Reshape((len(modalities), self.fusion_dim))(stacked)

        # Multi-head attention for fusion
        attention_output = layers.MultiHeadAttention(
            num_heads=4,
            key_dim=self.fusion_dim
        )(stacked, stacked)

        # Global average pooling across modalities
        fused = layers.GlobalAveragePooling1D()(attention_output)

        # Final fusion layer
        fused = layers.Dense(self.fusion_dim, activation='relu')(fused)

        self.fusion_model = keras.Model(
            inputs=list(modality_inputs.values()),
            outputs=fused
        )

        logger.info(f"Built multi-modal fusion model for {modalities}")

    def transform(self, data: Dict[str, np.ndarray]) -> np.ndarray:
        """
        Transform multi-modal data to fused features.

        Args:
            data: Dictionary mapping modality names to data arrays

        Returns:
            Fused feature matrix
        """
        if self.fusion_model is None:
            raise ValueError("Must build_fusion_model() first")

        # Prepare inputs in correct order
        inputs = [data[modality] for modality in self.modality_encoders.keys()]

        # Get fused representation
        fused_features = self.fusion_model.predict(inputs)

        return fused_features
\end{lstlisting}

\subsection{Online Feature Learning with Concept Drift Adaptation}

Adapt features dynamically in production as data distributions shift, maintaining model performance under concept drift.

\begin{lstlisting}[language=Python, caption={Adaptive Online Feature Learning}]
from river import drift, preprocessing, compose

class OnlineFeatureLearner:
    """
    Learn and adapt features online as new data arrives.
    Detects concept drift and re-learns features when needed.
    """

    def __init__(self, drift_detector: str = "adwin",
                 adaptation_threshold: float = 0.1):
        """
        Args:
            drift_detector: Drift detection method ("adwin", "ddm", "kswin")
            adaptation_threshold: Drift score threshold for re-learning
        """
        self.drift_detector_name = drift_detector
        self.adaptation_threshold = adaptation_threshold

        # Initialize drift detector
        if drift_detector == "adwin":
            self.drift_detector = drift.ADWIN()
        elif drift_detector == "ddm":
            self.drift_detector = drift.DDM()
        elif drift_detector == "kswin":
            self.drift_detector = drift.KSWIN()
        else:
            raise ValueError(f"Unknown drift detector: {drift_detector}")

        self.feature_transforms: Dict[str, Callable] = {}
        self.drift_detected_count = 0
        self.online_scaler = preprocessing.StandardScaler()

    def update(self, x: Dict, y: float, error: float) -> bool:
        """
        Update feature learner with new observation and prediction error.

        Args:
            x: Feature dictionary
            y: True target value
            error: Prediction error

        Returns:
            True if drift detected and adaptation triggered
        """
        # Update drift detector with error
        self.drift_detector.update(error)

        # Check for drift
        if self.drift_detector.drift_detected:
            self.drift_detected_count += 1
            logger.warning(f"Concept drift detected! (count: {self.drift_detected_count})")

            # Trigger feature adaptation
            self._adapt_features(x, y)
            return True

        # Update online feature transforms
        self._update_transforms(x)

        return False

    def _update_transforms(self, x: Dict) -> None:
        """Update online feature transformations (scaling, encoding, etc.)."""
        # Update online scaler
        self.online_scaler.learn_one(x)

    def _adapt_features(self, x: Dict, y: float) -> None:
        """
        Adapt feature transformations in response to drift.
        Re-learn feature parameters from recent data window.
        """
        logger.info("Adapting features in response to concept drift...")

        # Reset online scaler to adapt to new distribution
        self.online_scaler = preprocessing.StandardScaler()

        # Could also trigger:
        # - Re-training categorical encoders with new categories
        # - Updating binning thresholds for numerical features
        # - Re-computing interaction features with updated importance

    def transform(self, x: Dict) -> Dict:
        """
        Transform features using current (adapted) transformations.

        Args:
            x: Raw feature dictionary

        Returns:
            Transformed feature dictionary
        """
        # Apply online scaling
        x_scaled = self.online_scaler.transform_one(x)

        # Apply custom learned transformations
        for feature_name, transform_func in self.feature_transforms.items():
            if feature_name in x_scaled:
                x_scaled[f"{feature_name}_transformed"] = transform_func(
                    x_scaled[feature_name]
                )

        return x_scaled

    def get_drift_statistics(self) -> Dict[str, Any]:
        """Get statistics about detected drift events."""
        return {
            'total_drift_events': self.drift_detected_count,
            'detector_type': self.drift_detector_name,
            'current_drift_score': getattr(self.drift_detector, 'estimation', None)
        }
\end{lstlisting}

\section{Feature Platform Architecture Patterns}

Building enterprise-grade feature platforms requires careful architectural design to support scalability, reliability, and governance. This section presents comprehensive architecture patterns for production feature systems.

\subsection{Layered Feature Platform Architecture}

\begin{lstlisting}[language=Python, caption={Complete Feature Platform Architecture}]
from enum import Enum
from abc import ABC, abstractmethod

class ComputeLayer(Enum):
    """Feature computation layers with different SLAs."""
    BATCH = "batch"  # Hourly/daily batch jobs
    STREAMING = "streaming"  # Near real-time (seconds)
    REALTIME = "realtime"  # Online computation (<100ms)


class FeaturePlatform:
    """
    Enterprise feature platform supporting the full ML lifecycle.

    Architecture layers:
    1. Ingestion Layer: Raw data from sources
    2. Transformation Layer: Feature computation (batch/streaming/realtime)
    3. Storage Layer: Offline (training) and Online (serving) stores
    4. Serving Layer: Low-latency feature retrieval
    5. Monitoring Layer: Drift detection, quality metrics
    6. Governance Layer: Lineage, access control, compliance
    """

    def __init__(self,
                 offline_store: "OfflineStore",
                 online_store: "OnlineStore",
                 transformation_engine: "TransformationEngine",
                 monitoring_service: "MonitoringService",
                 governance_service: "GovernanceService"):
        self.offline_store = offline_store
        self.online_store = online_store
        self.transformation_engine = transformation_engine
        self.monitoring = monitoring_service
        self.governance = governance_service

        self.feature_registry: Dict[str, FeatureDefinition] = {}

    def register_feature(self, feature_def: "FeatureDefinition",
                        owner: str, team: str) -> None:
        """
        Register a new feature with governance metadata.

        Args:
            feature_def: Feature definition (computation logic)
            owner: Individual owner
            team: Team responsible for feature
        """
        # Validate feature definition
        self._validate_feature_definition(feature_def)

        # Check for naming conflicts
        if feature_def.name in self.feature_registry:
            raise ValueError(f"Feature '{feature_def.name}' already exists")

        # Register with governance
        self.governance.register_feature(
            feature_def=feature_def,
            owner=owner,
            team=team,
            registration_time=datetime.now()
        )

        # Add to registry
        self.feature_registry[feature_def.name] = feature_def

        logger.info(f"Registered feature '{feature_def.name}' (owner: {owner})")

    def materialize_features(self,
                            feature_group: str,
                            start_time: datetime,
                            end_time: datetime,
                            compute_layer: ComputeLayer = ComputeLayer.BATCH) -> None:
        """
        Compute and materialize features for a time range.

        Args:
            feature_group: Group of features to materialize
            start_time: Start of time window
            end_time: End of time window
            compute_layer: Where to compute (batch/streaming/realtime)
        """
        logger.info(f"Materializing '{feature_group}' from {start_time} to {end_time}")

        # Get features in group
        features = self._get_features_in_group(feature_group)

        # Execute transformation
        features_df = self.transformation_engine.compute_features(
            features=features,
            start_time=start_time,
            end_time=end_time,
            compute_layer=compute_layer
        )

        # Write to offline store (for training)
        self.offline_store.write_features(
            feature_group=feature_group,
            features_df=features_df,
            event_timestamp_column="timestamp"
        )

        # Write to online store (for serving)
        if compute_layer in [ComputeLayer.STREAMING, ComputeLayer.REALTIME]:
            self.online_store.write_features(
                feature_group=feature_group,
                features_df=features_df
            )

        # Update monitoring metrics
        self.monitoring.track_materialization(
            feature_group=feature_group,
            row_count=len(features_df),
            compute_time=(datetime.now() - start_time).total_seconds()
        )

    def get_training_features(self,
                             entity_df: pd.DataFrame,
                             features: List[str],
                             timestamp_column: str = "event_timestamp") -> pd.DataFrame:
        """
        Get point-in-time correct features for model training.

        Critical for preventing data leakage: only returns features
        that were available at the specified event timestamp.

        Args:
            entity_df: DataFrame with entities and timestamps
            features: List of feature names to retrieve
            timestamp_column: Column containing event timestamps

        Returns:
            DataFrame with entity IDs, timestamps, and features
        """
        # Validate features exist
        for feature in features:
            if feature not in self.feature_registry:
                raise ValueError(f"Feature '{feature}' not found")

        # Point-in-time join from offline store
        result = self.offline_store.get_point_in_time_features(
            entity_df=entity_df,
            features=features,
            timestamp_column=timestamp_column
        )

        # Track lineage
        self.governance.track_feature_usage(
            features=features,
            usage_type="training",
            timestamp=datetime.now()
        )

        return result

    def get_online_features(self,
                           entity_ids: List[str],
                           features: List[str]) -> pd.DataFrame:
        """
        Get features for real-time inference (low latency).

        Args:
            entity_ids: List of entity IDs
            features: Feature names to retrieve

        Returns:
            DataFrame with features for each entity
        """
        # Retrieve from online store
        result = self.online_store.get_features(
            entity_ids=entity_ids,
            features=features
        )

        # Monitor serving latency
        # Track usage for governance

        return result

    def monitor_feature_quality(self, feature_name: str,
                               production_data: pd.DataFrame) -> Dict[str, Any]:
        """
        Monitor feature quality in production.

        Checks:
        - Distribution drift vs. training data
        - Missing value rate
        - Outlier rate
        - Correlation with target (if available)
        """
        if feature_name not in self.feature_registry:
            raise ValueError(f"Feature '{feature_name}' not found")

        # Get training reference distribution
        feature_def = self.feature_registry[feature_name]
        training_reference = self.offline_store.get_reference_distribution(
            feature_name=feature_name
        )

        # Calculate quality metrics
        quality_metrics = self.monitoring.compute_quality_metrics(
            feature_name=feature_name,
            production_data=production_data[feature_name],
            reference_data=training_reference
        )

        # Check for alerts
        alerts = self.monitoring.check_alert_thresholds(
            feature_name=feature_name,
            metrics=quality_metrics
        )

        if alerts:
            logger.warning(f"Quality alerts for '{feature_name}': {alerts}")

        return quality_metrics

    def get_feature_lineage(self, feature_name: str) -> Dict[str, Any]:
        """
        Get complete lineage for a feature.

        Returns:
            - Source data tables and columns
            - Transformation logic
            - Downstream models using this feature
            - Owners and teams
            - Version history
        """
        return self.governance.get_feature_lineage(feature_name)

    def _validate_feature_definition(self, feature_def: "FeatureDefinition") -> None:
        """Validate feature definition before registration."""
        # Check required fields
        if not feature_def.name:
            raise ValueError("Feature name required")

        if not feature_def.transformation_logic:
            raise ValueError("Transformation logic required")

        # Validate computation feasibility
        if feature_def.compute_layer == ComputeLayer.REALTIME:
            estimated_latency = self._estimate_computation_latency(feature_def)
            if estimated_latency > 100:  # ms
                raise ValueError(
                    f"Feature computation too slow for realtime "
                    f"(estimated: {estimated_latency}ms)"
                )

    def _estimate_computation_latency(self, feature_def: "FeatureDefinition") -> float:
        """Estimate feature computation latency."""
        # Simplified estimation based on transformation complexity
        # In production, would profile actual computation
        base_latency = 10  # ms

        if "aggregation" in feature_def.transformation_logic.lower():
            base_latency += 50

        if "join" in feature_def.transformation_logic.lower():
            base_latency += 30

        return base_latency

    def _get_features_in_group(self, feature_group: str) -> List["FeatureDefinition"]:
        """Get all features belonging to a feature group."""
        return [
            feature_def for feature_def in self.feature_registry.values()
            if feature_def.feature_group == feature_group
        ]
\end{lstlisting}

This enterprise feature platform architecture ensures:
\begin{itemize}
    \item \textbf{Scalability}: Separate compute layers for different latency requirements
    \item \textbf{Consistency}: Point-in-time correctness prevents data leakage
    \item \textbf{Observability}: Comprehensive monitoring and quality metrics
    \item \textbf{Governance}: Lineage tracking, access control, and documentation
    \item \textbf{Reliability}: Fallback mechanisms and graceful degradation
\end{itemize}

\section{Exercises}

\subsection{Exercise 1: Basic Feature Engineering Pipeline (Easy)}

Create a feature engineering pipeline for a dataset with customer purchase history. Implement:
\begin{itemize}
    \item Temporal features from purchase dates
    \item Frequency encoding for product categories
    \item Basic validation checks
\end{itemize}

Test with sample data and verify all features pass validation.

\subsection{Exercise 2: Cyclic Feature Encoding (Easy)}

Implement cyclic encoding for time-based features (hour, day-of-week, month). Create visualizations showing why cyclic encoding is superior to linear encoding for capturing temporal patterns.

Compare model performance (simple logistic regression) using linear vs. cyclic encoding on a time-sensitive classification task.

\subsection{Exercise 3: High-Cardinality Categorical Encoding (Medium)}

You have a user\_id feature with 100,000 unique values and a binary target (clicked/not clicked). Implement and compare:
\begin{itemize}
    \item Frequency encoding
    \item Target encoding with smoothing
    \item Hash encoding
\end{itemize}

Evaluate which encoding strategy provides the best model performance and explain why.

\subsection{Exercise 4: Feature Selection Consensus (Medium)}

Create a synthetic dataset with:
\begin{itemize}
    \item 10 truly predictive features
    \item 20 random noise features
    \item 5 redundant features (copies with small noise)
\end{itemize}

Apply all four feature selection methods from the chapter. Analyze:
\begin{itemize}
    \item Which methods successfully identify the true features?
    \item How many methods are needed in consensus to filter out noise?
    \item How does correlation threshold affect redundancy detection?
\end{itemize}

\subsection{Exercise 5: Feature Stability Analysis (Medium)}

Implement a feature stability checker that compares feature importance across different train/test splits. For an unstable feature, investigate:
\begin{itemize}
    \item Why does it show high variance across folds?
    \item How does sample size affect stability?
    \item Can transformation (e.g., binning, smoothing) improve stability?
\end{itemize}

Create a visualization showing stability scores for all features.

\subsection{Exercise 6: Production Drift Detection (Advanced)}

Simulate production drift by:
\begin{enumerate}
    \item Training a model on 2023 e-commerce data
    \item Creating synthetic 2024 data with gradual drift (changing customer behavior)
    \item Implementing the FeatureMonitor to detect drift
\end{enumerate}

Set up alerting thresholds and create a dashboard showing:
\begin{itemize}
    \item Feature drift over time
    \item Model performance degradation
    \item Triggered alerts and their severity
\end{itemize}

\subsection{Exercise 7: End-to-End Feature Engineering System (Advanced)}

Build a complete feature engineering system for a real-world problem (e.g., credit risk, customer churn):

\begin{enumerate}
    \item Design domain-driven features based on problem understanding
    \item Implement a multi-stage pipeline with validation
    \item Apply multiple feature selection methods
    \item Validate stability and production readiness
    \item Set up monitoring with drift detection
    \item Version features using FeatureVersionManager
    \item Compare model performance: baseline vs. engineered features
\end{enumerate}

Document the impact of each stage on model performance and create a feature engineering report suitable for stakeholders.

\subsection{Exercise 8: Genetic Programming Feature Generation (Advanced)}

Implement an automated feature generation system using genetic programming:

\begin{enumerate}
    \item Use the \texttt{GeneticFeatureGenerator} class to evolve 20 features on a regression dataset
    \item Analyze the evolved mathematical expressions for interpretability
    \item Compare performance against manually engineered features
    \item Implement fitness function customization to penalize overly complex features
    \item Create a feature selection step to identify which evolved features are actually useful
\end{enumerate}

Deliverables:
\begin{itemize}
    \item Report showing the top 5 evolved features and their mathematical expressions
    \item Performance comparison: baseline vs. genetic features vs. manual features
    \item Analysis of feature complexity vs. predictive power trade-offs
    \item Recommendations for when genetic programming is worth the computational cost
\end{itemize}

\subsection{Exercise 9: Real-Time Feature Store Implementation (Advanced)}

Build a complete feature store with both real-time and batch serving capabilities:

\begin{enumerate}
    \item Design schema for online and offline feature tables
    \item Implement batch feature computation pipeline (using pandas/Spark)
    \item Create real-time feature serving API with sub-100ms latency
    \item Add point-in-time correctness for historical feature retrieval
    \item Implement feature caching strategy for frequently accessed features
    \item Set up consistency checks between online and offline stores
\end{enumerate}

Technical requirements:
\begin{itemize}
    \item Use Redis/DynamoDB for online store, Parquet/Delta for offline store
    \item Implement feature materialization jobs with backfill support
    \item Create monitoring dashboards for feature freshness and serving latency
    \item Handle feature value staleness with configurable TTLs
    \item Implement graceful degradation when features are unavailable
\end{itemize}

Test scenarios:
\begin{itemize}
    \item Verify point-in-time correctness: historical features match what was available at that time
    \item Load test: serve 10,000 requests/second with p99 latency under 50ms
    \item Consistency test: online and offline stores return same values (within configured freshness window)
\end{itemize}

\subsection{Exercise 10: Causal Feature Selection Framework (Advanced)}

Implement causal feature selection with confounding adjustment:

\begin{enumerate}
    \item Create a synthetic dataset with known causal structure (use DAG)
    \item Implement backdoor criterion to identify confounders
    \item Apply propensity score matching for feature selection
    \item Compare causal feature selection vs. correlation-based selection
    \item Analyze impact on model generalization to different distributions
\end{enumerate}

Use the \texttt{dowhy} or \texttt{causalml} library for causal inference. Create scenarios where:
\begin{itemize}
    \item Correlation-based selection includes spurious features
    \item Causal selection correctly identifies true causal features
    \item Models trained on causal features generalize better to shifted distributions
\end{itemize}

Deliverables:
\begin{itemize}
    \item Causal DAG visualization showing true causal relationships
    \item Comparison table: features selected by correlation vs. causality
    \item Performance evaluation on both i.i.d. test set and shifted distribution
    \item Report explaining when causal feature selection provides value
\end{itemize}

\subsection{Exercise 11: Feature Lineage and Governance System (Advanced)}

Build a comprehensive feature governance framework:

\begin{enumerate}
    \item Implement automated feature lineage tracking from raw data to final features
    \item Create feature documentation system with ownership and metadata
    \item Build feature discovery portal (searchable catalog)
    \item Implement access control for sensitive features (PII, protected attributes)
    \item Set up automated feature quality monitoring with SLA tracking
    \item Create deprecation workflow for retiring features
\end{enumerate}

Technical components:
\begin{itemize}
    \item Feature catalog database with full lineage graph
    \item Automated documentation extraction from pipeline code
    \item Data quality rules engine with configurable thresholds
    \item Audit logging for feature access and usage
    \item Integration with model registry to track feature-model relationships
\end{itemize}

Governance features:
\begin{itemize}
    \item Owner assignment and on-call rotation for feature issues
    \item Feature certification workflow (experimental → validated → production)
    \item Cost attribution per feature (storage + compute)
    \item Impact analysis: which models depend on a feature
    \item Compliance tracking for regulatory requirements
\end{itemize}

\subsection{Exercise 12: Online Feature Learning with Drift Adaptation (Advanced)}

Implement an online learning system that adapts features to concept drift:

\begin{enumerate}
    \item Create streaming feature pipeline that updates statistics in real-time
    \item Implement online feature selection with sliding window evaluation
    \item Build drift detector that triggers feature re-engineering
    \item Design adaptive binning/encoding that adjusts to distribution shifts
    \item Set up A/B testing framework for feature changes
\end{enumerate}

Scenarios to implement:
\begin{itemize}
    \item \textbf{Gradual drift}: User behavior slowly changes over months
    \item \textbf{Sudden drift}: Major event causes immediate distribution shift
    \item \textbf{Recurring drift}: Seasonal patterns require time-aware features
\end{itemize}

Technical implementation:
\begin{itemize}
    \item Use streaming framework (Kafka + Flink/Spark Streaming)
    \item Implement ADWIN or Page-Hinkley test for drift detection
    \item Create feature adaptation strategies:
    \begin{itemize}
        \item Dynamic bucketing based on current distribution
        \item Online updating of target encoding statistics
        \item Adaptive scaling based on running statistics
    \end{itemize}
    \item Build rollback mechanism if new features degrade performance
\end{itemize}

Evaluation metrics:
\begin{itemize}
    \item Time to detect drift after it occurs
    \item Model performance recovery time after adaptation
    \item False positive rate for drift detection
    \item A/B test results comparing static vs. adaptive features
\end{itemize}

\section{Summary}

This chapter presented a systematic, production-ready approach to feature engineering:

\begin{itemize}
    \item \textbf{Feature Engineering Pipeline}: Type-safe framework with validation, metadata tracking, and reproducibility
    \item \textbf{Domain-Driven Features}: Temporal extraction (cyclic encoding, lags, rolling), categorical encoding (automatic strategy selection), numerical transformations (distribution-aware)
    \item \textbf{Feature Selection}: Statistical tests, mutual information, model-based importance, RFE, and consensus methods
    \item \textbf{Feature Validation}: Stability analysis across CV folds, redundancy detection, production readiness checks
    \item \textbf{Production Monitoring}: Drift detection using KS tests (numerical) and chi-squared (categorical), automated alerting, historical tracking
    \item \textbf{Feature Store Integration}: Versioning, compatibility checking, and centralized feature management
\end{itemize}

Feature engineering is both an art and a science. While domain knowledge drives creativity, systematic engineering practices ensure reliability, reproducibility, and maintainability. By combining statistical rigor with production-ready tooling, teams can build features that not only improve model performance but remain stable and observable in production environments.
