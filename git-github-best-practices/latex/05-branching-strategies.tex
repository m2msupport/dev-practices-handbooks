\chapter{Branching Strategies}
\section{Chapter Overview}
Branch policy determines merge risk, conflict frequency, and release cadence. Whether you embrace trunk-based development or regulated release trains, the strategy must minimize drift and clarify responsibilities.

\section{Trunk-Based with Short-Lived Branches}
Trunk-based teams keep \texttt{main} releasable and rely on feature flags to hide incomplete work. Branches should live for days, not weeks, and CI must stay green or merges stop.

\begin{lstlisting}[style=shell,caption={Trunk-based flow with feature flags}]
$ git switch main
$ git pull --ff-only
$ git switch -c feat/alerts-toggle
$ git add flag_config.yaml alerts.py
$ git commit -m "feat: add alert suppression flag"
$ git push -u origin feat/alerts-toggle
$ gh pr merge --squash --auto feat/alerts-toggle
\end{lstlisting}

\section{Git Flow in Regulated Teams}
Highly regulated teams need long-lived release branches for audits. Automate merges between \texttt{develop}, \texttt{release/x.y}, and \texttt{main} so fixes propagate deterministically and no branch becomes the "real" source of truth.

\begin{lstlisting}[style=shell,caption={Synchronizing develop and release branches}]
$ git switch develop
$ git pull
$ git switch -c release/2024.06
$ git push -u origin release/2024.06
$ git switch release/2024.06
$ git merge --no-ff develop
$ git push origin release/2024.06
$ git switch develop
$ git merge --no-ff release/2024.06
$ git push origin develop
\end{lstlisting}

\section{Environment Branches Are Anti-Patterns}
Branches named \texttt{staging} or \texttt{qa} collect unreviewed fixes and confuse incident responders. Replace them with deployment tags or environment-specific workflows tied to the real source branch.

\begin{lstlisting}[style=shell,caption={Deploying via tags instead of environment branches}]
$ git tag -a staging-2024-05-18 -m "Deploy build 345 to staging"
$ git push origin staging-2024-05-18
$ gh workflow run deploy.yml -f environment=staging -f ref=staging-2024-05-18
\end{lstlisting}

\section*{Exercises}
\begin{enumerate}
  \item Inventory branch types in your org and document expected lifetime plus naming conventions.
  \item List remote branches older than 30 days and decide which to prune or archive.
  \item Design a feature-flag rollout plan that keeps \texttt{main} releasable even when work is half-done.
  \item Diagram how fixes move between \texttt{develop}, \texttt{release/x.y}, and \texttt{main} today; highlight bottlenecks.
  \item Replace one environment branch with a tag-driven deployment in a sandbox repo.
  \item Compare squash merges and merge commits for trunk-based work and justify your default.
  \item Set up monitoring for branch age and open-PR lifetime to spot stagnation early.
\end{enumerate}

