\chapter{Comments and Documentation}
\section{Chapter Overview}
Well-structured code minimises the need for comments, yet documentation remains essential
for sharing context, intent, and processes.

\begin{lstlisting}[caption={Tracking documentation freshness},label={lst:docs_overview}]
from __future__ import annotations

from dataclasses import dataclass
from datetime import datetime


@dataclass(slots=True)
class DocumentStatus:
    """Keep tabs on which guides need attention."""

    path: str
    last_reviewed: datetime


def needs_refresh(status: DocumentStatus, *, months: int = 6) -> bool:
    """Flag documents that have gone stale."""
    delta = datetime.now(tz=status.last_reviewed.tzinfo) - status.last_reviewed
    return delta.days > months * 30
\end{lstlisting}

\section{Docstrings that Educate}
Docstrings should answer three questions: what does this item do, what are its inputs and
outputs, and when does it raise exceptions?  Follow the "one-line summary + details"
format so tools like Sphinx can render the text neatly.

\begin{lstlisting}[caption={Docstring capturing behaviour and failure modes},label={lst:docs_docstring}]
from __future__ import annotations


def parse_currency(value: str) -> int:
    """Return cents for a currency string; raise ValueError on malformed input."""
    if not value.startswith("$"):
        raise ValueError("Currency must start with a dollar sign.")
    dollars, cents = value[1:].split(".")
    return int(dollars) * 100 + int(cents)
\end{lstlisting}

\section{Inline Comments for Intent}
Inline comments should explain \emph{why}, not \emph{what}.  They are ideal for noting
workarounds, referencing bug tickets, or clarifying domain constraints.  Delete them once
they become obsolete to avoid misleading readers.

\begin{lstlisting}[caption={Inline comments that explain surprising workarounds},label={lst:docs_inline}]
from __future__ import annotations

from datetime import UTC, datetime


def next_billing_window(now: datetime) -> datetime:
    """Return the start of the next billing window."""
    window = now.astimezone(UTC).replace(minute=0, second=0, microsecond=0)
    # Stripe batches reconcile at 5-minute intervals, so align to that cadence.
    return window.replace(minute=(window.minute // 5 + 1) * 5)
\end{lstlisting}

\section{Project Documentation Stack}
Combine lightweight Markdown guides with generated API references.  Example structure:
\texttt{README.md} for onboarding, \texttt{docs/architecture.md} for high-level design,
Sphinx or MkDocs for API references, and runbooks for deployment and incident response.

\begin{lstlisting}[caption={Generating a docs index programmatically},label={lst:docs_stack}]
from __future__ import annotations

from pathlib import Path


def ensure_docs_index(root: Path) -> None:
    """Create a landing page that references every Markdown guide."""
    guides = sorted(root.glob("*.md"))
    body = "\n".join(f"- [{guide.stem}]({guide.name})" for guide in guides)
    (root / "index.md").write_text(body, encoding="utf-8")
\end{lstlisting}

\section{Code Example: Documentation-Driven Development}
Listing~\ref{lst:documented_function} demonstrates an API that uses docstrings and inline
comments to highlight non-obvious behaviour.

\begin{lstlisting}[caption={Docstring and inline comment illustrating intent},label={lst:documented_function}]
from __future__ import annotations

from datetime import UTC, datetime


def anonymise_timestamp(raw_timestamp: str) -> datetime:
    """Return an hourly bucket so analytics cannot deanonymise individual events."""
    timestamp = datetime.fromisoformat(raw_timestamp).astimezone(UTC)
    # We discard minutes/seconds to comply with the analytics retention policy.
    return timestamp.replace(minute=0, second=0, microsecond=0)
\end{lstlisting}

\section{Scenario: Docs Saved a Launch}
A healthcare startup had to demonstrate auditability to regulators.  Because the team
maintained up-to-date architecture diagrams and onboarding walkthroughs, they produced the
requested evidence in hours instead of weeks.  The documentation discipline paid off even
though the company had never planned for a government review.

\begin{lstlisting}[caption={Collecting evidence links quickly},label={lst:docs_audit}]
from __future__ import annotations

from pathlib import Path


def gather_audit_artifacts(root: Path) -> list[Path]:
    """Return runbooks and diagrams needed during regulated reviews."""
    return [path for path in root.rglob("*") if path.suffix in {".md", ".drawio"}]
\end{lstlisting}

\section{Summary}
Documentation succeeds when it explains intent, evolves with the system, and exists at
multiple layers—from docstrings to architecture guides and runbooks.

\begin{lstlisting}[caption={Summarising documentation gaps},label={lst:docs_summary}]
from __future__ import annotations


def summarize_gaps(gaps: dict[str, list[str]]) -> str:
    """Turn gap analysis into a concise statement for leadership."""
    missing = ", ".join(sorted(gaps))
    return f"Docs need updates for {missing}; owners assigned."
\end{lstlisting}

\section*{Exercises}
\begin{enumerate}
  \item Choose an undocumented function and write a docstring that follows the style in
  Listing~\ref{lst:documented_function}.
  \item Create an inline comment explaining a surprising line of code, then refactor until
  the comment is unnecessary.
  \item Generate API documentation with Sphinx for a small module.  What metadata is
  missing?
  \item Draft a runbook for a scheduled job.  Include inputs, expected outputs, and failure
  procedures.
  \item Interview a new hire about confusing documentation gaps and log the findings as
  GitHub issues.
  \item Use the snippet below to explain how you would document the edge cases.
  \begin{lstlisting}[caption={Exercise: document behaviour},label={lst:docs_ex_doc}]
def compute(offset: int) -> int:
    if offset < 0:
        return 0
    return offset * 2
  \end{lstlisting}
  \item Sketch a documentation stack diagram that shows how README files, architecture
  records, and runbooks link to each other.  Identify who updates each artefact.
\end{enumerate}
