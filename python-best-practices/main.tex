\documentclass[11pt]{book}

\usepackage[utf8]{inputenc}
\usepackage[T1]{fontenc}
\usepackage[margin=2.5cm]{geometry}
\usepackage{lmodern}
\usepackage{hyperref}
\usepackage{xcolor}
\usepackage{listings}
\usepackage{microtype}
\usepackage{csquotes}

\hypersetup{
  colorlinks=true,
  linkcolor=blue,
  urlcolor=blue,
  citecolor=blue
}

\lstdefinestyle{python}{
  language=Python,
  basicstyle=\ttfamily\small,
  keywordstyle=\color{blue},
  stringstyle=\color{teal},
  commentstyle=\itshape\color{gray},
  showstringspaces=false,
  frame=single,
  breaklines=true,
  tabsize=4,
  captionpos=b
}
\lstset{style=python}

\title{Python Development Best Practices}
\author{Diogo Ribeiro\\
Data Scientist Lead\\
Portugal\\
\texttt{https://diogoribeiro7.github.io}\\
\texttt{https://github.com/DiogoRibeiro7}}
\date{\today}

\begin{document}

\frontmatter

\maketitle

\chapter*{Abstract}
This book expands a practitioner-oriented article into a comprehensive handbook for
intermediate and advanced Python developers.  It explains why best practices matter,
demonstrates how to apply them with idiomatic Python~3.12 examples, and describes how real
teams use these techniques to ship reliable software.  Each chapter blends theory,
scenarios, code, and exercises so readers can internalise the material and adapt it to
their own projects.

\tableofcontents

\mainmatter

\chapter{Introduction: Why Best Practices Matter}
\section{Chapter Overview}
Professional Python development rarely fails because of missing syntax knowledge.  Most
problems emerge when the surrounding practices—testing, structure, automation, culture—are
weak or inconsistent.  This opening chapter reframes best practices as strategic tools for
reducing risk and enabling collaboration.

\begin{lstlisting}[caption={Tracking practice adoption across a portfolio},label={lst:intro_practice_debt}]
from __future__ import annotations

from dataclasses import dataclass


@dataclass(slots=True)
class PracticeScore:
    """Represent how consistently a team applies a given practice."""

    name: str
    health: float  # 0-1 scale
    blocking_incidents: int


def needs_exec_attention(score: PracticeScore) -> bool:
    """Escalate when weakened practices repeatedly cause incidents."""
    return score.health < 0.6 and score.blocking_incidents >= 2
\end{lstlisting}

Teams often add similar lightweight metrics to post-incident reviews so they can link
failures back to neglected practices instead of blaming individuals.

\section{From Scripts to Systems}
Many engineers fall in love with Python by automating a tedious task.  The moment that
script becomes a service, library, or data pipeline, the definition of success changes.
Colleagues need repeatable environments, reliable releases, and discoverable behaviour.
Ignoring those expectations leads to outages that are harder to diagnose than the original
problem.

\begin{lstlisting}[caption={Promoting an exploratory script into a service-friendly module},label={lst:intro_script_to_system}]
from __future__ import annotations

from dataclasses import dataclass
from pathlib import Path


@dataclass(slots=True)
class SyncJobConfig:
    """Describe the inputs required to run the nightly sync job."""

    input_dir: Path
    output_dir: Path
    dry_run: bool = False


def run_sync_job(config: SyncJobConfig) -> int:
    """Convert ad-hoc filesystem work into a composable, testable unit."""
    processed = 0
    for source in config.input_dir.glob("*.csv"):
        destination = config.output_dir / source.name
        if not config.dry_run:
            destination.write_text(source.read_text(encoding="utf-8"), encoding="utf-8")
        processed += 1
    return processed
\end{lstlisting}

Bad scripts hide configuration inside global constants; production-ready modules model
dependencies explicitly the way `SyncJobConfig` does, making it obvious how to parameterise
jobs for CI or data replay.

\subsection{Scenario: The Overnight Script}
A support engineer wrote a quick file-parsing script that was then scheduled to run nightly
on production data.  Weeks later an upstream schema change broke the script silently and
the company did not notice until invoices failed.  The fix involved adding tests,
documentation, and alerts—essentially retrofitting everything a well-structured project
would have had from day one.  The lesson: treat every piece of code as a future dependency.

\begin{lstlisting}[caption={Anti-pattern: implicit schema assumptions},label={lst:intro_schema_bad}]
def nightly_parse(path: str) -> list[str]:
    """Process rows but fail silently when columns change."""
    results = []
    with open(path, encoding="utf-8") as handle:
        for line in handle:
            parts = line.strip().split(",")
            results.append(parts[3])  # hard-coded column index
    return results
\end{lstlisting}

\begin{lstlisting}[caption={Resilient parser with validation hooks},label={lst:intro_schema_good}]
from __future__ import annotations

from pathlib import Path

EXPECTED_HEADERS = ("invoice_id", "customer_id", "status", "total")


def parse_invoices(path: Path) -> list[dict[str, str]]:
    """Validate incoming files so incidents surface immediately."""
    lines = path.read_text(encoding="utf-8").splitlines()
    header = tuple(lines[0].split(","))
    if header != EXPECTED_HEADERS:
        raise ValueError(f"Unexpected schema: {header}")
    return [
        dict(zip(EXPECTED_HEADERS, row.split(","), strict=True))
        for row in lines[1:]
    ]
\end{lstlisting}

When the team deployed the validated parser, they also routed failures to PagerDuty,
transforming a silent data loss issue into a ten-minute debugging session.

\section{Four Pillars of Professional Python}
Best practices reinforce four outcomes:
\begin{description}
  \item[Readability] Code tells a clear story so reviewers and incident responders can
  reason quickly.
  \item[Reliability] Tests, type hints, and logging ensure behaviour is predictable and
  regressions are caught early.
  \item[Maintainability] Modular design, versioning, and automation keep change costs low
  even as teams grow.
  \item[Reproducibility] Environments, data, and configuration can be recreated on fresh
  machines, enabling confident deployments.
\end{description}

\begin{lstlisting}[caption={Capturing pillar health as structured data},label={lst:intro_pillars}]
from __future__ import annotations

from enum import Enum, auto


class Pillar(Enum):
    READABILITY = auto()
    RELIABILITY = auto()
    MAINTAINABILITY = auto()
    REPRODUCIBILITY = auto()


def missing_pillars(observed: set[Pillar]) -> list[Pillar]:
    """Return the set of pillars not satisfied by a service review."""
    return [pillar for pillar in Pillar if pillar not in observed]
\end{lstlisting}

Teams often embed functions like \texttt{missing\_pillars} inside CI linters so pull requests fail
when key artefacts—tests, docs, deployment manifests—are absent.

\section{A Motivating Code Example}
Listing~\ref{lst:intro_example} juxtaposes an exploratory script with a production-ready
function.  The logic is the same, but the maintainable version adds validation, typing,
and logging hooks to support future owners.

\begin{lstlisting}[caption={Transforming a script into a reusable function},label={lst:intro_example}]
from __future__ import annotations

from collections.abc import Iterable


def total_active_users(user_counts: Iterable[int]) -> int:
    """Return the sum of daily active users after validating inputs."""
    totals = []
    for count in user_counts:
        if count < 0:
            raise ValueError("daily active users cannot be negative")
        totals.append(count)
    return sum(totals)
\end{lstlisting}

\section{How to Use This Book}
Read sequentially if you are building a new codebase.  Dip into specific chapters when
improving an existing system.  Each chapter ends with exercises to reinforce the concepts
and to provide prompts for team discussions.

\begin{lstlisting}[caption={Generating a study plan from chapter metadata},label={lst:intro_study_plan}]
from __future__ import annotations

from dataclasses import dataclass


@dataclass(slots=True)
class Chapter:
    """Metadata describing how a team might prioritise chapters."""

    name: str
    focus: str
    effort_hours: int


def pick_next_chapter(chapters: list[Chapter], topic: str) -> Chapter:
    """Select the shortest chapter that matches the requested topic."""
    candidates = [chapter for chapter in chapters if chapter.focus == topic]
    return min(candidates, key=lambda chapter: chapter.effort_hours)
\end{lstlisting}

Embedding chapter planning in onboarding tooling helps managers pair new hires with the
sections most relevant to current incidents or projects.

\section{Summary}
Best practices succeed when teams treat every script as a future dependency, pursue the four
pillars consistently, and lean on automation plus documentation to scale collaboration.  The
rest of the book dives into each pillar with practical guidance.

\begin{lstlisting}[caption={Turning retrospectives into action items},label={lst:intro_summary}]
from __future__ import annotations

from collections.abc import Iterable


def summarize_retrospective(findings: Iterable[str]) -> dict[str, list[str]]:
    """Group retrospective findings by pillar for postmortem dashboards."""
    summary: dict[str, list[str]] = {"people": [], "process": [], "tooling": []}
    for finding in findings:
        bucket = "tooling" if "automation" in finding else "process"
        if "on-call" in finding:
            bucket = "people"
        summary[bucket].append(finding)
    return summary
\end{lstlisting}

\section*{Exercises}
\begin{enumerate}
  \item Identify a script in your organisation that quietly became a service.  List the
  missing practices (tests, logging, configuration) and sketch a plan to add them.
  \item Rewrite a quick-and-dirty function by applying the patterns from
  Listing~\ref{lst:intro_example}.  What impeded readability?
  \item Interview a teammate about a recent incident.  Which of the four pillars would have
  prevented or shortened it?
  \item Draft a one-page document explaining how new hires should navigate your repository.
  Share it with the team and capture feedback.
  \item Build a checklist for scripts that might graduate to production and store it in
  version control.
\end{enumerate}

\chapter{Code Style and Readability}
\section{Chapter Overview}
Style is not mere aesthetics; it is a communication contract.  When code follows shared
conventions, teams spend less time deciphering formatting and more time reasoning about
behaviour.

\begin{lstlisting}[caption={Capturing style rule adherence},label={lst:style_score}]
from __future__ import annotations

from dataclasses import dataclass


@dataclass(slots=True)
class StyleScore:
    """Lightweight structure for tracking readability drift."""

    module: str
    naming_rules_passed: bool
    formatter_ran: bool
\end{lstlisting}

\section{PEP~8 as a Shared Vocabulary}
PEP~8 codifies naming, indentation, and whitespace rules.  Adhering to it means any Python
developer can jump into the repository without a style orientation session.  When exceptions
are necessary—long SQL strings or generated code—document the rationale in the README or a
style guide.

\begin{lstlisting}[caption={Encoding naming rules in review helpers},label={lst:pep8_helper}]
from __future__ import annotations


def enforce_snake_case(names: list[str]) -> list[str]:
    """Return identifiers that require refactoring."""
    return [name for name in names if any(ch.isupper() for ch in name)]
\end{lstlisting}

Many teams run helpers like \texttt{enforce\_snake\_case} in review bots so regression is impossible
without an explicit waiver.

\section{Tooling Over Taste}
Automate style enforcement to reduce cognitive load.  \texttt{black} or \texttt{ruff
format} handle formatting, \texttt{ruff} enforces linting rules, and \texttt{isort} (or
\texttt{ruff}'s import subsystem) maintains deterministic import order.  Configure the tools
in \texttt{pyproject.toml} and run them via \texttt{pre-commit} so every change goes through
the same gate.

\begin{lstlisting}[caption={Guarding formatters inside CI scripts},label={lst:style_tooling}]
from __future__ import annotations

import subprocess


def run_formatter() -> None:
    """Exit the build early when the formatter changes files."""
    result = subprocess.run(["ruff", "format", "--check", "."], check=False)
    if result.returncode != 0:
        msg = "Formatting drift detected; run `ruff format` locally."
        raise SystemExit(msg)
\end{lstlisting}

\section{Bad vs Good Example}
Listing~\ref{lst:bad_style} and Listing~\ref{lst:good_style} show how naming and structure
transform comprehension.  Both compute a discounted total, but the improved version reveals
intent immediately.

\begin{lstlisting}[caption={Anti-pattern: inconsistent naming and hidden branching},label={lst:bad_style}]
def calc(d, p):
    t = 0
    for x in d:
        t += x
    if p:
        t -= t * p
    return t
\end{lstlisting}

\begin{lstlisting}[caption={Readable implementation with validation},label={lst:good_style}]
from __future__ import annotations


def calculate_discounted_total(prices: list[float], discount: float) -> float:
    """Return the discounted total after validating all inputs."""
    if discount < 0 or discount > 1:
        raise ValueError("discount must be a percentage between 0 and 1")

    if any(price < 0 for price in prices):
        raise ValueError("prices must be non-negative")

    subtotal = sum(prices)
    return subtotal * (1 - discount)
\end{lstlisting}

\section{Module-Sized Example}
Listing~\ref{lst:invoice} demonstrates how naming, docstrings, and dataclasses combine in a
realistic module that a billing service could ship.

\begin{lstlisting}[caption={Invoice summariser for a billing engine},label={lst:invoice}]
from __future__ import annotations

from dataclasses import dataclass
from datetime import datetime
from decimal import Decimal, ROUND_HALF_UP
from typing import Iterable


@dataclass(frozen=True)
class InvoiceLine:
    description: str
    quantity: int
    unit_price: Decimal

    def total(self) -> Decimal:
        """Return a rounded line total."""
        subtotal = self.unit_price * self.quantity
        return subtotal.quantize(Decimal("0.01"), rounding=ROUND_HALF_UP)


def summarise_invoice(lines: Iterable[InvoiceLine], issued_at: datetime) -> dict[str, object]:
    """Produce a payload for downstream email, PDF, or API consumers."""
    totals = [line.total() for line in lines]
    return {
        "issued_at": issued_at.isoformat(),
        "line_count": len(totals),
        "subtotal": sum(totals),
        "currency": "USD",
    }
\end{lstlisting}

\section{Scenario: Style Drift in a Release Crunch}
During a tight deadline, a payments team disabled the formatter because it produced merge
conflicts with an experimental branch.  Within two weeks, code reviews were bogged down by
spacing debates and subtle bugs introduced by inconsistent imports.  After the release,
they reinstated \texttt{pre-commit} hooks and added \texttt{ruff} checks to CI to prevent
future drift.

\begin{lstlisting}[caption={Detecting inconsistent formatting in a pull request},label={lst:style_drift}]
from __future__ import annotations

from pathlib import Path


def detect_style_drift(diff_root: Path) -> list[Path]:
    """Spot modules that violate formatting during emergency fixes."""
    offenders = []
    for path in diff_root.rglob("*.py"):
        if "\t" in path.read_text(encoding="utf-8"):
            offenders.append(path)
    return offenders
\end{lstlisting}

When the crunch was over, the team ran \texttt{detect\_style\_drift} against every hotfix branch,
opened follow-up issues for each offender, and cleared the debt before the next release.

\section{Summary}
Consistent style grows from agreed rules, automated enforcement, and shared examples that
clarify expectations.  Keep the tooling always-on so humans focus on logic rather than
indentation debates.

\begin{lstlisting}[caption={Summarising style violations for dashboards},label={lst:style_summary}]
from __future__ import annotations


def summarize_violations(violations: dict[str, int]) -> str:
    """Return a short sentence for incident reviews."""
    worst_offender = max(violations, key=violations.get)
    total = sum(violations.values())
    return f"{total} style issues detected; {worst_offender} needs a cleanup."
\end{lstlisting}

\section*{Exercises}
\begin{enumerate}
  \item Format a messy module manually, then re-run \texttt{black}.  Note every change the
  formatter made and decide which rules you would not have caught.
  \item Using Listing~\ref{lst:bad_style}, annotate each readability issue and propose a
  rule that would catch it automatically.
  \item Configure \texttt{ruff} to enforce import sorting.  Break the rule intentionally and
  observe the linter output.
  \item Pair with a teammate and review a 200-line pull request.  Track how often naming or
  formatting slowed understanding.  Compare notes and create a shared checklist.
  \item Extend your team's style guide with two domain-specific conventions (e.g., how to
  name dataclass factories) and circulate the update.
  \item Refactor the snippet below into readable code with docstrings and typing.
  \begin{lstlisting}[caption={Exercise: refactor for clarity},label={lst:style_exercise_refactor}]
def calc(x, y):
    return sum([a * y for a in x if a > 0])
  \end{lstlisting}
  \item Evaluate a formatter configuration that lives in \texttt{pyproject.toml}.  Explain
  which defaults you overrode and why they matter to your domain.
\end{enumerate}

\chapter{Comments and Documentation}
\section{Chapter Overview}
Well-structured code minimises the need for comments, yet documentation remains essential
for sharing context, intent, and processes.

\begin{lstlisting}[caption={Tracking documentation freshness},label={lst:docs_overview}]
from __future__ import annotations

from dataclasses import dataclass
from datetime import datetime


@dataclass(slots=True)
class DocumentStatus:
    """Keep tabs on which guides need attention."""

    path: str
    last_reviewed: datetime


def needs_refresh(status: DocumentStatus, *, months: int = 6) -> bool:
    """Flag documents that have gone stale."""
    delta = datetime.now(tz=status.last_reviewed.tzinfo) - status.last_reviewed
    return delta.days > months * 30
\end{lstlisting}

\section{Docstrings that Educate}
Docstrings should answer three questions: what does this item do, what are its inputs and
outputs, and when does it raise exceptions?  Follow the "one-line summary + details"
format so tools like Sphinx can render the text neatly.

\begin{lstlisting}[caption={Docstring capturing behaviour and failure modes},label={lst:docs_docstring}]
from __future__ import annotations


def parse_currency(value: str) -> int:
    """Return cents for a currency string; raise ValueError on malformed input."""
    if not value.startswith("$"):
        raise ValueError("Currency must start with a dollar sign.")
    dollars, cents = value[1:].split(".")
    return int(dollars) * 100 + int(cents)
\end{lstlisting}

\section{Inline Comments for Intent}
Inline comments should explain \emph{why}, not \emph{what}.  They are ideal for noting
workarounds, referencing bug tickets, or clarifying domain constraints.  Delete them once
they become obsolete to avoid misleading readers.

\begin{lstlisting}[caption={Inline comments that explain surprising workarounds},label={lst:docs_inline}]
from __future__ import annotations

from datetime import UTC, datetime


def next_billing_window(now: datetime) -> datetime:
    """Return the start of the next billing window."""
    window = now.astimezone(UTC).replace(minute=0, second=0, microsecond=0)
    # Stripe batches reconcile at 5-minute intervals, so align to that cadence.
    return window.replace(minute=(window.minute // 5 + 1) * 5)
\end{lstlisting}

\section{Project Documentation Stack}
Combine lightweight Markdown guides with generated API references.  Example structure:
\texttt{README.md} for onboarding, \texttt{docs/architecture.md} for high-level design,
Sphinx or MkDocs for API references, and runbooks for deployment and incident response.

\begin{lstlisting}[caption={Generating a docs index programmatically},label={lst:docs_stack}]
from __future__ import annotations

from pathlib import Path


def ensure_docs_index(root: Path) -> None:
    """Create a landing page that references every Markdown guide."""
    guides = sorted(root.glob("*.md"))
    body = "\n".join(f"- [{guide.stem}]({guide.name})" for guide in guides)
    (root / "index.md").write_text(body, encoding="utf-8")
\end{lstlisting}

\section{Code Example: Documentation-Driven Development}
Listing~\ref{lst:documented_function} demonstrates an API that uses docstrings and inline
comments to highlight non-obvious behaviour.

\begin{lstlisting}[caption={Docstring and inline comment illustrating intent},label={lst:documented_function}]
from __future__ import annotations

from datetime import UTC, datetime


def anonymise_timestamp(raw_timestamp: str) -> datetime:
    """Return an hourly bucket so analytics cannot deanonymise individual events."""
    timestamp = datetime.fromisoformat(raw_timestamp).astimezone(UTC)
    # We discard minutes/seconds to comply with the analytics retention policy.
    return timestamp.replace(minute=0, second=0, microsecond=0)
\end{lstlisting}

\section{Scenario: Docs Saved a Launch}
A healthcare startup had to demonstrate auditability to regulators.  Because the team
maintained up-to-date architecture diagrams and onboarding walkthroughs, they produced the
requested evidence in hours instead of weeks.  The documentation discipline paid off even
though the company had never planned for a government review.

\begin{lstlisting}[caption={Collecting evidence links quickly},label={lst:docs_audit}]
from __future__ import annotations

from pathlib import Path


def gather_audit_artifacts(root: Path) -> list[Path]:
    """Return runbooks and diagrams needed during regulated reviews."""
    return [path for path in root.rglob("*") if path.suffix in {".md", ".drawio"}]
\end{lstlisting}

\section{Summary}
Documentation succeeds when it explains intent, evolves with the system, and exists at
multiple layers—from docstrings to architecture guides and runbooks.

\begin{lstlisting}[caption={Summarising documentation gaps},label={lst:docs_summary}]
from __future__ import annotations


def summarize_gaps(gaps: dict[str, list[str]]) -> str:
    """Turn gap analysis into a concise statement for leadership."""
    missing = ", ".join(sorted(gaps))
    return f"Docs need updates for {missing}; owners assigned."
\end{lstlisting}

\section*{Exercises}
\begin{enumerate}
  \item Choose an undocumented function and write a docstring that follows the style in
  Listing~\ref{lst:documented_function}.
  \item Create an inline comment explaining a surprising line of code, then refactor until
  the comment is unnecessary.
  \item Generate API documentation with Sphinx for a small module.  What metadata is
  missing?
  \item Draft a runbook for a scheduled job.  Include inputs, expected outputs, and failure
  procedures.
  \item Interview a new hire about confusing documentation gaps and log the findings as
  GitHub issues.
  \item Use the snippet below to explain how you would document the edge cases.
  \begin{lstlisting}[caption={Exercise: document behaviour},label={lst:docs_ex_doc}]
def compute(offset: int) -> int:
    if offset < 0:
        return 0
    return offset * 2
  \end{lstlisting}
  \item Sketch a documentation stack diagram that shows how README files, architecture
  records, and runbooks link to each other.  Identify who updates each artefact.
\end{enumerate}

\chapter{Project Structure and Architecture}
\section{Chapter Overview}
A thoughtful structure reinforces boundaries and accelerates development.  This chapter
covers folder layout, layering, and refactoring strategies.

\begin{lstlisting}[caption={Capturing ownership of directories},label={lst:structure_overview}]
from __future__ import annotations

from dataclasses import dataclass


@dataclass(slots=True)
class ModuleBoundary:
    """Describe how code is grouped and who maintains it."""

    path: str
    owner_team: str
    depends_on: list[str]
\end{lstlisting}

\section{Repository Layout}
Place application code in \texttt{src/}, tests in \texttt{tests/}, documentation in
\texttt{docs/}, and experimentation in \texttt{experiments/}.  The goal is to make every
directory self-explanatory.

\begin{lstlisting}[caption={Predictable repository layout},label={lst:layout}]
python-best-practices/
|-- pyproject.toml
|-- README.md
|-- src/
|   `-- billing/
|       |-- __init__.py
|       |-- invoices.py
|       `-- repositories.py
|-- tests/
|   `-- test_invoices.py
`-- docs/
    `-- architecture.md
\end{lstlisting}

\section{Architectural Layers}
Layered architectures separate domain logic from infrastructure.  A typical service
contains domain models, repositories, adapters, and entry points (CLI, HTTP, messaging).
Each layer depends inward, making it easier to test and swap components.

\begin{lstlisting}[caption={Defining explicit boundaries between layers},label={lst:structure_layers}]
from __future__ import annotations

from dataclasses import dataclass


@dataclass(slots=True)
class Command:
    """DTO that carries user intent from the presentation layer."""

    user_id: str
    action: str
\end{lstlisting}

\section{Scenario: Refactoring a Monolith}
A logistics company inherited a single-file monolith that processed shipments, handled
notifications, and exposed a CLI.  By splitting responsibilities into modules that matched
their architecture diagram, they unlocked parallel development.  Teams could now work on
inventory, billing, or reporting without stepping on each other's toes.

\begin{lstlisting}[caption={Coarse-grained monolith vs refactored entry point},label={lst:structure_monolith}]
def process_all() -> None:
    """Monolithic script that mixes unrelated behaviours."""
    load_inventory()
    notify_customers()
    run_cli()
\end{lstlisting}

\begin{lstlisting}[caption={Modular orchestration after refactor},label={lst:structure_refactor}]
from __future__ import annotations


def orchestrate() -> None:
    """Dispatch into modules that mirror architecture boundaries."""
    inventory.run_pipeline()
    notifications.send_digest()
    cli.run()
\end{lstlisting}

\section{Code Example: Service Layer Pattern}
Listing~\ref{lst:service_layer} shows a service that decouples use cases from persistence.

\begin{lstlisting}[caption={Service layer separating domain and persistence},label={lst:service_layer}]
from __future__ import annotations

from dataclasses import dataclass
from typing import Protocol
from uuid import UUID


class UserGateway(Protocol):
    def deactivate(self, user_id: UUID) -> None: ...


class UserNotFoundError(RuntimeError):
    """Raised when a user identifier does not exist."""


@dataclass(slots=True)
class UserService:
    gateway: UserGateway

    def deactivate_user(self, user_id: UUID) -> None:
        try:
            self.gateway.deactivate(user_id)
        except UserNotFoundError as exc:
            raise ValueError(f"Unknown user {user_id}") from exc
\end{lstlisting}

\section{Summary}
Let the directory tree mirror the architecture, keep responsibilities separated by layers,
and refactor incrementally so structure evolves alongside the product.

\begin{lstlisting}[caption={Asserting architectural contracts in tests},label={lst:structure_summary}]
from __future__ import annotations


def assert_dependency_rule(module: str, allowed: set[str], imports: set[str]) -> None:
    """Guardrails to ensure layers only depend inward."""
    illegal = imports - allowed
    if illegal:
        raise AssertionError(f"{module} cannot import {illegal}")
\end{lstlisting}

\section*{Exercises}
\begin{enumerate}
  \item Draw an architecture diagram of your current project.  How well does the folder
  structure reflect that diagram?
  \item Apply the service-layer pattern from Listing~\ref{lst:service_layer} to another use
  case (e.g., invoice approval).
  \item Identify a module that performs both domain and persistence work.  Split it into two
  files and note how tests change.
  \item Design a strategy for handling cross-cutting concerns (validation, logging) without
  violating layering.
  \item Propose a directory naming convention for experiments or spikes and document it.
  \item Use the pseudo-monolith snippet in Listing~\ref{lst:structure_monolith} as input and
  sketch a refactoring plan that introduces adapters, services, and repositories.
\end{enumerate}
\begin{lstlisting}[caption={Validating repository layout in CI},label={lst:structure_layout}]
from __future__ import annotations

from pathlib import Path


def validate_layout(root: Path) -> None:
    """Ensure expected top-level directories exist."""
    required = {"src", "tests", "docs"}
    missing = required - {path.name for path in root.iterdir() if path.is_dir()}
    if missing:
        raise SystemExit(f"Missing required directories: {sorted(missing)}")
\end{lstlisting}

\chapter{Dependencies and Environments}
\section{Chapter Overview}
Environment drift is a leading cause of ``works on my machine'' bugs.  This chapter explains
how to manage dependencies systematically.

\begin{lstlisting}[caption={Capturing environment information},label={lst:deps_overview}]
from __future__ import annotations

from dataclasses import dataclass


@dataclass(slots=True)
class EnvironmentReport:
    """Track interpreter, dependency lock, and platform data."""

    python_version: str
    lockfile_hash: str
    platform: str
\end{lstlisting}

\section{Virtual Environments Everywhere}
Use \texttt{python -m venv}, \texttt{uv}, or \texttt{pipenv} to isolate dependencies per
project.  Document activation commands and add helper scripts that install tooling in one
step.

\begin{lstlisting}[caption={Bootstrapping a virtual environment programmatically},label={lst:deps_venv}]
from __future__ import annotations

import subprocess
from pathlib import Path


def create_environment(venv_dir: Path) -> None:
    """Create and populate a deterministic virtual environment."""
    subprocess.run(["python", "-m", "venv", str(venv_dir)], check=True)
    subprocess.run([str(venv_dir / "Scripts" / "python"), "-m", "pip", "install", "-U", "pip"], check=True)
\end{lstlisting}

\section{Pinning and Locking}
Declare dependencies in \texttt{pyproject.toml} and generate lock files with \texttt{uv
lock} or \texttt{pip-compile}.  Commit the lock files so CI and production match local
development.

\begin{lstlisting}[caption={Verifying lock files before deployment},label={lst:deps_lock}]
from __future__ import annotations

import hashlib
from pathlib import Path


def lock_hash(path: Path) -> str:
    """Return the hash used to verify reproducible installs."""
    return hashlib.sha256(path.read_bytes()).hexdigest()
\end{lstlisting}

\section{Example: pyproject Snippet}
\begin{lstlisting}[caption={pyproject snippet with optional dependencies},label={lst:pyproject}]
[project]
name = "billing-service"
version = "0.4.0"
requires-python = ">=3.12"
dependencies = [
  "pydantic>=2.5",
  "ruff==0.6.8",
]

[project.optional-dependencies]
dev = ["pytest>=8.3", "mypy>=1.11", "hypothesis>=6.99"]
\end{lstlisting}

\section{Scenario: Missing Dependency in Production}
An e-commerce team relied on a globally installed CLI utility that was present on their
laptops but not on CI servers.  A deployment failed hours before Black Friday.  The root
cause was a missing dependency specification.  After the incident, they moved all tooling
into \texttt{pyproject.toml} and added smoke tests to CI that bootstrap environments from
scratch.

\begin{lstlisting}[caption={Ensuring tools are declared explicitly},label={lst:deps_missing}]
from __future__ import annotations

from importlib import metadata


def assert_tool_declared(package: str) -> None:
    """Fail fast when required tooling is absent."""
    try:
        metadata.version(package)
    except metadata.PackageNotFoundError as exc:
        raise RuntimeError(f"Missing dependency: {package}") from exc
\end{lstlisting}

\section{Summary}
Reliable deployments depend on isolated environments, locked dependency graphs, and CI
pipelines that recreate installations exactly as production does.

\begin{lstlisting}[caption={Summarising dependency health for dashboards},label={lst:deps_summary}]
from __future__ import annotations


def summarize_dependencies(outdated: list[str]) -> str:
    """Provide leadership-friendly summary of update backlog."""
    if not outdated:
        return "All dependencies are current."
    return f"{len(outdated)} packages require upgrades: {', '.join(outdated[:5])}..."
\end{lstlisting}

\section*{Exercises}
\begin{enumerate}
  \item Regenerate your lock file using \texttt{uv} or \texttt{pip-compile}.  How many
  transitive dependencies changed?
  \item Write a shell script that recreates the project environment from scratch.  Run it on
  a clean machine or container.
  \item Audit your repository for globally installed tools.  Add them to \texttt{pyproject}
  extras.
  \item Introduce a deliberate dependency conflict and run \texttt{pip check} to observe the
  failure.
  \item Design a process for handling emergency dependency upgrades triggered by security
  advisories.
  \item Extend Listing~\ref{lst:deps_lock} to compare hash values between CI and production
  deployments.  Explain how you would alert on mismatches.
\end{enumerate}
\begin{lstlisting}[caption={Install dependencies from the defined groups},label={lst:deps_install}]
from __future__ import annotations

import subprocess


def install_group(group: str) -> None:
    """Install an optional dependency group from pyproject metadata."""
    subprocess.run(["uv", "pip", "install", f".[{group}]"], check=True)
\end{lstlisting}

\chapter{Type Hints and Static Analysis}
\section{Chapter Overview}
Type hints capture intent and enable automated reasoning.  Combined with static analyzers,
they prevent entire classes of bugs.

\begin{lstlisting}[caption={Capturing annotation coverage},label={lst:types_overview}]
from __future__ import annotations


def annotation_coverage(total_objects: int, typed_objects: int) -> float:
    """Return the proportion of typed callables and modules."""
    return round(typed_objects / total_objects, 2)
\end{lstlisting}

\section{Why Annotate}
Annotations clarify contracts for both humans and tools.  They document mutability,
optional values, and expected shapes.  Their primary benefit is catching mismatches at
development time rather than in production.

\begin{lstlisting}[caption={Capturing optional state via annotations},label={lst:types_optional}]
from __future__ import annotations

from dataclasses import dataclass
from datetime import datetime


@dataclass(slots=True)
class InvoiceDraft:
    """Represent an invoice that may or may not be finalised."""

    issued_at: datetime | None
    total: int


def finalize(draft: InvoiceDraft, *, issued_at: datetime) -> InvoiceDraft:
    """Produce a finalized invoice with explicit timestamps."""
    return InvoiceDraft(issued_at=issued_at, total=draft.total)
\end{lstlisting}

\section{Protocols and Dataclasses}
Protocols describe structural typing, allowing you to express behaviour without inheritance.
Dataclasses handle boilerplate while keeping types explicit.

\begin{lstlisting}[caption={Scheduling interface using protocols},label={lst:typed_interface}]
from __future__ import annotations

from collections.abc import Iterable
from dataclasses import dataclass
from datetime import datetime, timedelta
from typing import Protocol


class Schedulable(Protocol):
    priority: int

    def execute(self) -> None: ...


@dataclass(slots=True)
class ScheduledJob:
    job: Schedulable
    eta: datetime


def schedule_jobs(jobs: Iterable[Schedulable], *, start: datetime) -> list[ScheduledJob]:
    """Order jobs by priority and assign execution windows."""
    sorted_jobs = sorted(jobs, key=lambda job: job.priority, reverse=True)
    schedule: list[ScheduledJob] = []
    for offset, job in enumerate(sorted_jobs):
        schedule.append(ScheduledJob(job=job, eta=start + timedelta(minutes=offset)))
    return schedule
\end{lstlisting}

\section{Static Analysis Workflow}
Run \texttt{mypy} or \texttt{pyright} in CI alongside \texttt{ruff}.  Treat warnings as
actionable feedback.  If a suppression is necessary, add a comment explaining why.

\begin{lstlisting}[caption={Running mypy with strict settings programmatically},label={lst:types_mypy}]
from __future__ import annotations

import subprocess


def run_type_checks() -> None:
    """Invoke mypy with the strict profile used in CI."""
    subprocess.run(["mypy", "--strict", "src"], check=True)
\end{lstlisting}

\section{Scenario: Type Hints Caught a Regression}
A data platform refactored a parser to support new file formats.  Type hints and \texttt{mypy}
flagged that the new code returned \texttt{None} in a branch where downstream code expected
a \texttt{str}.  The regression never reached staging.

\begin{lstlisting}[caption={Regression prevented by precise return types},label={lst:types_regression}]
from __future__ import annotations


def parse_status(raw: str) -> str:
    """Return a validated status string."""
    match raw.lower():
        case "paid" | "pending":
            return raw.lower()
        case _:
            raise ValueError(f"Unknown status: {raw}")
\end{lstlisting}

\section{Summary}
Treat annotations as executable documentation, layer protocols and dataclasses to express
behaviour, and rely on static analyzers for immediate feedback during reviews and CI.

\begin{lstlisting}[caption={Summarising type-check results},label={lst:types_summary}]
from __future__ import annotations


def summarize_type_errors(errors: list[str]) -> str:
    """Condense mypy output for leadership updates."""
    if not errors:
        return "Type checks clean; no regressions."
    return f"{len(errors)} type errors remain; triage blockers first."
\end{lstlisting}

\section*{Exercises}
\begin{enumerate}
  \item Add type hints to an untyped module and run \texttt{mypy}.  Document each warning
  and the fix.
  \item Convert a nominal interface into a \texttt{Protocol} and note how it simplifies
  testing.
  \item Configure \texttt{pyright} or \texttt{mypy} in \texttt{pre-commit}.  Measure the
  runtime impact.
  \item Identify a place where you rely on \texttt{Any}.  Replace it with a precise type.
  \item Debate with your team when to use \texttt{TypedDict}, \texttt{dataclass}, or
  \texttt{attrs}.
\end{enumerate}

\chapter{Testing and Quality Assurance}
\section{Chapter Overview}
Testing gives confidence that code behaves as designed.  This chapter covers strategy,
tooling, and common pitfalls.

\begin{lstlisting}[caption={Capturing test suite health},label={lst:testing_overview}]
from __future__ import annotations


def test_success_rate(passed: int, total: int) -> float:
    """Return percentage of passing tests across the suite."""
    return round((passed / total) * 100, 2)
\end{lstlisting}

\section{Testing Pyramid}
Think of tests as layers:
\begin{itemize}
  \item Unit tests: fast, pure logic.
  \item Integration tests: real dependencies (databases, APIs).
  \item End-to-end tests: simulate user workflows.
\end{itemize}
Balance investment so feedback stays fast but coverage remains meaningful.

\begin{lstlisting}[caption={Defining representative tests per layer},label={lst:testing_layers}]
from __future__ import annotations


def classify_test(duration_seconds: float) -> str:
    """Return the pyramid layer a test belongs to."""
    if duration_seconds < 0.2:
        return "unit"
    if duration_seconds < 5:
        return "integration"
    return "end_to_end"
\end{lstlisting}

\section{Pytest Example}

\begin{lstlisting}[caption={Pytest module with fixtures and time control},label={lst:pytest}]
from datetime import UTC, datetime
from decimal import Decimal

import pytest

from billing.invoices import InvoiceLine, summarise_invoice


@pytest.fixture
def sample_lines() -> list[InvoiceLine]:
    return [
        InvoiceLine(description="Subscription", quantity=1, unit_price=Decimal("49.00")),
        InvoiceLine(description="Support", quantity=3, unit_price=Decimal("15.00")),
    ]


def test_summarise_invoice_counts_lines(sample_lines, freezer):
    freezer.move_to("2024-05-01T12:00:00Z")
    payload = summarise_invoice(sample_lines, issued_at=datetime.now(tz=UTC))
    assert payload["line_count"] == 2
    assert payload["subtotal"] == Decimal("94.00")
\end{lstlisting}

\section{Anti-Pattern: Untestable Code}

\begin{lstlisting}[caption={Deeply coupled function that defies testing},label={lst:untestable}]
import os
import shutil
from uuid import uuid4

import boto3


def sync_everything() -> None:
    client = boto3.client("s3")
    for path in os.listdir("/tmp"):
        if path.endswith(".json"):
            client.upload_file(path, "analytics-bucket", f"imports/{uuid4()}")
            shutil.move(path, "/archive")
\end{lstlisting}

Break the function into collaborators that can be faked in tests.  Inject the S3 client and
filesystem wrapper rather than instantiating them inline.

\section{Scenario: Catching a Production Bug}
A fintech company reproduced a production bug by writing a failing pytest that mimicked the
customer's inputs.  After the fix, the test joined the regression suite, ensuring the issue
could never return silently.

\begin{lstlisting}[caption={Regression test distilled from an incident},label={lst:testing_regression}]
from __future__ import annotations

from billing.reports import render_statement


def test_regression_missing_currency() -> None:
    """Lock in the fix for an incident triggered by malformed input."""
    payload = render_statement(currency="EUR", amount=0)
    assert "currency" in payload
\end{lstlisting}

\section{Summary}
Healthy suites blend fast unit tests with targeted integrations, emphasise testable code
structure, and capture regressions the moment they appear in CI pipelines.

\begin{lstlisting}[caption={Summarising flaky tests},label={lst:testing_summary}]
from __future__ import annotations


def summarize_flakes(failing: list[str]) -> str:
    """Produce a string for Slack alerts describing flaky tests."""
    if not failing:
        return "No flaky tests detected this week."
    return f"Flaky tests: {', '.join(sorted(failing))}"
\end{lstlisting}

\section*{Exercises}
\begin{enumerate}
  \item Rewrite Listing~\ref{lst:untestable} into a testable design with injected
dependencies.
  \item Add a property-based test using \texttt{hypothesis} for a function with tricky input
domains.
  \item Introduce contract tests for an external API your service consumes.
  \item Configure coverage reporting in CI.  Identify untested modules.
  \item Design a rollback drill: break a test intentionally, fix it, and measure the time it
takes.
  \item Extend Listing~\ref{lst:testing_layers} so it emits Prometheus metrics for each test
  classification.
  \item Pair-review a flaky test.  Capture the timeline of failures, hypothesise the root
  cause, and document the outcome in the regression suite.
\end{enumerate}
\begin{lstlisting}[caption={Improved design with injectable collaborators},label={lst:testing_good}]
from __future__ import annotations

from collections.abc import Iterable
from uuid import uuid4


def sync_documents(paths: Iterable[str], client, mover) -> None:
    """Delegate I/O to collaborators so tests can substitute them."""
    for path in paths:
        if path.endswith(".json"):
            client.upload_file(path, "analytics-bucket", f"imports/{uuid4()}")
            mover.move(path, "/archive")
\end{lstlisting}

\chapter{Error Handling and Logging}
\section{Chapter Overview}
Error handling and observability are inseparable.  Without clear error pathways and
structured logs, diagnosing issues becomes guesswork.

\begin{lstlisting}[caption={Categorising errors for alerting},label={lst:error_overview}]
from __future__ import annotations


def categorize_error(exc: Exception) -> str:
    """Return an alert bucket given an exception instance."""
    name = exc.__class__.__name__
    if "Timeout" in name:
        return "transient"
    if "Validation" in name:
        return "user"
    return "system"
\end{lstlisting}

\section{Domain-Specific Exceptions}
Define exceptions that reflect your business language so logs and alerts remain meaningful.
Listing~\ref{lst:service_layer} already illustrated raising domain errors from services.

\begin{lstlisting}[caption={Expressive domain-level exception hierarchy},label={lst:error_domain}]
from __future__ import annotations


class BillingError(RuntimeError):
    """Base error for billing workflows."""


class PaymentInstrumentDeclined(BillingError):
    """Raised when the processor rejects a card."""
\end{lstlisting}

\section{Structured Logging}

\begin{lstlisting}[caption={Structured logging with contextual metadata},label={lst:logging}]
import logging
from logging.config import dictConfig


def configure_logging() -> None:
    dictConfig(
        {
            "version": 1,
            "formatters": {
                "json": {
                    "format": "%(asctime)s %(levelname)s %(name)s %(message)s",
                    "class": "pythonjsonlogger.jsonlogger.JsonFormatter",
                }
            },
            "handlers": {
                "stderr": {
                    "class": "logging.StreamHandler",
                    "stream": "ext://sys.stderr",
                    "formatter": "json",
                }
            },
            "root": {"level": "INFO", "handlers": ["stderr"]},
        }
    )


logger = logging.getLogger("billing.events")
logger = logging.LoggerAdapter(logger, extra={"service": "billing"})


def log_payment_event(invoice_id: str, amount: str) -> None:
    logger.info("payment_completed", extra={"invoice_id": invoice_id, "amount": amount})
\end{lstlisting}

\section{Bad vs Good Error Handling}

\begin{lstlisting}[caption={Anti-pattern: swallowing every exception},label={lst:bare_except}]
def run_job() -> None:
    try:
        do_work()
        persist_results()
    except Exception:
        return None
\end{lstlisting}

\begin{lstlisting}[caption={Improved version with explicit logging},label={lst:handled_error}]
from __future__ import annotations

import logging


def run_job(logger: logging.Logger) -> None:
    try:
        do_work()
        persist_results()
    except NetworkError:
        logger.warning("Transient network issue; job will retry")
        raise
    except PersistenceError:
        logger.error("Failed to persist results", exc_info=True)
        raise
\end{lstlisting}

\section{Scenario: Incident Response}
During an outage, the on-call engineer used structured logs to correlate failed HTTP
requests with a downstream dependency.  Because each log entry included contextual fields
such as user ID and feature flag state, the team restored service within minutes.

\begin{lstlisting}[caption={Extracting timeline data from logs},label={lst:error_incident}]
from __future__ import annotations

from datetime import datetime


def build_incident_timeline(entries: list[dict[str, str]]) -> list[str]:
    """Return a human-readable sequence of incident events."""
    sorted_entries = sorted(entries, key=lambda entry: entry["timestamp"])
    return [
        f"{datetime.fromisoformat(entry['timestamp'])}: {entry['message']}"
        for entry in sorted_entries
    ]
\end{lstlisting}

\section{Summary}
Name your exceptions after real business events, log them with structure and context, and
practice incident drills so the information proves useful under pressure.

\begin{lstlisting}[caption={Summarising alert volume},label={lst:error_summary}]
from __future__ import annotations


def summarize_alerts(alerts: dict[str, int]) -> str:
    """Condense alert tallies into a planning sentence."""
    worst = max(alerts, key=alerts.get)
    return f"{sum(alerts.values())} alerts fired; {worst} noisy category."
\end{lstlisting}

\section*{Exercises}
\begin{enumerate}
  \item Replace bare \texttt{except} blocks in your codebase with targeted exceptions.
  \item Configure structured logging locally and confirm that observability dashboards parse
  the fields.
  \item Write a chaos test that injects failures into a dependency and verify the logs are
  actionable.
  \item Create a "runbook" log template that every subsystem uses for critical events.
  \item Review a recent incident report and map each detection or mitigation step to logging
  improvements.
  \item Extend Listing~\ref{lst:error_incident} so it groups events by feature flag and
  produces a Markdown report suitable for postmortems.
\end{enumerate}

\chapter{Configuration and Secrets Management}
\section{Chapter Overview}
Mismanaged configuration causes painful outages and leaks.  Treat configuration as data,
with types, validation, and lifecycle controls.

\begin{lstlisting}[caption={Tracking configuration versions},label={lst:config_overview}]
from __future__ import annotations

from dataclasses import dataclass
from datetime import datetime


@dataclass(slots=True)
class ConfigVersion:
    """Capture when and why configuration changed."""

    checksum: str
    applied_at: datetime
    change_reason: str
\end{lstlisting}

\section{Typed Configuration Loader}

\begin{lstlisting}[caption={Single entry point for configuration},label={lst:config_loader}]
from __future__ import annotations

from pathlib import Path

from pydantic import BaseSettings, Field


class Settings(BaseSettings):
    database_url: str = Field(..., env="DATABASE_URL")
    log_level: str = Field("INFO", env="LOG_LEVEL")
    feature_flag_path: Path = Field(Path("/etc/app/features.json"), env="FEATURE_FLAG_PATH")

    class Config:
        env_file = ".env"


def load_settings() -> Settings:
    return Settings()
\end{lstlisting}

\section{Secrets Discipline}
Never commit credentials.  Use environment variables, secret stores, or orchestrator
injections.  Run secret scanners like \texttt{trufflehog} in CI to catch mistakes early.

\begin{lstlisting}[caption={Validating secrets at startup},label={lst:config_secrets}]
from __future__ import annotations

import os


def ensure_secret(name: str) -> str:
    """Raise an error if a required secret is absent."""
    if value := os.getenv(name):
        return value
    raise RuntimeError(f"Missing secret: {name}")
\end{lstlisting}

\section{Scenario: Leaked Sandbox Key}
A developer committed sandbox credentials that looked harmless.  A malicious user discovered
the repo and pivoted into production through a little-known trust relationship.  After the
incident, the company adopted automated scanning, rotated all credentials, and required
code reviews for configuration files.

\begin{lstlisting}[caption={Scanning history for secrets},label={lst:config_leak}]
from __future__ import annotations

import subprocess


def scan_history() -> None:
    """Run a lightweight history scan before pushing."""
    subprocess.run(["git", "secrets", "--scan-history"], check=True)
\end{lstlisting}

\section{Summary}
Centralise configuration behind typed loaders, enforce secret hygiene, and document rotation
and validation processes so deployments remain predictable.

\begin{lstlisting}[caption={Summarising config coverage},label={lst:config_summary}]
from __future__ import annotations


def summarize_configs(configs: dict[str, bool]) -> str:
    """Report which services have adopted typed loaders."""
    adopted = [name for name, typed in configs.items() if typed]
    return f"{len(adopted)} services typed config; roll out remaining soon."
\end{lstlisting}

\section*{Exercises}
\begin{enumerate}
  \item Refactor configuration access to funnel through a typed loader similar to
  Listing~\ref{lst:config_loader}.
  \item Add secret scanning to your CI pipeline and test it by committing a fake key.
  \item Document the rotation process for every secret your service relies on.
  \item Implement runtime validation that refuses to start when required environment
  variables are missing.
  \item Design a feature flag strategy (file-based, database, or SaaS) and evaluate trade-offs.
  \item Extend Listing~\ref{lst:config_secrets} so it fetches secrets from your team's vault
  provider with caching and metrics hooks.
\end{enumerate}

\chapter{Performance and Scalability}
\section{Chapter Overview}
Optimise only after measuring.  This chapter describes profiling, data structure choices,
and concurrency strategies.

\begin{lstlisting}[caption={Capturing performance regressions},label={lst:perf_overview}]
from __future__ import annotations


def regression_ratio(old_ms: float, new_ms: float) -> float:
    """Compute slowdown factor for benchmarking dashboards."""
    return round(new_ms / old_ms, 2)
\end{lstlisting}

\section{Profiling Before Optimising}

\begin{lstlisting}[caption={CProfile harness for a reporting job},label={lst:profiling}]
import cProfile
import pstats

from billing.reports import build_monthly_report


def profile_report() -> None:
    with cProfile.Profile() as profiler:
        build_monthly_report(project_id="alpha")
    stats = pstats.Stats(profiler)
    stats.sort_stats("cumulative").print_stats(15)
\end{lstlisting}

\section{Data Structures}
Choose data structures that align with access patterns: \texttt{deque} for queues,
\texttt{set} for membership tests, \texttt{heapq} for priority queues, and \texttt{itertools}
for streaming.  Avoid expensive operations such as repeated string concatenation inside
loops.

\begin{lstlisting}[caption={Selecting the right container},label={lst:perf_data_structures}]
from __future__ import annotations

from collections import deque


def process_queue(items: list[int]) -> deque[int]:
    """Convert lists into deques when pop-left operations dominate."""
    queue: deque[int] = deque(items)
    while queue and queue[0] < 0:
        queue.popleft()
    return queue
\end{lstlisting}

\section{Concurrency Choices}
Use \texttt{asyncio} for I/O-bound services, threads for blocking I/O when async refactors
are impractical, and processes or native extensions for CPU-bound workloads.  Protect shared
state with locks or adopt immutable structures.

\begin{lstlisting}[caption={Running I/O-bound tasks concurrently},label={lst:perf_concurrency}]
from __future__ import annotations

import asyncio


async def fetch_record(record_id: str) -> str:
    """Pretend to call an external service."""
    await asyncio.sleep(0.1)
    return record_id


async def fetch_many(ids: list[str]) -> list[str]:
    """Drive multiple calls concurrently for throughput."""
    return await asyncio.gather(*(fetch_record(item) for item in ids))
\end{lstlisting}

\section{Scenario: Scaling a Report Generator}
A consulting firm profiled its overnight PDF generator and discovered 80\% of time spent in
template rendering.  By caching parsed templates and using \texttt{ThreadPoolExecutor} for
I/O-bound API calls, they cut runtime from hours to minutes.

\begin{lstlisting}[caption={Caching template compilation},label={lst:perf_template_cache}]
from __future__ import annotations

from functools import lru_cache


@lru_cache(maxsize=32)
def compile_template(name: str) -> str:
    """Simulate the expensive part of rendering."""
    return name.upper()
\end{lstlisting}

\section{Summary}
Let measurements guide every optimisation, choose data structures intentionally, and pick
concurrency models that match the workload's constraints.

\begin{lstlisting}[caption={Summarising optimisation impact},label={lst:perf_summary}]
from __future__ import annotations


def summarize_speedups(results: dict[str, float]) -> str:
    """Generate a concise update for stakeholders."""
    fastest = min(results, key=results.get)
    return f"{fastest} achieved best runtime at {results[fastest]:.2f}s."
\end{lstlisting}

\section*{Exercises}
\begin{enumerate}
  \item Profile a slow task using \texttt{cProfile}.  Optimise the top offending function
  and measure the difference.
  \item Rewrite a loop that concatenates strings with a more efficient approach using
  \texttt{"".join()}.
  \item Implement both threaded and async versions of a simple downloader.  Compare code
  complexity and throughput.
  \item Build a benchmark harness that records baseline metrics for a key workflow.
  \item Identify a cacheable computation and design an invalidation strategy.
  \item Extend Listing~\ref{lst:perf_concurrency} so it enforces a concurrency limit and
  records latency histograms.
\end{enumerate}

\chapter{Security for Python Developers}
\section{Chapter Overview}
Security is everyone's responsibility.  This chapter addresses dependency hygiene, input
validation, and safe filesystem access.

\begin{lstlisting}[caption={Capturing CVE exposure},label={lst:security_overview}]
from __future__ import annotations


def compute_vulnerability_rate(vulnerable: int, total: int) -> float:
    """Return proportion of dependencies with known issues."""
    return round(vulnerable / total, 2)
\end{lstlisting}

\section{Dependency Hygiene}
Run \texttt{pip-audit} or \texttt{Safety} regularly.  Automate updates with Dependabot or
Renovate so vulnerabilities are patched quickly.

\begin{lstlisting}[caption={Invoking pip-audit from Python},label={lst:security_audit}]
from __future__ import annotations

import subprocess


def run_pip_audit() -> None:
    """Fail the build when vulnerabilities exist."""
    subprocess.run(["pip-audit", "--strict"], check=True)
\end{lstlisting}

\section{Input Validation}
Use Pydantic models, \texttt{argparse}, or custom validators to treat all external input as
untrusted.  Explicitly constrain formats and ranges.

\begin{lstlisting}[caption={Validating API payloads},label={lst:security_validation}]
from __future__ import annotations

from pydantic import BaseModel, Field


class PaymentRequest(BaseModel):
    """Trusted structure for inbound payloads."""

    account_id: str = Field(min_length=8, max_length=32)
    amount_cents: int = Field(gt=0, lt=100_000_00)
\end{lstlisting}

\section{Safe File Handling}
\begin{lstlisting}[caption={Safe path resolution guards against traversal},label={lst:safe_path}]
from __future__ import annotations

from pathlib import Path


def read_report(report_name: str, base_dir: Path) -> str:
    reports_dir = base_dir / "reports"
    reports_dir.mkdir(exist_ok=True)
    candidate = (reports_dir / report_name).resolve()
    if reports_dir.resolve() not in candidate.parents:
        raise PermissionError("Illegal path traversal attempt detected")
    return candidate.read_text(encoding="utf-8")
\end{lstlisting}

\section{Scenario: Dependency Supply-Chain Attack}
An open-source package added a malicious post-install hook.  Because the team pinned
versions and used hash-checking installers, the compromised release never reached
production.  They then set up signed releases and mirrored dependencies internally.

\begin{lstlisting}[caption={Validating package hashes},label={lst:security_hashes}]
from __future__ import annotations

import hashlib
from pathlib import Path


def verify_package(path: Path, expected_hash: str) -> None:
    """Raise when downloaded artifacts do not match expectation."""
    digest = hashlib.sha256(path.read_bytes()).hexdigest()
    if digest != expected_hash:
        raise RuntimeError("Package hash mismatch detected")
\end{lstlisting}

\section{Summary}
Keep dependencies patched, treat every external input as hostile, and harden filesystem
access to avoid trivial escalation paths.

\begin{lstlisting}[caption={Summarising security posture},label={lst:security_summary}]
from __future__ import annotations


def summarize_findings(findings: list[str]) -> str:
    """Condense security findings for exec summaries."""
    if not findings:
        return "No blocking security findings."
    critical = [item for item in findings if "critical" in item.lower()]
    return f"{len(findings)} findings ({len(critical)} critical) remain."
\end{lstlisting}

\section*{Exercises}
\begin{enumerate}
  \item Run \texttt{pip-audit} on your project and triage any findings.
  \item Add validation for a CLI command that currently trusts user input.
  \item Design a security review checklist for third-party libraries.
  \item Implement hash checking for dependency installation using \texttt{pip --require-hashes}.
  \item Build a small demo exploiting a path traversal bug, then patch it as in
  Listing~\ref{lst:safe_path}.
  \item Extend Listing~\ref{lst:security_hashes} so it also validates a digital signature
  before trusting a binary package.
\end{enumerate}

\chapter{Packaging, Distribution, and Versioning}
\section{Chapter Overview}
Packaging turns code into artifacts that other teams can trust.  Follow modern packaging
standards and disciplined versioning.

\begin{lstlisting}[caption={Recording release metadata},label={lst:packaging_overview}]
from __future__ import annotations

from dataclasses import dataclass


@dataclass(slots=True)
class Release:
    """Track release metadata for internal dashboards."""

    version: str
    commit: str
    released_by: str
\end{lstlisting}

\section{Build Real Packages}
Provide \texttt{\_\_init\_\_.py} files, export stable APIs, and publish wheels to internal
or public registries.  Modern workflows revolve around \texttt{pyproject.toml}.

\begin{lstlisting}[caption={Building distributions via Python},label={lst:packaging_build}]
from __future__ import annotations

import subprocess


def build_package() -> None:
    """Invoke the build backend and emit sdist + wheel."""
    subprocess.run(["python", "-m", "build"], check=True)
\end{lstlisting}

\section{Entry Points}
\begin{lstlisting}[caption={Console script entry point},label={lst:entry_point}]
[project.scripts]
bill = "billing.cli:main"
\end{lstlisting}

\section{Versioning Discipline}
Adopt semantic versioning: MAJOR for breaking changes, MINOR for new features, PATCH for
bug fixes.  Tag releases in Git and maintain a human-readable changelog.

\begin{lstlisting}[caption={Bumping semantic versions programmatically},label={lst:packaging_version}]
from __future__ import annotations


def bump(version: str, level: str) -> str:
    """Return a new semver string at the requested level."""
    major, minor, patch = map(int, version.split("."))
    match level:
        case "major":
            major += 1
            minor = 0
            patch = 0
        case "minor":
            minor += 1
            patch = 0
        case "patch":
            patch += 1
        case _:
            raise ValueError("Unknown level")
    return f"{major}.{minor}.{patch}"
\end{lstlisting}

\section{Scenario: Coordinated Release Train}
A platform team shipped a library consumed by ten services.  They instituted a release train
where every Wednesday a new minor version shipped with release notes and migration guides.
Incidents dropped because downstream teams could plan upgrades.

\begin{lstlisting}[caption={Generating release train schedules},label={lst:packaging_schedule}]
from __future__ import annotations

from datetime import date, timedelta


def release_train(start: date, *, cadence_days: int = 7) -> list[date]:
    """Return upcoming release windows."""
    return [start + timedelta(days=cadence_days * offset) for offset in range(4)]
\end{lstlisting}

\section{Summary}
Ship libraries with modern metadata, expose clear entry points, and manage semantic versions
plus changelogs so downstream consumers can upgrade confidently.

\begin{lstlisting}[caption={Summarising adoption},label={lst:packaging_summary}]
from __future__ import annotations


def summarize_consumers(consumers: list[str]) -> str:
    """Return a summary of services pinned to the latest release."""
    return f"{len(consumers)} downstream services upgraded this sprint."
\end{lstlisting}

\section*{Exercises}
\begin{enumerate}
  \item Package a small module into a wheel and install it in a clean environment.
  \item Create a changelog entry for a hypothetical breaking change.
  \item Add release automation to CI (e.g., publish to TestPyPI on tag).
  \item Define compatibility guarantees for your public API and document them.
  \item Evaluate whether your project should use namespace packages or a monorepo layout.
  \item Extend Listing~\ref{lst:packaging_version} so it updates \texttt{pyproject.toml}
  and commits the change automatically.
\end{enumerate}
\begin{lstlisting}[caption={Implementing the console script target},label={lst:packaging_entrypoint}]
from __future__ import annotations


def main() -> None:
    """Entrypoint invoked by console scripts."""
    print("Billing CLI ready")
\end{lstlisting}

\chapter{Collaboration, Git, and CI/CD}
\section{Chapter Overview}
People practices sustain technical excellence.  This chapter covers source control habits,
code reviews, and automation.

\begin{lstlisting}[caption={Tracking collaboration metrics},label={lst:collab_overview}]
from __future__ import annotations


def review_latency(hours_waited: list[int]) -> float:
    """Return the median wait time for reviews."""
    sorted_hours = sorted(hours_waited)
    mid = len(sorted_hours) // 2
    return float(sorted_hours[mid])
\end{lstlisting}

\section{Git Hygiene}
Commit early with focused changes.  Write imperative commit messages.  Rebase feature
branches onto main before opening pull requests.

\begin{lstlisting}[caption={Guarding commit messages via hook},label={lst:collab_git}]
from __future__ import annotations


def validate_commit_message(message: str) -> None:
    """Ensure commits follow imperative style."""
    if not message or message[0].islower():
        raise SystemExit("Commit message must start with an imperative verb.")
\end{lstlisting}

\section{Code Review Culture}
Treat reviews as collaborative design conversations.  Authors provide context and testing
evidence; reviewers prioritise correctness and maintainability.

\begin{lstlisting}[caption={Summarising review context automatically},label={lst:collab_review}]
from __future__ import annotations


def build_pr_template(tests: list[str], risks: list[str]) -> str:
    """Produce a review-ready description."""
    tests_block = "\n".join(f"- {test}" for test in tests)
    risks_block = "\n".join(f"- {item}" for item in risks)
    return f"## Tests\n{tests_block}\n\n## Risks\n{risks_block}"
\end{lstlisting}

\section{Automation and Hooks}

\begin{lstlisting}[caption={Pre-commit configuration for consistent tooling},label={lst:precommit}]
repos:
  - repo: https://github.com/charliermarsh/ruff-pre-commit
    rev: v0.6.8
    hooks:
      - id: ruff
      - id: ruff-format
  - repo: https://github.com/psf/black
    rev: 24.8.0
    hooks:
      - id: black
  - repo: https://github.com/pre-commit/mirrors-mypy
    rev: v1.11.0
    hooks:
      - id: mypy
\end{lstlisting}

\section{Scenario: CI as Gatekeeper}
A data team configured CI to run linters, type checkers, tests, and deployment previews.  A
regression slipped through local testing but failed in CI because the environment built from
scratch.  The automated gate prevented a broken migration from reaching production.

\begin{lstlisting}[caption={Failing fast when CI requirements are missing},label={lst:collab_ci}]
from __future__ import annotations


def ensure_checks(checks: dict[str, bool]) -> None:
    """Guard CI merges until each check reports success."""
    missing = [name for name, passed in checks.items() if not passed]
    if missing:
        raise RuntimeError(f"CI gatekeeper blocked by {missing}")
\end{lstlisting}

\section{Summary}
Healthy collaboration relies on disciplined git hygiene, respectful reviews, and CI
pipelines that run the same automated checks for everyone.

\begin{lstlisting}[caption={Summarising collaboration signals},label={lst:collab_summary}]
from __future__ import annotations


def summarize_collab(metrics: dict[str, float]) -> str:
    """Return a narrative summary for the weekly sync."""
    return (
        f"Median review wait {metrics['review_latency']}h; "
        f"{metrics['ci_pass_rate'] * 100:.0f}% CI pass rate."
    )
\end{lstlisting}

\section*{Exercises}
\begin{enumerate}
  \item Review your last five commits.  Were they cohesive?  If not, how would you split
them?
  \item Establish a rotating reviewer schedule to spread knowledge.
  \item Add \texttt{pre-commit} to your repository and enforce it in CI.
  \item Create a checklist for pull request descriptions (context, tests, screenshots).
  \item Simulate a CI outage and document manual fallback steps.
  \item Extend Listing~\ref{lst:collab_ci} so it posts a Slack alert whenever a gatekeeper
  blocks a merge.
\end{enumerate}
\begin{lstlisting}[caption={Python helper to run pre-commit},label={lst:collab_precommit}]
from __future__ import annotations

import subprocess


def run_hooks() -> None:
    """Re-run hooks locally so CI never surprises engineers."""
    subprocess.run(["pre-commit", "run", "--all-files"], check=True)
\end{lstlisting}

\chapter{Case Studies and Patterns in Real Projects}
\section{Chapter Overview}
Abstract guidance becomes tangible through stories.  This chapter captures recurring
patterns from real teams.

\begin{lstlisting}[caption={Representing case study metadata},label={lst:case_overview}]
from __future__ import annotations


def case_study_summary(name: str, impact: str) -> str:
    """Return a short descriptor for dashboards."""
    return f"{name}: {impact}"
\end{lstlisting}

\section{Billing Platform Modernisation}
A SaaS billing team inherited a cron-based script that generated invoices.  They introduced
source layout discipline (Listing~\ref{lst:layout}), wrapped calculations with typed
modules (Listing~\ref{lst:invoice}), and added pytest coverage for every pricing rule.
Structured logging (Listing~\ref{lst:logging}) helped support staff diagnose customer
complaints.  Deployment frequency increased because engineers trusted their safety nets.

\begin{lstlisting}[caption={Capturing invoice recalculation state},label={lst:case_billing}]
from __future__ import annotations

from dataclasses import dataclass


@dataclass(slots=True)
class InvoiceProjection:
    """Aggregate data used during the refactor."""

    customer_id: str
    subtotal: float
    taxes: float


def total_due(projection: InvoiceProjection) -> float:
    """Return the amount customers see post-refactor."""
    return projection.subtotal + projection.taxes
\end{lstlisting}

\section{Data Pipeline Hardening}
An analytics squad maintained a nightly ingestion job similar to
Listing~\ref{lst:untestable}.  After a costly outage, they refactored into injectable
components, added property-based tests to validate CSV parsing, and enforced configuration
loading via \texttt{Settings}.  CI now spins up ephemeral storage backends for integration
tests, and performance monitoring from Chapter~10 alerts engineers when runtimes drift.

\begin{lstlisting}[caption={Injectable pipeline components},label={lst:case_pipeline}]
from __future__ import annotations

from collections.abc import Iterable


def transform_rows(rows: Iterable[str], *, delimiter: str = ",") -> list[list[str]]:
    """Ensure pipeline logic works with dependency injection."""
    return [row.split(delimiter) for row in rows]
\end{lstlisting}

\section{Future Work}
\% TODO: Add a machine-learning-focused case study covering experiment tracking,
reproducible data, and model deployment.

\begin{lstlisting}[caption={Scaffolding future case studies},label={lst:case_future}]
from __future__ import annotations


def placeholder_case(title: str) -> dict[str, str]:
    """Reserve space for future patterns."""
    return {"title": title, "status": "draft"}
\end{lstlisting}

\section{Summary}
Case studies show how layered practices—structure, typing, testing, observability—combine to
deliver predictable outcomes, and they highlight where future research is needed.

\begin{lstlisting}[caption={Summarising lessons learned},label={lst:case_summary}]
from __future__ import annotations


def summarize_lessons(lessons: list[str]) -> str:
    """Produce a concise statement for leadership debriefs."""
    return "; ".join(lessons[:3])
\end{lstlisting}

\section*{Exercises}
\begin{enumerate}
  \item Interview another team about their most successful refactor.  Map their steps to the
  practices in earlier chapters.
  \item Create a mini case study for your project, documenting before/after metrics.
  \item Design a "playbook" template for capturing future case studies.
  \item Identify a legacy system in your organisation and propose the first three incremental
  steps toward modernisation.
  \item Reflect on a failed initiative.  Which missing practices contributed to the outcome?
  \item Implement code based on Listing~\ref{lst:case_pipeline} that validates CSV headers
  before transformation to avoid future ingestion incidents.
\end{enumerate}
\begin{lstlisting}[caption={Capturing invoice recalculation state},label={lst:case_billing}]
from __future__ import annotations

from dataclasses import dataclass


@dataclass(slots=True)
class InvoiceProjection:
    """Aggregate data used during the refactor."""

    customer_id: str
    subtotal: float
    taxes: float


def total_due(projection: InvoiceProjection) -> float:
    """Return the amount customers see post-refactor."""
    return projection.subtotal + projection.taxes
\end{lstlisting}
\begin{lstlisting}[caption={Injectable pipeline components},label={lst:case_pipeline}]
from __future__ import annotations

from collections.abc import Iterable


def transform_rows(rows: Iterable[str], *, delimiter: str = ",") -> list[list[str]]:
    """Ensure pipeline logic works with dependency injection."""
    return [row.split(delimiter) for row in rows]
\end{lstlisting}


\appendix

\chapter{Checklists, Templates, and Further Reading}
\section{Chapter Overview}
The appendix collects reusable assets that keep teams aligned.

\begin{lstlisting}[caption={Representing checklist ownership},label={lst:appendix_overview}]
from __future__ import annotations


def checklist_owner(name: str) -> str:
    """Map checklist names to owners."""
    return f"{name} owner: platform team"
\end{lstlisting}

\section{Launch Checklist}
Before shipping, confirm that formatting, linting, typing, and tests pass locally and in CI.
Review documentation updates, rotate credentials if needed, and run deployment rehearsals.

\begin{lstlisting}[caption={Evaluating launch readiness programmatically},label={lst:appendix_launch}]
from __future__ import annotations


def launch_ready(checks: dict[str, bool]) -> bool:
    """Return True when every checklist item passes."""
    return all(checks.values())
\end{lstlisting}

\section{Template Repository}
Maintain a starter repo containing \texttt{pyproject.toml}, \texttt{noxfile.py}, CI
workflows, and documentation scaffolding.  New services can fork it to inherit best
practices.

\begin{lstlisting}[caption={Scaffolding a new repository from a template},label={lst:appendix_template}]
from __future__ import annotations

import shutil
from pathlib import Path


def scaffold_from_template(template: Path, destination: Path) -> None:
    """Copy the template repo into a new project directory."""
    shutil.copytree(template, destination, dirs_exist_ok=True)
\end{lstlisting}

\section{Further Reading}
Recommended resources include Brett Slatkin's \emph{Effective Python}, Hynek Schlawack's
\emph{Solid Python}, the official \emph{Python Packaging User Guide}, and the \emph{Twelve
Factor App}.  Follow tool maintainers (Black, Ruff, Pytest, Mypy) to stay current.

\begin{lstlisting}[caption={Generating reading lists},label={lst:appendix_reading}]
from __future__ import annotations


def reading_plan(topics: list[str]) -> list[str]:
    """Pair topics with recommended books."""
    return [f"{topic}: Effective Python (2nd Ed.)" for topic in topics]
\end{lstlisting}

\section{Summary}
Standardised checklists, templates, and curated resources ensure every new initiative starts
with the same proven guardrails.

\begin{lstlisting}[caption={Summarising appendix assets},label={lst:appendix_summary}]
from __future__ import annotations


def summarize_assets(counts: dict[str, int]) -> str:
    """Provide a short statement listing asset counts."""
    return ", ".join(f"{kind}: {count}" for kind, count in counts.items())
\end{lstlisting}

\section*{Exercises}
\begin{enumerate}
  \item Build a personalised launch checklist and compare it with this appendix.
  \item Create a template repository for your organisation and solicit feedback.
  \item Curate a reading list tailored to your team's domain (web, data, ML).
  \item Run a brown-bag session where each engineer shares one tool configuration tip.
  \item Evaluate whether your onboarding guide references every relevant checklist; update
  as needed.
  \item Extend Listing~\ref{lst:appendix_launch} so it prints which checklist item failed
  when the launch is blocked.
\end{enumerate}


\backmatter

\end{document}
